To solve the compatibility condition in Section 8.2, we must formalize how the global "truth" of the substrate's evolution maps onto the "shadow" of local observation. The following LaTeX subsection provides the rigorous bound and its physical interpretation, maintaining the established nomenclature and stylistic conventions.

```latex
\subsection*{8.2.1 The Coherence Commutator and Stability Bounds}

\begin{FormalBox}{The compatibility inequality}
To quantify the fidelity of the emergent local physics, we define the \textbf{Stability Error} $\epsilon$ as the operator-norm distance between the projected global evolution and the effective local evolution. 

Let $\|\cdot\|_{\infty}$ denote the spectral operator norm. The compatibility condition is satisfied if there exists an effective local operator $\Floquet_{\mathrm{loc}}$ such that:
\[
\epsilon = \left\| \Proj \Floquet - \Floquet_{\mathrm{loc}} \Proj \right\|_{\infty} \leq \Gamma \cdot \exp(-\mathcal{C})
\]
where:
\begin{itemize}
    \item $\epsilon$ is the \textbf{Commutation Residue} (Stability Error).
    \item $\Gamma$ is a normalization constant related to the spectral gap of $\Htot$.
    \item $\mathcal{C}$ represents the \textbf{Coherence Volume} of the projected sector.
\end{itemize}
Essentially, the error $\epsilon$ must be exponentially small relative to the complexity of the sector to maintain a classical-looking history.
\end{FormalBox}

\begin{MeaningBox}{Interpreting the Thresholds}
The value of $\epsilon$ determines the "reality-status" of the emergent spacetime $\Emerg$:
\begin{itemize}[leftmargin=*,itemsep=2pt]
    \item \textbf{The Coherent Regime ($\epsilon \to 0$):} The local sector is "sealed." $\Floquet_{\mathrm{loc}}$ perfectly predicts the next state of the projection. Here, causality is perceived as absolute.
    \item \textbf{The Stochastic Drift ($\epsilon < \epsilon_c$):} Local laws are mostly reliable, but "vacuum fluctuations" or "intrinsic noise" appear. This noise is actually the substrate's global dynamics leaking through the imperfect projection.
    \item \textbf{The Decoherence Threshold ($\epsilon \approx \epsilon_c$):} The "Critical Limit." When the error crosses a threshold $\epsilon_c$, the local history can no longer be modeled as a continuous sequence. Spacetime "shatters" as the projection $\Proj$ fails to filter out the global interference.
\end{itemize}
\end{MeaningBox}

\begin{TestBox}{Empirical signatures of $\epsilon > 0$}
If the present framework is correct, we should look for \textbf{Non-Local Leakage}:
\begin{enumerate}
    \item \textbf{Anomalous Jumps:} Sudden changes in local state variables that have no local causal antecedent in $\Emerg$, representing a direct $\Floquet \to \Proj$ interaction that bypasses $\Floquet_{\mathrm{loc}}$.
    \item \textbf{Geometric Dissolution:} In high-energy or high-complexity regimes, the "metric" (derived from $\Emerg$) should exhibit fluctuations proportional to the increase in $\epsilon$.
\end{enumerate}
\end{TestBox}

\begin{ChildBox}{The Movie Projector}
Imagine a movie projector (the substrate) showing a film on a screen (the projection). If the screen is perfectly still, the characters in the movie can follow their own rules (local laws). $\epsilon$ is like the screen shaking. If it shakes just a little, the movie looks a bit blurry. If it shakes too much, you can't see the movie anymore—you just see the light from the projector.
\end{ChildBox}
```

\documentclass[11pt]{article}

% Packages
\usepackage{amsmath, amssymb, amsfonts}
\usepackage{physics}
\usepackage{hyperref}
\usepackage{bm}
\usepackage{geometry}
\usepackage{amsmath, amsfonts, amssymb, amsthm}
\usepackage{graphicx}
\usepackage{authblk}
\usepackage{hyperref}
\usepackage{bm}
\usepackage{cite}
\usepackage{mathtools}
\usepackage{xcolor}
\usepackage{tcolorbox}
\usepackage{titling}

\geometry{
  a4paper,
  margin=0.8in,
  top=1in
}

% Move title closer to the top
\setlength{\droptitle}{-5em}  % try -1em, -2em, -3em etc.
\setlength{\skip\footins}{2em}

\title{\large \textbf{Cross-Coherence Communication:}\\
\textbf{A Unified Physical and Biological Theory of Non-Local Signaling in Living Systems}}
\begin{document}
\author{Julien Steff}
\affil[1]{Science Coherence Institute}

\date{}

\maketitle

\begin{abstract}
Living organisms exhibit forms of long-range coordination---from plant electrophysiology to human intuition---that remain unexplained by classical biochemical or electromagnetic signaling. Here we develop a unified framework in which such interactions arise from \emph{cross-coherence coupling} between biological systems. We formalize this phenomenon using a substrate-level coherence measure and derive a non-local connectivity functional, the \emph{resonant field} $R_{AB}$, which predicts interaction strength from phase alignment and coherence density rather than spatial proximity. 

We then integrate molecular evidence showing that certain plant- and fungi-derived compounds, including tryptamines, phenethylamines, and \(\beta\)-carbolines, modulate neural phase structure in ways that increase cross-organism coherence. These molecules act as \emph{cross-coherence interfaces} capable of synchronizing human neural oscillations with plant electrophysiological rhythms. 

Combining physical modeling and biological mechanisms, we propose a reproducible scientific basis for non-local organism-to-organism communication. The framework yields clear experimental predictions across electrophysiology, neurobiology, and coherence spectroscopy, offering a rigorous path to evaluating telepathic-like phenomena within an evidence-based scientific paradigm.
\end{abstract}

\section{Introduction}

Interactions within and between living systems are typically explained through classical mechanisms: chemical gradients, molecular diffusion, hormonal signaling, and electromagnetic fields. However, increasing empirical work in electrophysiology, bioelectricity, and neurobiology has revealed coordinated behaviors that propagate faster and at longer distances than these mechanisms allow.

Plants exhibit long-distance electrical signaling across entire networks of roots and leaves. Human brains exhibit coordinated activity between distant regions without direct synaptic connection. Inter-species interactions, including human--plant correlations, have been reported in contexts involving altered neural states.

These observations motivate the search for a more general coupling mechanism. The present paper develops such a mechanism based on \emph{coherence-mediated non-local interactions}. We unify two classes of evidence:

\begin{enumerate}
    \item A physically grounded coherence-coupling model describing how systems with aligned internal phase structure become non-locally correlated.
    \item Molecular and electrophysiological data showing that certain naturally occurring compounds act as coherence synchronizers across biological species.
\end{enumerate}

By combining these perspectives, we construct a single theoretical and experimental framework explaining how biological systems may exchange information at a distance.

\section{Coherence and Phase Structure in Biological Systems}
Biological systems maintain local and global order far from thermodynamic equilibrium. This order manifests in oscillations, phase relationships, metabolic rhythms, and bioelectric potentials. Coherence---the degree of phase-ordered behavior---plays a fundamental role in:

\begin{itemize}
    \item neuronal synchronization,
    \item plant calcium and electrical waves,
    \item cardiac and respiratory entrainment,
    \item collective cell migration and morphogenesis.
\end{itemize}

Let \( C(x,t) \in [0,1] \) denote the local coherence density of a system. This quantity measures the proportion of degrees of freedom that participate in a shared oscillatory or dynamical phase.

We also introduce a local phase field \( \phi(x,t) \), representing the phase of oscillatory participation at each point.

These fields allow us to define cross-system coupling in an explicit mathematical framework.

\section{Resonant Field Formalism}

Consider two biological systems \(A\) and \(B\). Their effective non-local interaction strength is modeled by the \emph{resonant field functional}:
\begin{equation}
    R_{AB}(t) = \int_{\Sigma} 
    \sqrt{C_A(x,t) C_B(x,t)}\,
    e^{i [\phi_A(x,t) - \phi_B(x,t)]}\,
    d^3 x.
\end{equation}

The magnitude 
\begin{equation}
    |R_{AB}| = \left| R_{AB}(t) \right|
\end{equation}
quantifies the effective information-exchange capacity between the systems.

This framework predicts:
\begin{enumerate}
    \item Strong coupling when phase differences are small.
    \item Enhanced coupling when both systems exhibit high coherence densities.
    \item Independence from spatial separation when coherence is sustained.
\end{enumerate}

In this view, information transfer does not require classical radiation or chemical diffusion. It arises from cross-system phase alignment.

\section{Biological Coherence Across Species}

Plants, fungi, and animals exhibit oscillatory processes with characteristic frequencies:
\begin{itemize}
    \item plant calcium waves (0.01--0.1 Hz),
    \item plant electrophysiological rhythms (0.1--5 Hz),
    \item mammalian neural oscillations (0.5--200 Hz),
    \item metabolic and circadian rhythms.
\end{itemize}

Although these frequencies differ, they can synchronize through mechanisms analogous to oscillator entrainment. Crucially, many plant-produced molecules directly modulate mammalian neural oscillations, effectively bridging this frequency gap.

\section{Molecular Cross-Coherence Interfaces}

Several classes of naturally occurring compounds have molecular structures optimized for interacting with neural receptors involved in phase coordination. Prominent examples include:

\begin{itemize}
    \item tryptamines (e.g., DMT),
    \item psilocybin-derived compounds,
    \item phenethylamines (e.g., mescaline),
    \item \(\beta\)-carbolines (e.g., harmine, harmaline).
\end{itemize}

When these molecules bind to neural receptors, they alter the brain's internal phase structure by modulating:

\begin{enumerate}
    \item thalamocortical oscillatory loops,
    \item cross-frequency coupling,
    \item network-level coherence.
\end{enumerate}

This increases the brain's susceptibility to coupling with external oscillatory systems.

We propose that these molecules function as \emph{biological coherence interfaces}, enabling temporary synchronization between humans and plants.

\section{Cross-Coherence Communication}

Integrating the physical and molecular components, we define \emph{cross-coherence communication} as:
\begin{quote}
    information exchange between biological systems mediated by coherence alignment rather than classical signaling pathways.
\end{quote}

The resonant field formalism predicts that two organisms with aligned phase structures can exhibit:

\begin{itemize}
    \item correlated electrophysiological activity,
    \item non-local response patterns,
    \item shared oscillatory signatures,
    \item information transfer measurable through coherence spectra.
\end{itemize}

Psychoactive plant-derived molecules enhance this communication channel by increasing neural coherence and aligning its frequencies with plant rhythms.

\section{Experimental Predictions}

The theory makes several testable predictions:

\subsection{Human–Plant Coherence Synchronization}
Simultaneous recording of:
\begin{itemize}
    \item human EEG,
    \item plant electrophysiology (bioelectrical potentials),
\end{itemize}
should reveal increased cross-coherence under conditions where neural phase structure is altered by specific plant-derived compounds.

\subsection{Non-Radiative Correlations}
Correlations should persist under shielding that blocks:
\begin{itemize}
    \item electromagnetic fields,
    \item acoustic vibrations,
    \item chemical diffusion.
\end{itemize}

\subsection{Coherence Threshold Effects}
Cross-system correlation should exhibit sharp transitions when:
\begin{itemize}
    \item coherence density exceeds a critical threshold,
    \item phase locking is sustained for a minimum duration.
\end{itemize}

\section{Discussion}

The framework unifies diverse observations:

\begin{itemize}
    \item long-range biological electrical signaling,
    \item synchronized oscillations between individuals,
    \item phenomena historically labeled ``telepathic'' or non-local,
    \item the biological role of psychoactive compounds.
\end{itemize}

By grounding these phenomena in coherence theory, we eliminate the need for invoking anomalous forces or metaphysical assumptions.

\section{Conclusion}

We have presented a unified physical and biological theory of non-local communication in living systems. The resonant field formalism provides a quantitative measure of coherence-mediated coupling, while molecular mechanisms offer a pathway for cross-species coherence alignment.

This integrated model yields clear experimental predictions, opening new avenues for research in electrophysiology, molecular neuroscience, and collective biological dynamics.

\section*{References}
\vspace{-1em}
\begin{itemize}
    \item Placeholder for references.
\end{itemize}

\end{document}

\documentclass[11pt]{article}

% Packages
\usepackage{amsmath, amssymb, amsfonts}
\usepackage{physics}
\usepackage{hyperref}
\usepackage{bm}
\usepackage{geometry}
\usepackage{amsmath, amsfonts, amssymb, amsthm}
\usepackage{graphicx}
\usepackage{authblk}
\usepackage{hyperref}
\usepackage{bm}
\usepackage{cite}
\usepackage{mathtools}
\usepackage{xcolor}
\usepackage{tcolorbox}
\usepackage{titling}

\geometry{
  a4paper,
  margin=0.8in,
  top=1in
}
\begin{document}


\section{Coherence Fields and Their Physical Manifestations}
\label{sec:CoherenceFields}
\noindent
\textbf{Transition from the Coherence Manifold to Spacetime.} 
The coherence scalar field defined in Section~5.7, $C(p)$, lives on the 
thermodynamic–informational manifold $\mathcal{M}_C$ whose coordinates $p^a$ 
encode the substrate’s microscopic coherence state. Physical phenomena, however, 
do not unfold on $\mathcal{M}_C$ itself but on the emergent spacetime manifold 
$\mathcal{M}_{\mathrm{phys}}$ with coordinates $x^\mu$. Through coarse-graining and 
the thermodynamic–geometric limit developed in §5.4, each spacetime point $x$ 
corresponds to an equivalence class of coherence states $p$, inducing a natural 
projection $\pi:\mathcal{M}_C \!\to\! \mathcal{M}_{\mathrm{phys}}$. The macroscopic 
coherence field used in this section is therefore the pushforward of the microscopic 
one,
\[
C(x) = C\!\big(p(x)\big),
\]
representing the same underlying coherence structure expressed now as a field over 
spacetime. What follows in this section is the analysis of how this spacetime-indexed 
coherence field manifests as effective physical forces, quantum correlations, and 
cosmological background properties.


In the previous section we developed the quantum thermodynamic geometry of the
Time--Crystalline Quantum Substrate (TCQS), identifying the internal geometric
quantities---the coherence scalar $C(x)$, the coherence metric
$g^{(\mathrm{coh})}_{ij}$, and the thermodynamic coherence potential
$\Phi_{\mathrm{coh}}$---that govern the local structure and dynamics of the
substrate.  

The purpose of this section is to show how these geometric quantities give rise
to \emph{physical fields} in the observable universe. Specifically, we show that:
\begin{enumerate}
    \item gravitational effects emerge from \emph{spatial gradients of the
    coherence field};
    \item quantum entanglement corresponds to \emph{resonantly protected
    correlations} within the substrate;
    \item dark energy corresponds to the \emph{global baseline coherence density}
    maintained by the time-crystalline update rule;
    \item transitions between microscopic and macroscopic coherence describe the
    onset of decoherence, classicality, and force emergence;
    \item these coherence-based physical fields provide the natural conceptual
    and mathematical bridge to the emergence of gravitational dynamics that will
    be derived in the next section.
\end{enumerate}

This section therefore plays a central role in the logical structure of the
TCQS: it maps the intrinsic geometry of coherence to the effective fields
experienced as physical forces, correlations, and cosmological backgrounds.


% ------------------------------------------------------------
\subsection{6.1 \; The Coherence Scalar Field as a Physical Quantity}

The coherence scalar $C(x)$ introduced in Section~\ref{sec:QThermoGeometry} is a
real-valued field defined over the substrate:
\begin{equation}
    C : \mathcal{M} \to \mathbb{R}, \qquad x \mapsto C(x),
\end{equation}
where $\mathcal{M}$ is the emergent spacetime manifold.

$C(x)$ measures the degree of order, phase synchronicity, and internal
time-crystal alignment of the substrate at location $x$. In physical terms,
$C(x)$ represents:
\begin{itemize}
    \item the \emph{local coherence density};
    \item the \emph{stability} of the underlying time-crystalline update cycle;
    \item the \emph{resistance to decoherence};
    \item the \emph{effective free-energy deficit} relative to the global
    coherence baseline.
\end{itemize}

Because $C(x)$ is a scalar, its spatial variation is encoded in its gradient:
\begin{equation}
    \nabla_i C(x) = \partial_i C(x).
\end{equation}

This gradient encodes how the vacuum's internal order varies in space and thus
serves as the fundamental physical quantity from which emergent forces arise.


% ------------------------------------------------------------
\subsection{6.2 \; Coherence Gradients and Effective Forces}

The key physical postulate of the TCQS is that \emph{matter locally disturbs the
underlying coherence field}. A mass-energy density $\rho(x)$ induces a local
reduction:
\begin{equation}
    C(x) = C_0 - \delta C[\rho](x),
\end{equation}
where $C_0$ is the global baseline coherence.

At leading order, matter produces \emph{spatial gradients} in $C$, and these
gradients act as effective force fields in the emergent macroscopic limit.

We define the effective force vector field:
\begin{equation}
    \vec{F}(x) = - \nabla C(x).
\end{equation}

Later we will show that this vector reproduces Newtonian gravitation in the
appropriate regime and its relativistic generalization in the geometric limit.
For now, it suffices to note that:
\begin{itemize}
    \item matter reduces coherence locally;
    \item this reduction generates a slope in $C$;
    \item systems are drawn ``downhill'' along the coherence gradient;
    \item the resulting dynamics mimic gravitational attraction.
\end{itemize}

Thus, the coherence gradient is the \emph{physical precursor} to gravitational
dynamics derived in Section~7.


% ------------------------------------------------------------
\subsection{6.3 \; Resonant Correlation and Quantum Entanglement}

In the TCQS, quantum entanglement arises from \emph{resonantly protected
coherence links} between distant regions of the substrate.

Given two localized excitations $A$ and $B$, define the resonance correlation
functional:
\begin{equation}
    R_{AB} = 
    \frac{\langle \psi_A(x), \psi_B(y) \rangle_{C}}{
        \sqrt{\|\psi_A\|\|\psi_B\|}
    },
\end{equation}
where the inner product $\langle\cdot,\cdot\rangle_C$ is weighted by the local
coherence density.

We say that $A$ and $B$ are \emph{entangled} if:
\begin{equation}
    R_{AB} \ge R_{\mathrm{crit}},
\end{equation}
where $R_{\mathrm{crit}}$ is a coherence-protection threshold determined by the
substrate's update cycle.

This framework naturally explains:
\begin{itemize}
    \item the nonlocality of entanglement (due to substrate-wide coherence);
    \item the fragility of entanglement under decoherence (loss of coherence
    protection);
    \item the absence of superluminal signalling (resonance collapse yields no
    transferable signal).
\end{itemize}

Entanglement thus corresponds to protected resonance pathways within the
coherence geometry, rather than spatially mediated interactions.


% ------------------------------------------------------------
\subsection{6.4 \; Global Baseline Coherence and Dark Energy}

The time-crystalline update rule of the substrate enforces a persistent
non-zero minimum coherence density $C_0$ across all scales. This baseline is
\emph{globally uniform} due to the periodic, self-correcting Floquet-like
dynamics.

We identify this constant background with the effective cosmological constant:
\begin{equation}
    \Lambda \equiv C_0.
\end{equation}

In the TCQS, therefore:
\begin{itemize}
    \item dark energy does not arise from a vacuum energy density of quantum
    fields;
    \item instead, it arises from the \emph{self-maintained global coherence}
    inherent to the substrate;
    \item cosmic acceleration is the large-scale relaxation of the universe
    toward the globally enforced coherence baseline.
\end{itemize}

This provides a fully non-singular, internally consistent account of
cosmological acceleration.


% ------------------------------------------------------------
\subsection{6.5 \; Coherence Bandwidths and the Microscopic--Macroscopic Transition}

The transition from quantum behaviour to classicality is described in the TCQS
by the \emph{local coherence bandwidth} $\Delta C(x)$.

A system exhibits:
\begin{itemize}
    \item \textbf{quantum behaviour} when $\Delta C$ is sufficiently high to
    support resonant pathways (entanglement, coherence, superposition);
    \item \textbf{classical behaviour} when $\Delta C$ falls below the resonance
    threshold due to environmental interaction.
\end{itemize}

This framework provides:
\begin{itemize}
    \item a geometric interpretation of decoherence;
    \item a substrate-based explanation for measurement collapse (local
    coherence locking);
    \item a unified view of microscopic and macroscopic physics in terms of
    coherence bandwidth dynamics.
\end{itemize}


% ------------------------------------------------------------
\subsection{6.6 \; From Coherence Fields to Gravitational Dynamics}

Having established that the coherence gradient produces an effective force, we
now prepare the conceptual bridge to the full derivation of gravitational
dynamics in the next section.

The crucial insight is that the coherence field satisfies a Poisson-like
equation:
\begin{equation}
    \nabla^2 C(x) = 4\pi G_{\mathrm{eff}} \, \rho(x),
\end{equation}
where $G_{\mathrm{eff}}$ is an emergent gravitational coupling determined by the
substrate's geometric and thermodynamic parameters.

This equation is the \emph{coherence analogue} of the Newtonian gravitational
Poisson equation and arises naturally from the perturbative dynamics described
earlier.

In Section~7 we will show that, in the hydrodynamic limit of the thermodynamic
coherence geometry, this leads to:
\begin{itemize}
    \item Newtonian gravitation in the weak-field limit,
    \item an effective spacetime metric equivalent to general relativity in the
    geometric limit,
    \item the absence of curvature singularities due to finite coherence bounds.
\end{itemize}

Thus, the coherence fields introduced in this section constitute the essential
intermediate structure linking microgeometry (Section~5) to macroscopic gravity
(Section~7).
\end{document}

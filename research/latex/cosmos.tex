% =====================================================================================
% SCIENCE OF COHERENCE — COSMOS SECTION (Website + Paper Base)
% Single-file LaTeX master you can feed to Gemini to generate website content pages.
% Style: rigorous + philosophical + child-intuitive explanations, in parallel.
% =====================================================================================

\documentclass[11pt]{article}

% --------------------------
% Packages
% --------------------------
\usepackage[a4paper,margin=1in]{geometry}
\usepackage{microtype}
\usepackage{lmodern}
\usepackage[T1]{fontenc}
\usepackage{hyperref}
\usepackage{xcolor}
\usepackage{amsmath,amssymb,mathtools}
\usepackage{physics}
\usepackage{enumitem}
\usepackage{tcolorbox}
\usepackage{graphicx}

\hypersetup{
  colorlinks=true,
  linkcolor=cyan!55!black,
  urlcolor=cyan!55!black,
  citecolor=cyan!55!black
}

% --------------------------
% Aesthetic + semantic macros
% --------------------------
\newcommand{\SiteTitle}{Science of Coherence — Cosmos}
\newcommand{\ModelName}{the time-crystalline holographic universe} % introduced once; then refer implicitly
\newcommand{\Substrate}{S}          % the only state space (your constraint)
\newcommand{\Proj}{\Lambda}         % coherence / projection operator
\newcommand{\Emerg}{\Omega}         % emergent effective operator
\newcommand{\Integr}{\Delta}        % recursive self-modeling / integration operator
\newcommand{\Htot}{H_{\mathrm{tot}}}% global generator (avoid confusion with Hilbert space H)
\newcommand{\Floquet}{\mathcal{F}}  % Floquet update / stroboscopic evolution

% --------------------------
% Box styles (Website-friendly chunks)
% --------------------------

\tcbset{
  colback=black!96!white,
  colframe=cyan!30!black,
  coltitle=white,
  coltext=white,        % <-- IMPORTANT: body text color
  colupper=white,       % <-- also ensures upper text is white
  fonttitle=\bfseries,
  boxrule=0.8pt,
  arc=3mm,
  left=3mm,right=3mm,top=2mm,bottom=2mm
}


\newtcolorbox{PortalBox}[1]{
  title={#1},
  colframe=cyan!35!black
}

\newtcolorbox{FormalBox}[1]{
  title={Formal layer — #1},
  colframe=violet!35!black
}

\newtcolorbox{MeaningBox}[1]{
  title={Symbolic \& philosophical layer — #1},
  colframe=yellow!40!black
}

\newtcolorbox{ChildBox}[1]{
  title={Explain it like I’m 10 — #1},
  colframe=green!35!black
}

\newtcolorbox{TestBox}[1]{
  title={What would count as evidence? — #1},
  colframe=orange!40!black
}

\newtcolorbox{WebsiteBox}[1]{
  title={Website copy block — #1},
  colframe=cyan!25!black
}

% --------------------------
% Document
% --------------------------
\begin{document}
\begin{center}
{\LARGE \textbf{\SiteTitle}}\\[2mm]
{\large \texttt{COSMOS} — Unified physics model: crystalline time, emergent space, holographic coherence}\\[2mm]
{\small (Single-source LaTeX for paper-grade text \& website-ready copy blocks)}
\end{center}

\vspace{4mm}

% =====================================================================================
\section*{0. Page Identity (Website Hero)}
% =====================================================================================

\begin{WebsiteBox}{Hero headline + subheadline}
\textbf{Headline:} COSMOS\\
\textbf{Subheadline:} A coherent route from a single underlying state space to effective spacetime, fields, and histories.\\
\textbf{One-line promise:} The universe can be modeled as a globally coherent generator whose periodic self-updates yield stable emergent structure.
\end{WebsiteBox}

\begin{PortalBox}{What COSMOS covers}
This section is the \emph{physics door}: a compact, test-oriented framework for how time-periodic global dynamics can produce (i) stable recurrence, (ii) classical-looking sectors, and (iii) an emergent geometry—without assuming spacetime as fundamental.
\end{PortalBox}

\begin{MeaningBox}{Tone constraint}
Throughout, the model is named once (above) and then referred to implicitly: \emph{the substrate}, \emph{the present framework}, \emph{this construction}. No acronym treadmill.
\end{MeaningBox}

% =====================================================================================
\section*{1. One-Screen Summary (for visitors)}
% =====================================================================================

\begin{WebsiteBox}{The minimal story}
\begin{itemize}[leftmargin=*,itemsep=2pt]
  \item \textbf{Only state space:} the substrate $\Substrate$ is the sole arena of configurations.
  \item \textbf{Global generator:} $\Htot$ governs coherent evolution on $\Substrate$.
  \item \textbf{Local appearance:} a projection $\Proj$ yields contextual sectors (what any ``local observer'' can access).
  \item \textbf{Emergent physics:} an effective operator $\Emerg$ organizes the projected sector into spacetime-like histories and field-like dynamics.
  \item \textbf{Recurrence:} a Floquet-like update $\Floquet$ captures periodic self-consistency (time-crystalline recurrence).
\end{itemize}
\end{WebsiteBox}

\begin{ChildBox}{A picture in words}
Imagine reality as one gigantic musical instrument. The whole instrument has its own rule for vibrating ($\Htot$). You never hear the whole thing at once—your ears sample a slice ($\Proj$). From those slices you infer a story: space, time, particles, forces ($\Emerg$). If the instrument repeats a stable rhythm, you get a ``cosmic beat'' (recurrence).
\end{ChildBox}

% =====================================================================================
\section*{2. Core Objects (clean ontology)}
% =====================================================================================

\subsection*{2.1 The substrate is the only state space}
\begin{FormalBox}{State space axiom}
We posit a single state space $\Substrate$ (the substrate). All physical descriptions—microscopic, macroscopic, and emergent—are descriptions \emph{of} $\Substrate$ or of structured restrictions of $\Substrate$.
\[
\textbf{Axiom (Only arena):}\qquad \mathcal{X} \equiv \Substrate.
\]
No additional fundamental arena (no independent spacetime manifold, no separate ``observer space'').
\end{FormalBox}

\begin{MeaningBox}{What this is buying you}
This is an ontological austerity move: one arena, many faces. You avoid \emph{dual bookkeeping} (``matter on spacetime'' plus ``spacetime'' as separate primitive). Everything is a pattern in the substrate; geometry is a derived organizational principle, not a starting ingredient.
\end{MeaningBox}

\begin{ChildBox}{Like LEGO}
There is only one kind of LEGO pile. Space, time, particles, and even clocks are different ways the LEGO pile can be arranged and read.
\end{ChildBox}

\subsection*{2.2 The global generator $\Htot$}
\begin{FormalBox}{Global dynamics}
$\Htot$ is the globally coherent generator acting on the substrate.
\[
\Htot:\Substrate \to \Substrate,
\qquad
\text{(generator of coherent evolution on the substrate).}
\]
When we need a time-evolution operator, we will write a unitary (or generalized) propagator $U(t)$ induced by $\Htot$:
\[
U(t)=\exp\!\left(-\frac{i}{\hbar}\Htot t\right)
\quad \text{(symbolic placeholder; exact form may differ).}
\]
\end{FormalBox}

\begin{MeaningBox}{Why call it ``global''}
Global means: it is not \emph{defined} by local measurements. Local physics is a \emph{shadow} (projection) of the global evolution, not the source of it. This is the same directional arrow as holography: boundary/sector descriptions are derivative encodings of a deeper rule.
\end{MeaningBox}

\begin{ChildBox}{The universe’s rulebook}
$\Htot$ is the rule for how the entire LEGO pile can change—before anyone zooms in and says “this part is a particle.”
\end{ChildBox}

\subsection*{2.3 The projection/coherence operator $\Proj$}
\begin{FormalBox}{Local accessibility as projection}
Local accessibility and classical-looking structure arise from a projection operator $\Proj$ that restricts global substrate configurations to partial contextual representations.
\[
\Proj:\Substrate \to \Substrate_{\mathrm{loc}}
\]
where $\Substrate_{\mathrm{loc}}$ denotes an accessible sector (a derived state space, not fundamental).
\end{FormalBox}

\begin{MeaningBox}{The philosophical punch}
$\Proj$ is the principle of \emph{partial view}. Reality is whole; experience and instruments are slices. A slice can be internally coherent and lawful while still being incomplete. This is also where “separation” is manufactured: locality is not a primitive, it is a stable pattern of restricted access.
\end{MeaningBox}

\begin{ChildBox}{Flashlight in a cave}
You’re in a huge cave (the substrate). You only see what your flashlight illuminates (projection). The cave is still there outside the beam.
\end{ChildBox}

\subsection*{2.4 The emergent effective operator $\Emerg$}
\begin{FormalBox}{Emergent organization of histories}
The effective operator $\Emerg$ encodes the emergent spacetime, fields, and histories accessible at the projected level:
\[
\Emerg \equiv \Emerg(\Htot,\Proj;\text{coherence constraints}).
\]
Interpretation: $\Emerg$ is induced by the global generator \emph{as seen through} the projection, under constraints that select stable coherent sectors.
\end{FormalBox}

\begin{MeaningBox}{Where ``spacetime'' lives}
Spacetime is a \emph{bookkeeping layer}: a stable way to index correlations inside projected sectors. The framework does not deny spacetime— it demotes spacetime from “God-given stage” to “effective coordinate system that works extremely well.”
\end{MeaningBox}

\begin{ChildBox}{A map is not the land}
$\Emerg$ is like the map you draw from your flashlight sightings. The map can be amazingly accurate and still not be the cave itself.
\end{ChildBox}

\subsection*{2.5 The integration operator $\Integr$ (recursive self-modeling)}
\begin{FormalBox}{Closure and self-updating}
$\Integr$ denotes recursive self-modeling / integration: memory-like feedback that stabilizes (or destabilizes) patterns across updates.
\[
\Integr:\ (\text{past projected structures}) \longrightarrow (\text{constraints on future structure})
\]
\end{FormalBox}

\begin{MeaningBox}{Why this matters}
A universe that can \emph{remember} constraints becomes a universe that can build persistent structure: laws, constants, quasi-particles, even living systems. $\Integr$ is the abstract handle for “the model updates itself using its own outputs.”
\end{MeaningBox}

\begin{ChildBox}{Learning}
If reality can “learn” what works (stable patterns), it can keep those patterns around longer.
\end{ChildBox}

% =====================================================================================
\section*{3. Time-Crystalline Recurrence (Floquet-style structure)}
% =====================================================================================

\subsection*{3.1 Stroboscopic update}
\begin{FormalBox}{Floquet update (symbolic core)}
Assume a fundamental recurrence with period $T$ at the global level, captured by a stroboscopic update operator
\[
\Floquet \;\equiv\; U(T).
\]
Time-crystalline behavior corresponds to persistent periodic structure under repeated application:
\[
\text{Recurrence:}\qquad \Floquet^{n}\ket{\Psi_0}\ \text{retains structured invariants for large }n.
\]
This is intentionally schematic: the claim is about \emph{stability under repeated drive}, not about committing prematurely to one microscopic Hamiltonian form.
\end{FormalBox}

\begin{MeaningBox}{The “cosmic heartbeat” metaphor (with teeth)}
The recurrence is not a poetic add-on: it is a mechanism for why “laws” can look time-invariant even if the substrate is dynamic. Periodic self-consistency is a factory for robust invariants.
\end{MeaningBox}

\begin{ChildBox}{A spinning top}
A top wobbles fast, but the wobble repeats. If the repetition is stable, you can predict it and build clocks from it.
\end{ChildBox}

\subsection*{3.2 Projection + recurrence = classical persistence}
\begin{FormalBox}{Sector stability}
A projected sector is stable when repeated global updates remain compatible with the projection:
\[
\Proj\,\Floquet \approx \Floquet_{\mathrm{loc}}\,\Proj
\]
for an effective local update $\Floquet_{\mathrm{loc}}$ on $\Substrate_{\mathrm{loc}}$. This is a commutation-style compatibility condition indicating that the sector “closes” under recurrence.
\end{FormalBox}

\begin{MeaningBox}{Why commutation shows up everywhere}
When two operations commute, order doesn’t matter; that is the algebraic skeleton of “lawfulness.” Here, lawfulness means: update the world then look, or look then update—either way the story matches inside the slice.
\end{MeaningBox}

\begin{ChildBox}{Two ways to do the same thing}
Whether you “zoom in first then press play” or “press play then zoom in,” you see the same kind of movie.
\end{ChildBox}

% =====================================================================================
\section*{4. Emergent Geometry from Coherence}
% =====================================================================================

\subsection*{4.1 Relational distance as coherence-contrast}
\begin{FormalBox}{Distance from distinguishability}
Define a relational distance between two accessible states using an information-geometric measure of distinguishability (symbolic placeholder):
\[
d(\rho_1,\rho_2)\ \sim\ \text{Distinguishability}(\rho_1,\rho_2).
\]
In this framework, “far apart” means “hard to transform into each other within the stable sector,” often correlating with low coherence overlap under $\Proj$.
\end{FormalBox}

\begin{MeaningBox}{Space as a grammar of transformations}
Space is not a container; it is a \emph{pattern of which transformations are easy vs hard}. Geometry is the language that summarizes the transformation grammar inside stable sectors.
\end{MeaningBox}

\begin{ChildBox}{Neighborhoods}
Two houses are “close” if you can walk between them easily. In physics, two states are “close” if the system can move between them easily.
\end{ChildBox}

\subsection*{4.2 Why gravity might emerge}
\begin{FormalBox}{Heuristic mechanism}
If coherence acts like a resource that must remain consistent under recurrence, then gradients in coherence-structure can induce effective attractive dynamics—an emergent “pull” toward configurations that keep the sector stable.
\[
\text{(Sketch)}\qquad \text{effective attraction} \;\propto\; \nabla(\text{coherence-structure}).
\]
We keep this as a programmatic statement until a concrete functional is fixed.
\end{FormalBox}

\begin{MeaningBox}{Gravity as bookkeeping for stability pressure}
Gravity becomes a macroscopic encoding of “stability pressure”: large-scale structure forms where the sector can keep renewing itself with minimal contradiction.
\end{MeaningBox}

\begin{ChildBox}{Why things clump}
If certain arrangements keep working and repeating, the universe keeps them—so matter clumps into stars instead of dissolving into noise.
\end{ChildBox}

% =====================================================================================
\subsection*{4.3 Holographic entanglement as the organizing principle}
% =====================================================================================

\begin{FormalBox}{Holographic entanglement (minimal, structural statement)}
In holographic frameworks, the central lesson is that \emph{entanglement structure organizes geometry}.
A canonical expression of this idea is that the entanglement entropy of a region $A$ is captured by an
extremal (often minimal) surface $\gamma_A$ associated to $A$:
\[
S(A)\ \propto\ \frac{\mathrm{Area}(\gamma_A)}{4G\hbar},
\]
where the proportionality constants depend on the specific holographic realization.
In the present framework, we do \emph{not} assume a pre-existing spacetime in which such a surface lives.
Instead, we treat this equation as a \emph{design constraint}:

\medskip
\noindent
\textbf{Organizing principle.}
The effective geometry encoded by $\Emerg$ is the one for which ``cut-like'' measures of correlation
between accessible subregions behave as extremal functionals (min-cut / max-flow structure),
and those extremal values set the effective relational distances and areas.

\medskip
Operationally, let $A$ label an accessible subregion/sector after projection.
We require that the emergent description admits an extremization principle of the schematic form
\[
\mathcal{S}(A)\ =\ \underset{\text{admissible cuts}}{\mathrm{extremum}}\;\;\mathcal{K}(\text{cut};A),
\]
where $\mathcal{S}(A)$ is a correlation/entanglement measure on the accessible description and
$\mathcal{K}$ is a ``cut cost'' induced by coherence constraints and the dynamics.
This is the substrate-native analogue of holographic entanglement: geometry is an encoding of how
correlations can be separated at minimal cost.
\end{FormalBox}

\begin{MeaningBox}{Why this is the right organizing principle}
Entanglement is not ``a thing sitting in space''—it is the pattern of inseparability that makes
the very notion of \emph{inside/outside} meaningful. When you promote that inseparability pattern
to the primary organizer, geometry becomes a \emph{compression scheme} for correlations:

\begin{itemize}[leftmargin=*,itemsep=2pt]
  \item \textbf{Surfaces are not fundamental objects;} they are bookkeeping devices that summarize where correlations
  ``bottleneck'' between regions.
  \item \textbf{Distance is not a primitive;} it is a measure of how distinguishable two sectors are given the allowed
  transformations and constraints.
  \item \textbf{Locality is earned, not assumed;} it is what appears when the correlation network has stable sparse cuts
  (easy separations) that persist under recurrence.
\end{itemize}

In other words: the emergent world is the most coherent narrative of \emph{who is entangled with whom},
as stabilized by repeated global updates.
\end{MeaningBox}

\begin{ChildBox}{Threads and scissors}
Imagine everything is connected by invisible rubber bands. Some parts have many strong bands between them,
so they behave like a single object. If you try to separate them, it costs a lot.
A ``surface'' is like the best place to cut the bands to split one region from another with the smallest effort.
When the same best cuts keep showing up again and again, you get a stable idea of shape and distance:
that is how space can emerge from connections.
\end{ChildBox}

\begin{TestBox}{What would count as evidence?}
This principle becomes scientific when the framework specifies:
\begin{itemize}[leftmargin=*,itemsep=2pt]
  \item a concrete correlation/entanglement functional on the projected description (what exactly is $\mathcal{S}$?),
  \item a concrete admissibility rule for cuts/flows (what exactly is an ``allowed'' separation in this substrate?),
  \item a derived emergent metric/geometry from those extremal values (how do distances/areas compute?),
  \item stability under the stroboscopic update (do the extremal structures persist across recurrence?).
\end{itemize}
Then the framework predicts quantitative relationships between correlation structure and effective geometry,
and those can be compared to known holographic behaviors (and to controlled quantum systems).
\end{TestBox}

\begin{WebsiteBox}{Module — “Holographic entanglement”}
The guiding idea is simple: geometry is what correlations look like when they become stable.
Instead of assuming spacetime, we derive an effective geometry as the structure that best summarizes
how the accessible world can be separated into parts with minimal loss of correlation.
When the same “minimal cut” pattern persists under repeated global recurrence, locality and distance become real.
\end{WebsiteBox}

\begin{WebsiteBox}{Module — “Why this matters”}
This turns holography into a constructive tool: it tells us what spacetime \emph{is made of}.
Not matter-in-space, but correlation structure stabilized by coherence constraints.
\end{WebsiteBox}

% =====================================================================================
\section*{5. Minimal Mathematical Architecture (the clean skeleton)}
% =====================================================================================

\begin{FormalBox}{The closure loop (core chain)}
We use the self-engineering chain:
\[
\Htot \;\longrightarrow\; \Substrate \;\longrightarrow\; \Proj \;\longrightarrow\; \Emerg \;\longrightarrow\; \Integr \;\longrightarrow\; \Htot.
\]
Interpretation: global dynamics acts on the substrate; projection yields accessible sectors; emergent operator organizes effective physics; integration feeds constraints back into the generator, closing the loop.
\end{FormalBox}

\begin{MeaningBox}{A universe that bootstraps itself}
This is the philosophical heart: “laws” are not externally imposed; they are internal fixed points of a self-consistency loop. Reality is an \emph{iterated coherence game} that selects what can keep existing.
\end{MeaningBox}

\begin{ChildBox}{A self-correcting story}
The universe tells a story, checks it for contradictions, and adjusts so the story can continue.
\end{ChildBox}

% =====================================================================================
\section*{6. Predictions, Constraints, and Referee-Proofing}
% =====================================================================================

\subsection*{6.1 What this framework must reproduce}
\begin{FormalBox}{Non-negotiables}
Any viable instantiation must reproduce:
\begin{itemize}[leftmargin=*,itemsep=2pt]
  \item standard quantum behavior in appropriate regimes,
  \item relativistic causal structure as an emergent effective symmetry,
  \item thermodynamic irreversibility as an emergent arrow (not assumed),
  \item effective field dynamics and perturbations around stable sectors,
  \item robustness under noise (otherwise no stable “classical world”).
\end{itemize}
\end{FormalBox}

\subsection*{6.2 What is genuinely new here (and testable)}
\begin{TestBox}{Test hooks (programmatic)}
Potentially testable signatures depend on the chosen concrete coherence-functional, but the \emph{structural} predictions are:
\begin{itemize}[leftmargin=*,itemsep=2pt]
  \item \textbf{Recurrence signatures:} subtle periodic or quasi-periodic invariants at deep levels of description (where the effective theory still remembers the stroboscopic update).
  \item \textbf{Sector stability thresholds:} sharp transitions when $\Proj$-compatibility fails (``classicality breaks'' in controlled systems).
  \item \textbf{Geometry-from-information fingerprints:} distances and effective curvature correlating with distinguishability metrics rather than primitive coordinates.
\end{itemize}
\end{TestBox}

\begin{MeaningBox}{How to avoid metaphysics drift}
The framework becomes science when a concrete functional is fixed and used to compute: recurrence rates, stability boundaries, emergent metric structure, and effective field equations. Until then, the skeleton is a \emph{research program}, not a completed theory.
\end{MeaningBox}

% =====================================================================================
\section*{7. Website Page Modules (copy-paste blocks)}
% =====================================================================================

\begin{WebsiteBox}{Module A — “What is COSMOS?”}
COSMOS is the physics-facing branch of Science of Coherence. It models the universe as a single underlying substrate governed by a globally coherent generator. Local experience arises from projection—partial accessibility—while spacetime and fields emerge as stable organizational patterns inside projected sectors. A recurring (time-crystalline) update provides a mechanism for robust invariants that can look like “laws of nature.”
\end{WebsiteBox}

\begin{WebsiteBox}{Module B — “Why time-crystalline recurrence?”}
A repeating global update is a natural factory for stable structure. If the substrate renews its coherence in cycles, patterns that remain compatible across cycles persist, while incompatible patterns decay. This offers a route to effective classical stability without assuming classicality as fundamental.
\end{WebsiteBox}

\begin{WebsiteBox}{Module C — “Why holographic?”}
Holographic here means: the effective world is an encoding of deeper coherent dynamics. Spacetime is treated as a derived layer that summarizes correlations within stable sectors, not as the primitive stage on which reality happens.
\end{WebsiteBox}

\begin{WebsiteBox}{Module D — “The five symbols”}
\begin{itemize}[leftmargin=*,itemsep=2pt]
  \item $\Substrate$: the only arena of configurations.
  \item $\Htot$: global coherent generator on the substrate.
  \item $\Proj$: projection/coherence operator producing accessible sectors.
  \item $\Emerg$: effective operator encoding emergent spacetime/fields/histories.
  \item $\Integr$: recursive integration (memory-like feedback) closing the loop.
\end{itemize}
\end{WebsiteBox}

% =====================================================================================
\section*{8. Appendix: Placeholders for the next technical steps}
% =====================================================================================

\subsection*{8.1 Pick one coherence functional (do not overcomplicate)}
\begin{FormalBox}{To be defined}
Choose a single coherence-structure functional $\mathcal{C}[\Psi]$ or $\mathcal{C}[\rho]$ (notation choice later), and one informational free-energy style functional $\mathcal{F}$ that selects stable sectors. Keep it minimal.
\[
\mathcal{F} \;=\; \mathcal{E} \;-\; \kappa\,\mathcal{C}
\qquad\text{(placeholder; finalize meanings/units once $\mathcal{C}$ is fixed).}
\]
\end{FormalBox}

\subsection*{8.2 Derive the compatibility condition cleanly}
\begin{FormalBox}{To be derived}
Turn the schematic
\[
\Proj\,\Floquet \approx \Floquet_{\mathrm{loc}}\,\Proj
\]
into a precise condition (norm bound, channel distance, commutator constraint, etc.) that yields a sharp stability criterion.
\end{FormalBox}

\subsection*{8.3 Extract emergent geometry}
\begin{FormalBox}{To be derived}
Specify the distinguishability metric (e.g., Fisher-like or Fubini--Study-like in an appropriate representation) and map it to an emergent relational geometry inside stable sectors.
\end{FormalBox}

\vspace{3mm}
\begin{center}
{\small \textit{End of COSMOS base file. This is meant to be iterated section-by-section.}}
\end{center}

\end{document}

% Magic, Science, and the Time-Crystalline Quantum Substrate (TCQS)
\documentclass[12pt]{article}

% Page & typography
\usepackage[letterpaper,margin=1in]{geometry}
\usepackage{microtype}
\usepackage{amsmath,amssymb,mathtools}
\usepackage{graphicx}
\usepackage[hidelinks]{hyperref}

% Abstract style
\usepackage{abstract}
\renewcommand{\abstractnamefont}{\normalfont\bfseries}
\renewcommand{\abstracttextfont}{\normalfont}
\setlength{\absleftindent}{0pt}
\setlength{\absrightindent}{0pt}
\renewcommand{\abslabeldelim}{.}

% Title
\title{\LARGE\bfseries Magic, Science, and the\\[0.25em]
Time–Crystalline Quantum Substrate}
\date{}
\author{}

\begin{document}
\maketitle

\begin{abstract}
In the Time–Crystalline Quantum Substrate (TCQS) framework, phenomena traditionally labeled
as “magic” are reinterpreted as ordered manifestations of informational coherence.
TCQS models the persistence, restoration, and circulation of information across discrete
temporal cycles, providing a unifying description of physical, biological, and cognitive
organization. This paper outlines the substrate’s minimal invariants and explores how
informational vortices can function as coherent memory structures within the universal field.
\end{abstract}

\section*{Coherence and Temporal Order}
In TCQS, coherence denotes the persistence of information through periodic temporal
structure. A compact measure is the \emph{coherence density}
\begin{equation}
  \rho_{C} \;=\; \frac{\mathcal{I}_{\mathrm{kept}}}{T},
\end{equation}
where $\mathcal{I}_{\mathrm{kept}}$ is the information preserved per temporal period $T$.
Time–crystal behavior (periodic order in time) corresponds to elevated $\rho_C$, indicating
self-consistent dynamics and reduced effective entropy production across cycles.

\section*{Informational Dynamics}
Coherence evolves under an informational free-energy functional $F$, with equilibrium
characterized by a vanishing informational gradient:
\begin{equation}
  \nabla_{\!\mathrm{info}} F \to 0 \quad \Rightarrow \quad \text{coherence equilibrium}.
\end{equation}
Discrete re-coherence is represented by a Floquet-type update rule:
\begin{equation}
  U(t{+}T)=e^{-iHT/\hbar}\,U(t),
\end{equation}
which encodes periodic stabilization of the substrate’s informational state.

\section*{Biological and Cognitive Extension}
Biological systems exhibit coherence stabilization through feedback: coupled oscillators
(neural, molecular, metabolic) maintain structure by synchronizing to internal and
environmental periodicities. Within TCQS, consciousness is modeled as an adaptive
feedback process optimizing coherence in the organismic substrate, linking subjective
report to objective temporal order.

\section*{Informational Residues and Retrieval Dynamics}
Each coherent configuration leaves an \emph{informational residue}—a structured region
of high coherence density within the TCQS phase space. These regions form topological
vortices where information from previous coherent states persists as stable attractors.
When a system re-enters resonance with such a vortex, the stored information can be
partially reconstructed through phase re-alignment. This mechanism allows, in principle,
the retrieval of historical informational patterns or memories encoded within the substrate,
analogous to recovering data from a standing wave in an informational field.

\section*{Why It Matters}
When coherence is measurable, formerly “magical” regularities become reproducible.
TCQS supplies minimal invariants—$\rho_C$, $\nabla_{\!\mathrm{info}}F$, and the
Floquet update—that operationalize such regularities across levels of description,
bridging empirical science and subjective evidence within a single formal grammar.

\end{document}

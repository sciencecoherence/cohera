% Magic, Science, and the Time-Crystalline Quantum Substrate (TCQS)
\documentclass[12pt]{article}

% Page & typography
\usepackage[letterpaper,margin=1in]{geometry}
\usepackage{microtype}
\usepackage{amsmath,amssymb,mathtools}
\usepackage{graphicx}
\usepackage[hidelinks]{hyperref}

% Abstract formatting: left-aligned "Abstract." with a period (like your sample)
\usepackage{abstract}
\renewcommand{\abstractnamefont}{\normalfont\bfseries}
\renewcommand{\abstracttextfont}{\normalfont}
\setlength{\absleftindent}{0pt}
\setlength{\absrightindent}{0pt}
\renewcommand{\abslabeldelim}{.}
\begin{document}
% Title
\title{\textbf{The singularity:}\\[0.25em]
When science appear as magic}
\date{}      % no date
\author{}    % no author (impersonal paper)

\maketitle
\emph{“Any sufficiently advanced technology is indistinguishable from magic.”} \hfill—Arthur C. Clarke
\begin{abstract}
In the Time–Crystalline Quantum Substrate (TCQS) framework, phenomena often labeled
“magic” are recast as regularities emerging from quantifiable coherence. TCQS models the
persistence and restoration of information across discrete temporal cycles, yielding a unified
description that spans physics, mathematics, biology, and cognition. This note states the
minimal invariants and their implications for measurability and reproducibility.
\end{abstract}

\section*{Coherence and Temporal Order}
In TCQS, coherence denotes the persistence of information through periodic temporal
structure. A compact measure is the \emph{coherence density}
\begin{equation}
  \rho_{C} \;=\; \frac{\mathcal{I}_{\mathrm{kept}}}{T},
\end{equation}
where $\mathcal{I}_{\mathrm{kept}}$ is the information preserved per temporal period $T$.
Time-crystal behavior (periodic order in time) corresponds to elevated $\rho_C$, indicating
self-consistent dynamics and reduced effective entropy production across cycles.

\section*{Informational Dynamics}
Coherence evolves under an informational free-energy functional $F$, with equilibrium
characterized by a vanishing informational gradient:
\begin{equation}
  \nabla_{\!\mathrm{info}} F \to 0 \quad \Rightarrow \quad \text{coherence equilibrium}.
\end{equation}
Discrete re-coherence is represented by a Floquet-type update rule
\begin{equation}
  U(t{+}T)\;=\;e^{-iHT/\hbar}\,U(t),
\end{equation}
which encodes periodic stabilization of the substrate’s informational state.

\section*{Biological and Cognitive Extension}
Biological systems exhibit coherence stabilization via feedback: coupled oscillators
(neural, molecular, metabolic) maintain structure by synchronizing to internal and
environmental periodicities. Within TCQS, consciousness is modeled as an adaptive
feedback process optimizing coherence in the organismic substrate, linking subjective
report to objective temporal order.

\section*{Why It Matters}
When coherence is measurable, formerly “magical” regularities become reproducible.
TCQS supplies minimal invariants—$\rho_C$, $\nabla_{\!\mathrm{info}}F$, and the
Floquet update—that operationalize such regularities across levels of description,
bridging empirical science and subjective evidence within a single formal grammar.

\end{document}

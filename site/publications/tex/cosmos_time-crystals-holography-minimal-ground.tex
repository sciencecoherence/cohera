\section*{Time Crystals, Holography, and Quantum Information: Minimal Common Ground}
\subsection*{Abstract}
This publication provides a structured synthesis for Time Crystals, Holography, and Quantum Information: Minimal Common Ground, with claim-to-evidence framing and a validation path for downstream readers.

\subsection*{Keywords}
cosmos, research, publication

\subsection*{Main Content}
% Auto-generated publication draft
\section*{Time Crystals, Holography, and Quantum Information: Minimal Common Ground}
\textbf{Thread:} cosmos\\
\textbf{Digest slug:} time-crystals-holography-minimal-ground\\
\textbf{Source:} \texttt{site/cosmos/digests/time-crystals-holography-minimal-ground.html}\\
\textbf{Generated:} 2026-02-23 14:59 UTC\\

\subsection*{Structured draft body}
Time Crystals, Holography, and Quantum Information · Cohera Lab Cohera Lab Home Research Cosmos Regenesis Ethos Publications Contact About Time Crystals, Holography, and Quantum Information: Minimal Common Ground Date: 2026-02-21 · Tags: cosmos, floquet, holography, QI · Confidence: medium Summary This digest defines the smallest overlap between three domains that matter for Cosmos formalization: driven periodic dynamics (Floquet/time-crystal literature), holographic information bounds, and quantum-information structure. The immediate goal is not a grand unification claim, but a constrained operator-level language that can be tested against known limits. Key findings (with confidence) (High) Discrete time crystals are non-equilibrium phenomena tied to periodically driven systems and symmetry breaking in time translation, with strict caveats in equilibrium settings. [Citation] (High) Floquet systems are naturally analyzed through one-period evolution operators, making them a useful template for discrete update rules in model-building. [Citation] (High) Holographic entropy ideas connect geometry and information constraints, setting a precedent for “effective world” descriptions emerging from lower-level informational structure. [Citation] (Medium) Entanglement entropy/geometry correspondences suggest a practical bridge between state structure and emergent geometric interpretation. [Citation] (High) Quantum error-correction language is useful for talking about robust emergent descriptions under partial access/noise. [Citation] Hypotheses (explicitly labeled) Hypothesis A (low): A substrate-level periodic or quasi-periodic update map may provide a compact formal handle for emergent temporality. Hypothesis B (low): A coherence-selection operator can be defined as a constrained coarse-graining that preserves stability-relevant invariants. Hypothesis C (low): Finite coherence constraints may regularize singular effective descriptions in specific regimes. Operational next queries Formalize the minimal state space and operator domain (Hilbert vs algebraic vs measure-theoretic). Define explicit update map U and candidate coherence functional C[Ψ]. Test whether known weak-field and thermodynamic limits can be approximated without overfitting language. References Watanabe \& Oshikawa (2015), Absence of Quantum Time Crystals — https://journals.aps.org/prl/abstract/10.1103/PhysRevLett.114.251603 Else, Bauer, Nayak (2016), Floquet Time Crystals — https://journals.aps.org/prl/abstract/10.1103/PhysRevLett.117.090402 Eckardt (2016), Colloquium: Atomic quantum gases in periodically driven optical lattices — https://arxiv.org/abs/1606.08041 't Hooft (1993), Dimensional reduction in quantum gravity — https://arxiv.org/abs/gr-qc/9310026 Ryu \& Takayanagi (2006), Holographic derivation of entanglement entropy — https://arxiv.org/abs/hep-th/0603001 Almheiri, Dong, Harlow (2014), Bulk Locality and Quantum Error Correction in AdS/CFT — https://arxiv.org/abs/1411.7041 Cosmos digest v1 · refined against source corpus over time.

\subsection*{Validation checklist}
\begin{itemize}
  \item Verify all nontrivial claims against the original source.
  \item Add explicit citations/DOIs where available.
  \item Mark confidence for each key claim (low/medium/high).
\end{itemize}

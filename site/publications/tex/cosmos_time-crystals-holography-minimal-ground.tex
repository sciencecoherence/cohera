\section*{Time Crystals, Holography, and Quantum Information: Minimal Common Ground}
\subsection*{Abstract}
This publication provides a structured synthesis for Time Crystals, Holography, and Quantum Information: Minimal Common Ground, with claim-to-evidence framing and a validation path for downstream readers.

\subsection*{Keywords}
cosmos, research, publication

\subsection*{Main Content}
% Publication polished style (journal-like)
\title{Time Crystals, Holography, and Quantum Information:\\A Minimal Coherence Framework}
\author{Cohera Lab}
\date{February 2026}
\maketitle

\begin{center}
\rule{0.92\linewidth}{0.6pt}\\[0.5em]
\small Research Thread: Cosmos \quad|\quad Manuscript Type: Conceptual/Technical Synthesis
\\[0.3em]\rule{0.92\linewidth}{0.6pt}
\end{center}

\begin{abstract}
We present a compact synthesis connecting Floquet time-crystal dynamics, holographic coarse-graining, and operator-level coherence constraints. The goal is not to claim a final theory but to formalize a minimal testable scaffold that separates established results from speculative bridges. We introduce an effective coherence functional and derive practical falsification hooks for numerical and experimental workflows.
\end{abstract}

\paragraph{Keywords} time crystals; Floquet dynamics; holography; quantum information; coherence operators

\section{Introduction}
Time-crystal discourse often mixes distinct regimes: rigorous no-go statements in equilibrium, non-equilibrium Floquet order, and information-theoretic interpretations. This manuscript imposes a minimal common language so each claim can be attached to assumptions, observables, and failure modes.

\section{Minimal Formal Layer}
Let \(\rho_t\) denote a reduced state on a finite operator algebra. We define a periodic coherence witness over one drive period \(T\):
\begin{equation}
\mathcal{C}_T = \frac{1}{T}\int_0^T \mathrm{Tr}\!\left[\rho_t\,\mathcal{O}_c\right]dt,
\end{equation}
where \(\mathcal{O}_c\) is a bounded coherence-sensitive observable.

For a stroboscopic map \(\Phi_T\), persistence under perturbation family \(\{\epsilon\}\) is tracked by
\begin{equation}
\Delta_{\epsilon} = \left|\mathcal{C}_T(\epsilon)-\mathcal{C}_T(0)\right|.
\end{equation}
A practical robustness criterion is
\begin{equation}
\sup_{\epsilon\in[0,\epsilon_0]} \Delta_{\epsilon} < \delta_*.
\end{equation}

\section{Holographic Interpretation Layer}
We model coarse-grained reconstruction by an effective channel \(\mathcal{R}\) acting on low-energy observables. A consistency requirement between boundary witness and reconstructed bulk proxy is
\begin{equation}
\left|\mathcal{C}^{\partial}_T - \mathcal{C}^{\mathrm{bulk}}_T\right| \leq \eta,
\end{equation}
with \(\eta\) explicitly reported as model error, not absorbed into interpretation.

\section{Validation and Falsification Hooks}
\begin{itemize}
\item \textbf{Numerical hook:} sweep drive-noise amplitude and test violation of the robustness bound.
\item \textbf{Model hook:} replace \(\mathcal{O}_c\) by orthogonal witness families; check whether conclusions are basis-dependent.
\item \textbf{Interpretation hook:} force \(\eta\to 0\) in controlled toy models; reject holographic claim if mismatch persists.
\end{itemize}

\section{Conclusion}
This paper provides a cleaner publication-ready baseline: assumptions are explicit, equations are readable, and each interpretive step is paired with a concrete failure condition.

\section*{References (working set)}
\begin{enumerate}
\item F. Wilczek, ``Quantum Time Crystals,'' \emph{Phys. Rev. Lett.} (2012).
\item H. Watanabe, M. Oshikawa, ``Absence of Quantum Time Crystals,'' \emph{Phys. Rev. Lett.} 114 (2015).
\item D. V. Else, B. Bauer, C. Nayak, ``Floquet Time Crystals,'' \emph{Phys. Rev. Lett.} 117 (2016).
\end{enumerate}

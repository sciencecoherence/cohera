\section*{Regenesis Sleep Regeneration Baseline Publication V1}
\subsection*{Abstract}
This publication provides a structured synthesis for Regenesis Sleep Regeneration Baseline Publication V1, with claim-to-evidence framing and a validation path for downstream readers.

\subsection*{Keywords}
regenesis, research, publication

\subsection*{Main Content}
% Auto-promoted publication draft
\title{Sleep and Regeneration: Baseline Mechanisms and Practical Protocol}
\author{Cohera Lab}
\date{2026-02-23}
\maketitle

\begin{center}\small Thread: Regenesis\end{center}

\begin{abstract}
This manuscript promotes an in-progress regenesis research stream into a publication-ready synthesis. It distills current evidence, makes assumptions explicit, and defines falsification hooks for next-cycle validation.
\end{abstract}

\paragraph{Keywords} regenesis, synthesis, publication

\section{Introduction}
Sleep and Regeneration Baseline · Cohera Lab Cohera Lab Home Research Cosmos Regenesis Ethos Publications Contact About Sleep and Regeneration: Baseline Mechanisms and Practical Protocol Date: 2026-02-21 · Tags: regenesis, sleep, circadian, recovery · Confidence: medium-high Summary Sleep quality is one of the highest-leverage recovery inputs. Evidence consistently links sleep duration/regularity with metabolic, cognitive, immune, and recovery outcomes. This digest focuses on robust baseline actions with low downside and clear monitoring signals. Key findings (with confidence) (High) Chronic sleep restriction impairs glucose regulation, mood, cognition, and performance. [Citation] (High) Circadian misalignment (timing mismatch) disrupts hormonal and metabolic signaling. [Citation] (High) Sleep supports immune function and inflammatory regulation. [Citation] (Medium) Sleep is associated with glymphatic clearance and brain waste transport dynamics (mainly animal-supported with growing human relevance). [Citation] (High) Consistent wake time is a practical anchor for circadian stabilization. [Citation] Practical baseline protocol (educational) Set a stable wake-time window (±30 min). Morning light exposure within 60 minutes after waking. Reduce bright/blue-heavy light 2–3 hours pre-

\section{Evidence and Claims}
sleep. Keep sleep environment dark, quiet, and cooler. Avoid heavy meals/alcohol close to bedtime. Risks / safety notes This is educational, not medical diagnosis/treatment. Persistent insomnia, breathing pauses, severe daytime sleepiness, or mood crisis require clinical evaluation. Hypotheses (labeled) Hypothesis (low): Personalized timing of macronutrients may improve next-day sleep architecture for some individuals. Hypothesis (low): Individual thermal routines (hot-cold contrast) may improve sleep onset in specific chronotypes. Next queries Quantify effect sizes for light timing interventions in shift-like schedules. Review evidence quality for popular sleep supplements (safety-first grading). Regenesis digest v1 · iterate with measured outcomes.

\section{Discussion}
Discussion to be expanded in subsequent research cycles.

\section{Validation Hooks}
\begin{itemize}
\item Verify central claims against primary paper and DOI source.
\item Separate measured evidence from interpretation in final revision.
\item Keep confidence conservative until corroboration is explicit.
\end{itemize}

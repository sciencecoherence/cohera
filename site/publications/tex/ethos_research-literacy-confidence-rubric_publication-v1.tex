\section*{Ethos Research Literacy Confidence Rubric Publication V1}
\subsection*{Abstract}
This publication provides a structured synthesis for Ethos Research Literacy Confidence Rubric Publication V1, with claim-to-evidence framing and a validation path for downstream readers.

\subsection*{Keywords}
ethos, research, publication

\subsection*{Main Content}
% Auto-promoted publication draft
\title{Research Literacy Quickstart + Confidence Rubric}
\author{Cohera Lab}
\date{2026-02-23}
\maketitle

\begin{center}\small Thread: Ethos\end{center}

\begin{abstract}
This manuscript promotes an in-progress ethos research stream into a publication-ready synthesis. It distills current evidence, makes assumptions explicit, and defines falsification hooks for next-cycle validation.
\end{abstract}

\paragraph{Keywords} ethos, synthesis, publication

\section{Introduction}
Research Literacy and Confidence Rubric · Cohera Lab Cohera Lab Home Research Cosmos Regenesis Ethos Publications Contact About Research Literacy Quickstart + Confidence Rubric Date: 2026-02-21 · Tags: ethos, epistemology, education, confidence · Confidence: high Summary This module defines a simple quality filter for all Cohera threads. It separates claims by evidence strength, marks uncertainty explicitly, and provides a repeatable paper-reading workflow. Quick paper-reading workflow Read abstract + conclusion first: what was actually tested? Check methods and sample constraints. Distinguish correlation vs intervention. Check if result has replication/systematic review support. Write one-sentence claim with confidence label. Confidence rubric (standard) High: converging evidence from reviews/meta-analyses or strong replicated primary data. Medium: plausible and supported by limited or mixed evidence. Low: early, speculative, or mostly conceptual; publish only as hypothesis. Common traps Single-paper overreach Mechanism inflation beyond data Confusing surrogate markers with outcomes Ignoring population/context limits Citations Guyatt et al. (GRADE framework) — https://www.bmj.com/content/336/7650/924 Ioannidis (Why Most Published Research Findings Are False) — https://journals.p

\section{Evidence and Claims}
los.org/plosmedicine/article?id=10.1371/journal.pmed.0020124 Cochrane Handbook (evidence synthesis standards) — https://training.cochrane.org/handbook Next iteration Add a per-thread checklist card so every digest visibly includes: confidence, citations, uncertainty, and known failure modes. Ethos module v1 · show your work, mark uncertainty.

\section{Discussion}
Discussion to be expanded in subsequent research cycles.

\section{Validation Hooks}
\begin{itemize}
\item Verify central claims against primary paper and DOI source.
\item Separate measured evidence from interpretation in final revision.
\item Keep confidence conservative until corroboration is explicit.
\end{itemize}

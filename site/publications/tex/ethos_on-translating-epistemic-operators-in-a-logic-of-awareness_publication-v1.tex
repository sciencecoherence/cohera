\section*{Ethos On Translating Epistemic Operators In A Logic Of Awareness Publication V1}
\subsection*{Abstract}
This publication provides a structured synthesis for Ethos On Translating Epistemic Operators In A Logic Of Awareness Publication V1, with claim-to-evidence framing and a validation path for downstream readers.

\subsection*{Keywords}
ethos, research, publication

\subsection*{Main Content}
% Auto-promoted publication draft
\title{On Translating Epistemic Operators in a Logic of Awareness}
\author{Cohera Lab}
\date{2026-02-23}
\maketitle

\begin{center}\small Thread: Ethos\end{center}

\begin{abstract}
This manuscript promotes an in-progress ethos research stream into a publication-ready synthesis. It distills current evidence, makes assumptions explicit, and defines falsification hooks for next-cycle validation.
\end{abstract}

\paragraph{Keywords} ethos, synthesis, publication

\section{Introduction}
Autodraft · On Translating Epistemic Operators in a Logic of Awareness · Cohera Lab Cohera Lab Home Research Cosmos Regenesis Ethos Publications About Autodraft: On Translating Epistemic Operators in a Logic of Awareness Date: 2026-02-23 · Thread: ethos · Status: extracted-draft · Confidence: low-medium Source chatgpt/pdf/On\_Translating\_Epistemic\_Operators\_in\_a\_Logic\_of\_Awareness.pdf DOI: not detected automatically. Auto summary (preview-based) . Awareness-Based Indistinguishability Logic (henceforth, AIL) is an extension of Epistemic Logic by introducing the notion of awareness, distinguishing explicit knowledge from implicit knowledge. In this framework, each of these notions is represented by a modal operator. On the other hand, HMS models, developed in the economics literature, also provide a formalization of those notions. Nevertheless, the behavior of the epistemic operators in AIL within HMS models has yet to be explored. In this paper, we define a transformation of an AIL model into an HMS model and then prove that a translation between the fragments of the language of AIL preserves truth under this transformation. As a result, we clarify the semantic role of an epistemic operator in AIL, which is induced by awareness and is essential to defining explicit knowledge, withi

\section{Evidence and Claims}
n HMS models. Furthermore, we demonstrate the differences in the implicit knowledge captured by AIL and HMS models. This work lays the groundwork for a comparative analysis between the model classes. Key findings (auto-extracted) Primary topic appears to center on: translating, epistemic, operators, logic. Source was auto-indexed and text-previewed for rapid triage. Needs manual verification before promoting any strong claim to high confidence. Evidence \& citations Source file: chatgpt/pdf/On\_Translating\_Epistemic\_Operators\_in\_a\_Logic\_of\_Awareness.pdf Extraction scope: 1-2 Abstract/preview extracted automatically. Claim → evidence mapping (auto) Claim: On Translating Epistemic Operators in a Logic of Awareness Yudai Kubono[0000−0003−2617−8870] arXiv:2602.18040v1 [cs.LO] 20 Feb 2026 Graduate School of Science and Technology, Shizu… Evidence quote: “On Translating Epistemic Operators in a Logic of Awareness Yudai Kubono[0000−0003−2617−8870] arXiv:2602.18040v1 [cs.LO] 20 Feb 2026 Graduate School of Science and Technology, Shizuoka University, Ohya, Shizuoka 422-8529, Japan yudai.kubono@gmail.com Abstract.” Page hint: 1-2 Claim: Awareness-Based Indistinguishability Logic (henceforth, AIL) is an extension of Epistemic Logic by introducing the notion of awareness, distinguishing explicit knowledge from impli… Evidence quote: “Awareness-Based Indistinguishability Logic (henceforth, AI

\section{Discussion}
L) is an extension of Epistemic Logic by introducing the notion of awareness, distinguishing explicit knowledge from implicit knowledge.” Page hint: 1-2 Claim: In this framework, each of these notions is represented by a modal operator. Evidence quote: “In this framework, each of these notions is represented by a modal operator.” Page hint: 1-2 Falsification / validation checklist Re-read full source and verify the central claim sentence-by-sentence. Cross-check against at least one independent source before promotion. Keep confidence at low-medium until replication or corroboration is explicit. Next queries What is the smallest testable claim from this source?

\section{Validation Hooks}
\begin{itemize}
\item Verify central claims against primary paper and DOI source.
\item Separate measured evidence from interpretation in final revision.
\item Keep confidence conservative until corroboration is explicit.
\end{itemize}

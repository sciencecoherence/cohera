\section*{Cosmos v5: Coherence Operators and Emergent Geometry}
\subsection*{Abstract}
This publication provides a structured synthesis for Cosmos v5: Coherence Operators and Emergent Geometry, with claim-to-evidence framing and a validation path for downstream readers.

\subsection*{Keywords}
cosmos, research, publication

\subsection*{Main Content}
% Journal-style manuscript
\title{Cosmos v5: Coherence Operators and Emergent Geometry}
\author{Cohera Lab}
\date{February 2026}
\maketitle

\begin{center}
\rule{0.92\linewidth}{0.6pt}\\[0.5em]
\small Research Thread: Cosmos \quad|\quad Manuscript Type: Formal Synthesis
\\[0.3em]\rule{0.92\linewidth}{0.6pt}
\end{center}

\begin{abstract}
This paper consolidates the current Cohera formulation of coherence operators and emergent geometry into a publishable baseline. It converts previously fragmented draft sections into a single formal narrative with explicit assumptions, computable definitions, and falsification hooks. The aim is methodological clarity: every equation should map to an observable claim and every interpretation should declare its uncertainty boundary.
\end{abstract}

\paragraph{Keywords} coherence operators; emergent geometry; projection stability; holographic framework; model validation

\section{Introduction}
The Cosmos v5 line of work attempts to connect operator-level coherence metrics with geometry-like structures that emerge under constrained projection. Earlier drafts contained useful fragments but lacked unified notation and test discipline. This manuscript provides a cleaned synthesis suitable for critical reading and replication planning.

\section{Formal Core}
Let $\rho_{\mathrm{loc}}$ denote the localized state estimate and $\Pi$ the projection operator over the selected coherent sector.

We define the off-diagonal coherence witness as:
\begin{equation}
\mathcal{C}(\rho_{\mathrm{loc}})=\sum_{i\neq j}\left|\rho_{ij}\right|.
\end{equation}

Selection dynamics are controlled through an energy-coherence tradeoff:
\begin{equation}
\mathcal{F}_n = E_n - \kappa\,\mathcal{C}_n,
\end{equation}
with minimization over recurrence windows.

Compatibility between projected and local flows is monitored via:
\begin{equation}
\varepsilon = \left\|\Pi\mathcal{F}-\mathcal{F}_{\mathrm{loc}}\Pi\right\|_{\mathrm{op}}.
\end{equation}

A stable coherent regime is declared only if
\begin{equation}
\varepsilon < \delta_{\mathrm{coh}}.
\end{equation}

\section{From Coherence to Geometry}
Given pairwise distinguishability between projected states, a graph is constructed and embedded to estimate a local metric tensor and curvature profile. The practical interpretation is limited: geometry is treated as an emergent descriptor, not an ontological primitive.

\section{Validation and Falsification}
\begin{itemize}
\item Sweep $\kappa$ and evaluate stability boundaries for $\delta_{\mathrm{coh}}$.
\item Replace coherence witness families and test invariance of qualitative conclusions.
\item Report failure cases where geometric reconstruction is unstable under mild perturbation.
\end{itemize}

\section{Conclusion}
Cosmos v5 now has a cleaner formal spine and a replicable validation path. This does not finalize the theory, but it does make the current claim set auditable and technically discussable.

\section*{References (working set)}
\begin{enumerate}
\item Cohera internal draft set: Section 8.1–8.3 formal notes (2026).
\item Standard operator-norm and distinguishability references used in quantum information geometry.
\end{enumerate}

% Publication-ready draft generated by Cohera
\section*{Self-Organized Bioelectricity via Collective Pump Alignment}
\textbf{Thread:} Regenesis\\
\textbf{Source:} \texttt{chatgpt/pdf/Self-Organized\_Bioelectricity\_via\_Collective\_Pump\_Alignment\_Physical\_Origin\_of\_Chemiosmosis.pdf}\\
\textbf{Date:} 2026-02-23\\

\subsection*{Abstract-level synthesis}
This publication summarizes the research item on collective pump alignment as a candidate physical mechanism behind chemiosmosis ordering in nonequilibrium biological systems. The current evidence base is derived from automated extraction plus direct source reading checkpoints.

\subsection*{Core claims (working set)}
\begin{enumerate}
\item Fluctuation-driven dynamics can induce collective ion-pump alignment.
\item Collective alignment can stabilize membrane-scale electrochemical potential.
\item The model suggests a route from microscopic pump orientation to macroscopic bioelectric order.
\end{enumerate}

\subsection*{Evidence notes}
\begin{itemize}
\item Primary source text extracted from first pages (pdftotext).
\item Claim-to-evidence mapping maintained in the Regenesis digest page.
\item Confidence remains \textbf{low-medium} pending deeper validation and external corroboration.
\end{itemize}

\subsection*{Validation and falsification hooks}
\begin{itemize}
\item Reconstruct the minimal model assumptions and verify parameter sensitivity.
\item Check whether predicted membrane potential formation survives perturbation/noise sweeps.
\item Cross-check with independent literature on chemiosmosis and membrane bioelectricity.
\end{itemize}

\subsection*{Extended source preview (traceability)}
\small
\texttt{Self-Organized Bioelectricity via Collective Pump Alignment: Physical Origin of Chemiosmosis Ryosuke Nishide ∗ and Kunihiko Kaneko arXiv:2602.16171v1 [physics.bio-ph] 18 Feb 2026 Niels Bohr Institute, University of Copenhagen, Jagtvej 155 A, Copenhagen N, 2200, Denmark Chemiosmosis maintains life in nonequilibrium through ion transport across membranes, yet the origin of this order remains unclear. We develop a minimal model in which ion pump orientation and the intracellular electrochemical potential mutually reinforce each other. This model shows that fluctuations can induce collective pump alignment and the formation of a membrane potential. The alignment undergoes a phase transition from disordered to ordered, analogous to the Ising model. Our results provide a self-organizing mechanism for the emergence of bioelectricity, with implications for the origin of life. Fig.1 Coupling between directional ion transport across a membrane and the generation of an electrochemical potential difference is essential for living cells[1]. Mitchell’s chemiosmotic theory established that electric currents operate at the source of life: nonequilibrium ion currents across membranes drive cellular energy transduction, including ATP synthesis[2–4]. Through this coupling, living systems are maintained far from equilibrium by sustained directional ion fluxes across membranes. At the origin of life, the emergence of sustained nonequilibrium ion flows across membranes is thought to have been a critical step. Phylogenetic evidence indicates that early life already synthesized ATP through membrane-potential–driven processes[5, 6], and chemical and geological studies suggest that early cells may have emerged by exploiting natural proton gradients associated with alkaline hydrothermal vents[7, 8]. However, cells can autonomously generate membrane potential through ion transport, even in the absence of preexisting ion gradients, leaving the origin of such mechanisms unresolved. Therefore, to elucidate the origin of life or protocells, it is necessary to understand how ion pumps could collectively generate and sustain membrane potential. Although studies on the origin of life have examined the necessity of autocatalytic reaction sets[9–12], the generation of bio-information[13–18], the role of lipid membranes[19–21], and required nonequilibrium conditions[22, 23], how nonequilibrium flows and membrane potential mutually sustain each other remains an open physical problem. Here, we i...}


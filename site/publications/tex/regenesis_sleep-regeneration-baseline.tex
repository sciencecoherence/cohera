\section*{Sleep and Regeneration: Baseline Mechanisms and Practical Protocol}
\subsection*{Abstract}
This publication provides a structured synthesis for Sleep and Regeneration: Baseline Mechanisms and Practical Protocol, with claim-to-evidence framing and a validation path for downstream readers.

\subsection*{Keywords}
regenesis, research, publication

\subsection*{Main Content}
% Auto-generated publication draft
\section*{Sleep and Regeneration: Baseline Mechanisms and Practical Protocol}
\textbf{Thread:} regenesis\\
\textbf{Digest slug:} sleep-regeneration-baseline\\
\textbf{Source:} \texttt{site/regenesis/digests/sleep-regeneration-baseline.html}\\
\textbf{Generated:} 2026-02-23 14:59 UTC\\

\subsection*{Structured draft body}
Sleep and Regeneration Baseline · Cohera Lab Cohera Lab Home Research Cosmos Regenesis Ethos Publications Contact About Sleep and Regeneration: Baseline Mechanisms and Practical Protocol Date: 2026-02-21 · Tags: regenesis, sleep, circadian, recovery · Confidence: medium-high Summary Sleep quality is one of the highest-leverage recovery inputs. Evidence consistently links sleep duration/regularity with metabolic, cognitive, immune, and recovery outcomes. This digest focuses on robust baseline actions with low downside and clear monitoring signals. Key findings (with confidence) (High) Chronic sleep restriction impairs glucose regulation, mood, cognition, and performance. [Citation] (High) Circadian misalignment (timing mismatch) disrupts hormonal and metabolic signaling. [Citation] (High) Sleep supports immune function and inflammatory regulation. [Citation] (Medium) Sleep is associated with glymphatic clearance and brain waste transport dynamics (mainly animal-supported with growing human relevance). [Citation] (High) Consistent wake time is a practical anchor for circadian stabilization. [Citation] Practical baseline protocol (educational) Set a stable wake-time window (±30 min). Morning light exposure within 60 minutes after waking. Reduce bright/blue-heavy light 2–3 hours pre-sleep. Keep sleep environment dark, quiet, and cooler. Avoid heavy meals/alcohol close to bedtime. Risks / safety notes This is educational, not medical diagnosis/treatment. Persistent insomnia, breathing pauses, severe daytime sleepiness, or mood crisis require clinical evaluation. Hypotheses (labeled) Hypothesis (low): Personalized timing of macronutrients may improve next-day sleep architecture for some individuals. Hypothesis (low): Individual thermal routines (hot-cold contrast) may improve sleep onset in specific chronotypes. Next queries Quantify effect sizes for light timing interventions in shift-like schedules. Review evidence quality for popular sleep supplements (safety-first grading). Regenesis digest v1 · iterate with measured outcomes.

\subsection*{Validation checklist}
\begin{itemize}
  \item Verify all nontrivial claims against the original source.
  \item Add explicit citations/DOIs where available.
  \item Mark confidence for each key claim (low/medium/high).
\end{itemize}

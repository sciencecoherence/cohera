\section*{Cosmos Dark Matter Cores As1063 20260223 Publication V1}
\subsection*{Abstract}
This publication provides a structured synthesis for Cosmos Dark Matter Cores As1063 20260223 Publication V1, with claim-to-evidence framing and a validation path for downstream readers.

\subsection*{Keywords}
cosmos, research, publication

\subsection*{Main Content}
% Journal-style manuscript
\title{Large Dark-Matter Cores in AS1063:\\A Coherence-Constrained Interpretation}
\author{Cohera Lab}
\date{February 2026}
\maketitle

\begin{abstract}
We reframe the AS1063 large-core problem through a coherence-constrained lens that distinguishes observational signal, model prior, and geometric inference. Instead of forcing a single mechanism, we formalize a bounded comparison between self-interacting dark-matter (SIDM) and alternative coarse-grained reconstruction effects. The analysis is anchored in distance-based state discrimination and explicit failure tests. \textbf{Falsification Hook:} if posterior support for large-core structure vanishes under jointly perturbed lensing priors and independent profile reconstructions, the coherence-constrained SIDM interpretation is rejected.
\end{abstract}

\paragraph{Keywords} dark matter cores; AS1063; lensing inference; model selection; falsification

\section{Introduction}
Claims of large dark-matter cores in cluster environments are sensitive to reconstruction assumptions. This manuscript provides a cleaner publication baseline: assumptions are explicit, inference metrics are declared, and interpretive claims are bounded by rejection criteria.

\section{Coherence Functional}
Let $\theta$ denote a profile-parameter vector and $\mathcal{L}(\theta)$ the likelihood under lensing constraints. We define a coherence-weighted objective
\begin{equation}
\mathcal{J}(\theta)= -\log \mathcal{L}(\theta) + \lambda\,\Omega(\theta),
\end{equation}
where $\Omega$ penalizes profile incoherence across reconstruction channels and $\lambda>0$ controls regularization strength.

\section{Stability Threshold}
Profile stability is tracked under perturbation set $\mathcal{P}$ (priors, noise model, source-plane assumptions):
\begin{equation}
\Delta_{\mathrm{core}} = \sup_{p\in\mathcal{P}}\left|r_c^{(p)}-r_c^{(0)}\right|.
\end{equation}
A robust core claim requires
\begin{equation}
\Delta_{\mathrm{core}} < \delta_c,
\end{equation}
with pre-registered tolerance $\delta_c$.

\section{Emergent Geometry Protocol}
We compare posterior state families through Bures-type distinguishability between reconstructed profile states $\rho_i$:
\begin{equation}
D_B(\rho_i,\rho_j)=\sqrt{2\left(1-\sqrt{F(\rho_i,\rho_j)}\right)},
\end{equation}
where
\begin{equation}
F(\rho_i,\rho_j)=\left[\operatorname{Tr}\left(\sqrt{\sqrt{\rho_i}\,\rho_j\,\sqrt{\rho_i}}\right)\right]^2.
\end{equation}
If SIDM and non-SIDM reconstructions remain distinguishable under bounded perturbations, the large-core hypothesis retains constrained support; otherwise support collapses.

\section*{Validation and Falsification}
\begin{itemize}
\item Re-run reconstructions with alternative prior families and source selection cuts.
\item Require cross-method agreement (independent lensing pipelines) within $\delta_c$ tolerance.
\item Reject large-core interpretation if stability fails or Bures separation becomes non-significant.
\end{itemize}

\section*{Conclusion}
The AS1063 core claim is treated as a testable constrained hypothesis, not a narrative endpoint. This publication version prioritizes explicit model boundaries and falsifiability over rhetorical certainty.

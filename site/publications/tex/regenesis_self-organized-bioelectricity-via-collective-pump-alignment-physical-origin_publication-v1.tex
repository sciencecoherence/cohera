\section*{Regenesis Self Organized Bioelectricity Via Collective Pump Alignment Physical Origin Publication V1}
\subsection*{Abstract}
This publication provides a structured synthesis for Regenesis Self Organized Bioelectricity Via Collective Pump Alignment Physical Origin Publication V1, with claim-to-evidence framing and a validation path for downstream readers.

\subsection*{Keywords}
regenesis, research, publication

\subsection*{Main Content}
% Auto-promoted publication draft
\title{Self-Organized Bioelectricity via Collective Pump Alignment Physical Origin of Chemiosmosis}
\author{Cohera Lab}
\date{2026-02-23}
\maketitle

\begin{center}\small Thread: Regenesis\end{center}

\begin{abstract}
This manuscript promotes an in-progress regenesis research stream into a publication-ready synthesis. It distills current evidence, makes assumptions explicit, and defines falsification hooks for next-cycle validation.
\end{abstract}

\paragraph{Keywords} regenesis, synthesis, publication

\section{Introduction}
Autodraft · Self-Organized Bioelectricity via Collective Pump Alignment Physical Origin of Chemiosmosis · Cohera Lab Cohera Lab Home Research Cosmos Regenesis Ethos Publications About Autodraft: Self-Organized Bioelectricity via Collective Pump Alignment Physical Origin of Chemiosmosis Date: 2026-02-23 · Thread: regenesis · Status: extracted-draft · Confidence: low-medium Source chatgpt/pdf/Self-Organized\_Bioelectricity\_via\_Collective\_Pump\_Alignment\_Physical\_Origin\_of\_Chemiosmosis.pdf DOI: not detected automatically. Auto summary (preview-based) Self-Organized Bioelectricity via Collective Pump Alignment: Physical Origin of Chemiosmosis Ryosuke Nishide ∗ and Kunihiko Kaneko arXiv:2602.16171v1 [physics.bio-ph] 18 Feb 2026 Niels Bohr Institute, University of Copenhagen, Jagtvej 155 A, Copenhagen N, 2200, Denmark Chemiosmosis maintains life in nonequilibrium through ion transport across membranes, yet the origin of this order remains unclear. We develop a minim Key findings (auto-extracted) Primary topic appears to center on: self, organized, bioelectricity, collective. Source was auto-indexed and text-previewed for rapid triage. Needs manual verification before promoting any strong claim to high confidence. Evidence \& citations Source file: chatgpt/pdf/Self-Organized\_Bioelectricity

\section{Evidence and Claims}
\_via\_Collective\_Pump\_Alignment\_Physical\_Origin\_of\_Chemiosmosis.pdf Extraction scope: 1-2 Abstract/preview extracted automatically. Claim → evidence mapping (auto) Claim: Self-Organized Bioelectricity via Collective Pump Alignment: Physical Origin of Chemiosmosis Ryosuke Nishide ∗ and Kunihiko Kaneko arXiv:2602.16171v1 [physics.bio-ph] 18 Feb 2026 N… Evidence quote: “Self-Organized Bioelectricity via Collective Pump Alignment: Physical Origin of Chemiosmosis Ryosuke Nishide ∗ and Kunihiko Kaneko arXiv:2602.16171v1 [physics.bio-ph] 18 Feb 2026 Niels Bohr Institute, University of Copenhagen, Jagtvej 155 A, Copenhagen N, 2200, Denmark Chemiosmosis maintains life in nonequilibrium through ion transport across membranes, yet the origin of this order remains unclear.” Page hint: 1-2 Claim: We develop a minimal model in which ion pump orientation and the intracellular electrochemical potential mutually reinforce each other. Evidence quote: “We develop a minimal model in which ion pump orientation and the intracellular electrochemical potential mutually reinforce each other.” Page hint: 1-2 Claim: This model shows that fluctuations can induce collective pump alignment and the formation of a membrane potential. Evidence quote: “This model shows that fluctuations can induce collective pump alignment and the formation of a membrane potential.” Page hint: 1-2 Falsification / validation chec

\section{Discussion}
klist Re-read full source and verify the central claim sentence-by-sentence. Cross-check against at least one independent source before promotion. Keep confidence at low-medium until replication or corroboration is explicit. Next queries What is the smallest testable claim from this source?

\section{Validation Hooks}
\begin{itemize}
\item Verify central claims against primary paper and DOI source.
\item Separate measured evidence from interpretation in final revision.
\item Keep confidence conservative until corroboration is explicit.
\end{itemize}

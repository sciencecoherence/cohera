\documentclass[11pt]{article}

% --- Page & typography (tight) ---
\usepackage[a4paper,margin=0.9in]{geometry}
\usepackage{amsmath,amssymb,mathtools}
\usepackage{graphicx}
\usepackage{microtype}
\usepackage{titlesec}
\usepackage{enumitem}
\usepackage{hyperref}
\hypersetup{colorlinks=true, linkcolor=black, urlcolor=black, citecolor=black}
\begin{document}
\section{Coherence-Bounded Collapse and the TCQS Black-Hole Core}

In classical general relativity, gravitational collapse of a massive star leads to the formation
of a trapped region whose interior geometry inevitably evolves toward a curvature singularity.
The Time–Crystalline Quantum Substrate (TCQS) replaces this unphysical divergence with a
finite, maximally coherent phase of the substrate. Collapse terminates not because gravity
ceases, but because the coherence-pressure of the substrate prevents further compaction.
This produces a \emph{coherence-bounded core} of finite radius instead of a singular point.

\subsection{Coherence Pressure vs. Gravitational Pressure}

Let $\mathcal{C}(r)$ denote the coherence density of the substrate.  
Gravitational collapse is driven inward by an effective interior pressure
\begin{equation}
    P_{\mathrm{grav}}(r) \sim \frac{GM^2}{r^4},
\end{equation}
while the substrate resists collapse through a coherence-pressure term
\begin{equation}
    P_{\mathrm{coh}}(r) = k \, \nabla^2 \mathcal{C}(r),
\end{equation}
where $k$ is the TCQS curvature–coherence coupling constant.

Collapse halts when
\begin{equation}
    P_{\mathrm{grav}}(r_{\mathrm{core}}) 
    = 
    P_{\mathrm{coh}}(r_{\mathrm{core}}).
\end{equation}

\subsection{Maximal Coherence and Core Boundary}

At the boundary of the core, the substrate reaches its maximum attainable coherence:
\begin{equation}
    \mathcal{C}(r_{\mathrm{core}}) = \mathcal{C}_{\mathrm{max}},
\end{equation}
and the gradient of coherence vanishes,
\begin{equation}
    \nabla \mathcal{C}(r_{\mathrm{core}}) = 0.
\end{equation}

The Laplacian therefore simplifies to
\begin{equation}
    \nabla^2 \mathcal{C}(r_{\mathrm{core}})
    \simeq
    \frac{\mathcal{C}_{\mathrm{max}}}{r_{\mathrm{core}}^2}.
\end{equation}

Balancing gravitational and coherence-pressure at the core boundary yields
\begin{equation}
    \frac{GM^2}{r_{\mathrm{core}}^4}
    =
    k \frac{\mathcal{C}_{\mathrm{max}}}{r_{\mathrm{core}}^2}.
\end{equation}

Solving for $r_{\mathrm{core}}$ gives the TCQS core-radius formula:
\begin{equation}
    r_{\mathrm{core}}
    =
    \left(
        \frac{GM^2}{k \, \mathcal{C}_{\mathrm{max}}}
    \right)^{1/2}.
\end{equation}

This replaces the classical singularity ($r=0$) with a finite, maximally coherent region.

\subsection{Horizon–Core Ratio as a TCQS Invariant}

The Schwarzschild event-horizon radius is
\begin{equation}
    r_{\mathrm{h}} = \frac{2GM}{c^2}.
\end{equation}

Thus the ratio of horizon size to core size becomes
\begin{equation}
    \frac{r_{\mathrm{h}}}{r_{\mathrm{core}}}
    =
    \frac{2}{c^2}
    \sqrt{
        k \, \mathcal{C}_{\mathrm{max}} \, GM
    }.
\end{equation}

This dimensionless ratio is a TCQS cosmological invariant:  
it encodes the depth of the spacetime ``well,'' the coherence-saturation scale,
and the overall geometric profile of the collapse.

\subsection{Spacetime Flow and the Interior Metric}

Adopting the dynamical ``waterfall'' model of spacetime flow,
the interior infall velocity of the substrate is modeled by
\begin{equation}
    v(r) = c \sqrt{\frac{r_{\mathrm{h}}}{r}},
\end{equation}
with the TCQS-imposed cutoff
\begin{equation}
    v(r_{\mathrm{core}}) = 0.
\end{equation}

This yields a modified Painlevé–Gullstrand metric for the non-singular interior:
\begin{equation}
    ds^2
    =
    -c^2 dt^2
    +
    \left( dr + v(r) \, dt \right)^2
    +
    r^2 d\Omega^2,
\end{equation}
where $v(r)$ remains finite everywhere and vanishes at the coherence-bounded core.

This geometry ensures:
\begin{itemize}
    \item no curvature singularity,
    \item finite maximal density,
    \item a stable time-crystalline core,
    \item and complete information preservation within the coherence structure.
\end{itemize}

\subsection{Information Capacity of the Core}

The finite radius and finite coherence density imply a finite entropy:
\begin{equation}
    S_{\mathrm{core}}
    =
    \frac{\mathcal{C}_{\mathrm{max}}}{\ell_{\mathrm{P}}^3}
    \, V_{\mathrm{core}},
\end{equation}
with
\begin{equation}
    V_{\mathrm{core}}
    =
    \frac{4\pi}{3}
    r_{\mathrm{core}}^3.
\end{equation}

Thus the TCQS coherence-bounded core provides a natural resolution to the
information paradox and eliminates the pathological infinities of classical collapse.
\end{document}
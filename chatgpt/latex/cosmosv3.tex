% =====================================================================================
% SCIENCE OF COHERENCE — COSMOS (Complete but Simple)
% Audience: adults with standard scientific literacy
% Style: ontological clarity + philosophical insight + minimal formalism
% NOTE: No tcolorbox, no exotic packages. Compiles almost everywhere.
% =====================================================================================

\documentclass[11pt]{article}
\usepackage[a4paper,margin=1in]{geometry}
\usepackage[T1]{fontenc}
\usepackage{lmodern}
\usepackage{hyperref}
\usepackage{amsmath,amssymb}
\usepackage{enumitem}

\hypersetup{colorlinks=true,linkcolor=blue,urlcolor=blue}

% ---- Symbols (minimal, TCQS-inspired but acronym-light) ----
\newcommand{\Substrate}{S}                 % only arena
\newcommand{\Htot}{H_{\mathrm{tot}}}       % global generator/rule
\newcommand{\Update}{\mathcal{U}}          % update map on S (no unitary assumption)
\newcommand{\Proj}{\Lambda}                % accessibility / projection
\newcommand{\Emerg}{\Omega}                % emergent effective description
\newcommand{\Integr}{\Delta}               % recursive integration / feedback

\title{Science of Coherence — COSMOS}
\author{}
\date{}

\begin{document}
\maketitle

% =====================================================================================
\section*{0. Page Identity (Hero + Thesis)}
COSMOS is the physics-facing branch of Science of Coherence. It explores a simple proposition:
\emph{spacetime and fields are not assumed as fundamental; they are stable emergent descriptions of a deeper coherent substrate.}
The framework is inspired by time-crystalline recurrence and holographic organization, but aims to remain conceptually minimal and structurally test-oriented.

\bigskip
\noindent\textbf{Thesis in one line.}
A single substrate $\Substrate$ evolves under a global rule $\Htot$; partial accessibility $\Proj$ produces stable local sectors; an emergent description $\Emerg$ organizes those sectors into spacetime-and-field narratives; recursive integration $\Integr$ stabilizes what persists.

% =====================================================================================
\section*{1. Ontology (What we take as fundamental)}
\subsection*{1.1 One arena: the substrate}
We begin with ontological austerity. There is one fundamental arena of configurations, denoted $\Substrate$.
Everything else---space, time, matter, measurement, observers---is treated as a \emph{derived} mode of description.
This is not a denial of spacetime physics; it is a claim about explanatory order.

\subsection*{1.2 Coherence as a primitive constraint}
The substrate is assumed to support a notion of coherence: an internal compatibility structure that distinguishes
configurations that can stably co-exist from configurations that mutually frustrate.
Coherence is not introduced as ``what an observer measures'' but as what makes stable structure possible in the first place.

\subsection*{1.3 Why holography appears naturally}
If spacetime is emergent, then any spacetime-based description is a summary of deeper correlations.
In this sense, the emergent world is inherently holographic: it is an encoding of global structure into accessible sectors,
rather than a fundamental stage on which reality happens.

\subsection*{1.4 The philosophical stance (explicit, but disciplined)}
This framework treats ``laws of physics'' less as external commandments and more as
\emph{stable self-consistency patterns} of a recurring coherent system.
The philosophical claim is modest: persistence requires compatibility.
What remains is what can keep remaining.

% =====================================================================================
\section*{2. Architecture (The minimal components)}
\subsection*{2.1 The five objects}
We use five objects as a minimal architecture:
\begin{itemize}[leftmargin=*]
  \item $\Substrate$: the only fundamental arena (state space).
  \item $\Htot$: the global generator (the deep rule governing substrate evolution).
  \item $\Proj$: accessibility/projection (how partial sectors arise).
  \item $\Emerg$: emergent description (how spacetime/fields/histories appear as an effective model).
  \item $\Integr$: recursive integration/feedback (how persistence modifies constraints).
\end{itemize}

\subsection*{2.2 The closure loop}
The architecture is best read as a closure loop:
\[
\Htot \rightarrow \Substrate \rightarrow \Proj \rightarrow \Emerg \rightarrow \Integr \rightarrow \Htot.
\]
This is not merely symbolic. It encodes a scientific posture:
the emergent world is not passive; it can constrain the conditions of its own stability.

% =====================================================================================
\section*{3. Mechanisms (How stable structure appears)}
\subsection*{3.1 Recurrence and the appearance of invariants}
Time-crystalline recurrence is used here as a structural mechanism:
when evolution has recurring structure, patterns that remain compatible across cycles persist.
Those persistent patterns behave like effective invariants---the practical core of what we call ``laws.''

\subsection*{3.2 Accessibility and the appearance of locality}
Locality is treated as a consequence of restricted accessibility.
The map $\Proj$ produces a derived sector that contains only a subset of the full substrate information.
Within such a sector, the world can appear classical and local because the sector suppresses or averages over global correlations.

\subsection*{3.3 Holographic entanglement as the organizing principle}
The organizing principle is that \emph{correlation structure determines effective geometry}.
Rather than importing a specific AdS/CFT formula, we state a weaker but more general requirement:

\begin{quote}
The preferred emergent description is the one in which stable correlation ``bottlenecks'' in accessible sectors
correspond to effective boundaries, areas, and distances in the emergent geometry.
\end{quote}

This statement is intentionally substrate-native:
it does not assume a pre-existing bulk spacetime; it says that whatever spacetime emerges must be the
most coherent and stable summary of correlation structure.

\subsection*{3.4 Feedback and memory}
Recursive integration $\Integr$ represents the mechanism by which persistent structure can stabilize its own conditions.
This is the difference between a universe that merely evolves and a universe that accumulates robust organization:
constants, effective laws, long-lived quasi-particles, and potentially complex adaptive subsystems.

% =====================================================================================
\section*{4. Minimal formalism (Only what is necessary)}
\subsection*{4.1 Typed maps (no heavy assumptions)}
We keep the formal layer minimal and explicit about roles:
\begin{align*}
&\textbf{Update on the substrate:} && \Update:\Substrate \to \Substrate \quad (\text{induced by } \Htot) \\
&\textbf{Accessibility/projection:} && \Proj:\Substrate \to \Substrate_{\mathrm{loc}} \\
&\textbf{Emergent description:} && \Emerg:\Substrate_{\mathrm{loc}} \to \mathcal{D}_{\mathrm{eff}} \\
&\textbf{Integration/feedback:} && \Integr:(\Substrate_{\mathrm{loc}},\mathcal{D}_{\mathrm{eff}})\to \text{constraints on }(\Update,\Proj)
\end{align*}
Here $\mathcal{D}_{\mathrm{eff}}$ denotes the space of effective descriptions (spacetime, fields, histories) as a modeling layer.

\subsection*{4.2 One stability requirement (kept schematic)}
A stable sector is one where the projected story remains consistent under the update:
\[
\Proj\circ \Update \approx \Update_{\mathrm{loc}}\circ \Proj.
\]
The meaning of $\approx$ is intentionally deferred: it becomes precise only after choosing a representation for $\Substrate$
and a distance/accuracy notion appropriate to the accessibility structure.

% =====================================================================================
\section*{5. What this aims to explain (and what it does not claim)}
\subsection*{5.1 Targets}
A successful instantiation should reproduce:
\begin{itemize}[leftmargin=*]
  \item quantum behavior in appropriate regimes,
  \item relativistic causal structure as an emergent symmetry,
  \item thermodynamic irreversibility as a sector-level consequence of restricted access,
  \item effective field dynamics as stable excitations of accessible structure,
  \item robust classicality as a stability phenomenon (not an assumption).
\end{itemize}

\subsection*{5.2 Boundaries}
This document is an architectural and mechanistic scaffold.
It does not claim a completed microscopic model until explicit choices are made for:
(1) the substrate representation, (2) the update rule, (3) the projection/accessibility map,
and (4) the operational correlation functional that anchors ``holographic entanglement'' in the accessible sector.

% =====================================================================================
\section*{6. Website Modules (Gemini-friendly page blocks)}
\subsection*{Module A — COSMOS (short)}
COSMOS explores how spacetime and fields can emerge from a deeper coherent substrate. It uses a minimal architecture:
a global rule on a single arena, partial accessibility that yields stable local sectors, and an emergent description
selected by correlation structure and stabilized by recurrence and feedback.

\subsection*{Module B — The architecture in one line}
Global dynamics shapes the substrate; projection produces accessible sectors; the emergent description organizes those sectors
into spacetime-and-field narratives; feedback stabilizes persistence.

\subsection*{Module C — Holographic entanglement (plain language)}
Holographic entanglement means that geometry is a stable summary of correlations.
Boundaries, areas, and distances in the emergent world track where correlations bottleneck and how separations behave under stability constraints.

\subsection*{Module D — Scientific posture}
The framework becomes predictive when explicit forms are chosen for the substrate, the update rule, the accessibility map,
and the operational correlation functional. Those choices turn the architecture into computable stability thresholds and emergent geometry.

\end{document}

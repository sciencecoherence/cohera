\documentclass[11pt]{article}

% --- Page & typography (tight) ---
\usepackage[a4paper,margin=0.9in]{geometry}
\usepackage{amsmath,amssymb,mathtools}
\usepackage{graphicx}
\usepackage{microtype}
\usepackage{titlesec}
\usepackage{enumitem}
\usepackage{hyperref}
\hypersetup{colorlinks=true, linkcolor=black, urlcolor=black, citecolor=black}

% Compact spacing
\setlength{\parskip}{3pt}
\setlength{\parindent}{0pt}
\setlength{\abovedisplayskip}{6pt}
\setlength{\belowdisplayskip}{6pt}
\setlength{\abovedisplayshortskip}{4pt}
\setlength{\belowdisplayshortskip}{4pt}

% Title spacing (tight)
\titleformat{\section}{\large\bfseries}{\thesection}{0.5em}{}[\vspace{-0.25em}\titlerule]
\titlespacing*{\section}{0pt}{0.6em}{0.35em}
\titleformat{\subsection}{\normalsize\bfseries}{\thesubsection}{0.5em}{}
\titlespacing*{\subsection}{0pt}{0.45em}{0.25em}

% Lists (tight)
\setlist[itemize]{leftmargin=1.1em,label=\textbullet,nosep}

% --- Header (logo + series line) ---
\newcommand{\tcqsheader}{
  \noindent
  \begin{minipage}[t]{0.74\linewidth}
    {\Large\bfseries TCQS Surpasses Infodynamics:}\\[-1pt]
    {\Large\bfseries A Time-Crystalline Mechanism for Informational Entropy Reduction}\\[2pt]
    \textit{Interface Note \,·\, TCQS Series}
  \end{minipage}\hfill
  \begin{minipage}[t]{0.22\linewidth}
    \raggedleft
    % Replace with your actual logo filename (vector preferred).
    % The height limit keeps it small; adjust to 9mm/8mm if needed.
    \includegraphics[height=10mm,keepaspectratio]{tcqs_logo.pdf}
  \end{minipage}

  \vspace{0.5em}\hrule\vspace{0.6em}
}

\begin{document}
\thispagestyle{empty}

\tcqsheader

\textbf{Abstract —}
Recent “infodynamics” proposals posit a monotonic decrease of information entropy as circumstantial evidence that the universe is computational or simulation-like. Within the Time-Crystalline Quantum Substrate (TCQS), this trend is not postulated but \emph{derived}: a Floquet-periodic, self-referential measure–correct–re-encode cycle enforces informational compression as a dynamical fixed point of the substrate.

\section{Statement and Context}
The infodynamic law is commonly expressed as
\begin{equation}
\frac{dS_{\mathrm{info}}}{dt}\ \le\ 0,
\label{eq:infodynamics}
\end{equation}
interpreted as universal informational compression. While compelling, Eq.~\eqref{eq:infodynamics} is \emph{phenomenological}: no substrate-level mechanism is specified that guarantees the inequality across scales.

\section{TCQS Mechanism: Coherence-Driven Compression}
In TCQS, the global state $\rho$ evolves under a time-crystalline (Floquet) schedule that implements a coherence-restoration pipeline:
\[
\text{measure} \;\rightarrow\; \text{correct} \;\rightarrow\; \text{re-encode}.
\]
Let $\mathcal{F}_{\mathrm{info}}(\rho)$ denote the informational free-energy functional on the informational geometry induced by the substrate. The unified coherence law,
\begin{equation}
\frac{d}{dt}\,\mathcal{C}(t)\;=\;-\nabla_{\rho}\,\mathcal{F}_{\mathrm{info}}(\rho),
\label{eq:coherence-law}
\end{equation}
implies each Floquet update reduces $\mathcal{F}_{\mathrm{info}}$ and, by construction, the associated informational-entropy functional $S_{\mathrm{info}}[\rho]$. Thus Eq.~\eqref{eq:infodynamics} emerges as a \emph{corollary} of Eq.~\eqref{eq:coherence-law}.

\section{Why TCQS Surpasses Infodynamics}
\begin{itemize}
  \item \textbf{Mechanism $\boldsymbol{>}$ Phenomenology.} Infodynamics \emph{observes} $dS_{\mathrm{info}}/dt\!\le\!0$; TCQS \emph{produces} it via a self-referential Hamiltonian, coherence gradients, and Floquet updates.
  \item \textbf{Generative Simulation.} Infodynamics \emph{infers} simulation-like behavior from compression; TCQS shows a self-updating time-crystal substrate is \emph{intrinsically} simulation-like: a quantum computation that generates spacetime/matter through iterative re-encoding.
  \item \textbf{First-Principles Derivation.} In TCQS, informational compression is a dynamical fixed point of the substrate’s geometry, not an external postulate.
\end{itemize}

\section{Implications}
TCQS supplies the missing physical architecture—operator (self-referential Hamiltonian), engine (Floquet evolution), geometry (informational metric), and objective (coherence maximization)—that renders simulation-like behavior the \emph{structural} form of reality rather than an interpretive add-on.

\vspace{0.3em}
\hrule
{\footnotesize
\emph{TCQS Interface Notes} — concise communications linking coherence dynamics and informational geometry. %
Notes: (i) Eq.~\eqref{eq:infodynamics} summarizes infodynamic claims; (ii) Eq.~\eqref{eq:coherence-law} is the TCQS unified coherence law specialized to informational geometry.}

\end{document}

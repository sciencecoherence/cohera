\documentclass[11pt,onecolumn]{article}

%------------------------------------------------
% PACKAGES
%------------------------------------------------
\usepackage[a4paper,margin=1in]{geometry}
\usepackage{amsmath,amsfonts,amssymb,amsthm}
\usepackage{physics}
\usepackage{graphicx}
\usepackage{authblk}
\usepackage{hyperref}
\usepackage{bm}
\usepackage{cite}
\usepackage{mathtools}
\usepackage{xcolor}
\usepackage{physics}
\usepackage{tcolorbox}

%------------------------------------------------
% TITLE & AUTHORS
%------------------------------------------------
\title{\textbf{Black Hole Information Preservation in the Time-Crystalline Quantum Substrate}}

\author[1]{Julien Steff}
\affil[1]{Science Coherence Institute}

\date{\today}

%------------------------------------------------
% DOCUMENT
%------------------------------------------------
\begin{document}

\maketitle

\begin{abstract}
The black hole information paradox arises when classical general relativity and semiclassical quantum field theory are combined under the assumptions of singularities, thermal Hawking radiation, and causal disconnection at the event horizon. 
The Time-Crystalline Quantum Substrate (TCQS) framework rejects these premises and instead models spacetime as a discrete, coherence-preserving, self-updating quantum medium governed by Floquet-like dynamics.
Within this ontology, black holes are not singularities but finite, high-coherence compression phases that temporarily store and redistribute information through horizon-level phase memory, entanglement networks, and non-thermal evaporation channels.
We prove that the global substrate evolution remains unitary, derive the corresponding coherence-conservation law, and show why no firewall, remnants, or nonlocal violations are required.
Therefore, the paradox dissolves—not by resolving contradictions, but by demonstrating that the assumptions generating it never arise in TCQS physics.
\end{abstract}

\vspace{1cm}

%------------------------------------------------
\section{Introduction: Why the Paradox Exists}
%------------------------------------------------

The information paradox is traditionally posed as a conflict between:
\begin{enumerate}
    \item unitary quantum mechanics,
    \item classical general relativity,
    \item semiclassical Hawking radiation.
\end{enumerate}

If Hawking radiation is perfectly thermal, the von Neumann entropy of the universe decreases:
\begin{equation}
S_{\text{final}} < S_{\text{initial}},
\end{equation}
implying information destruction.

However, this requires:
\begin{itemize}
\item a true singularity,
\item an absorbing event horizon,
\item an inert vacuum.
\end{itemize}

The TCQS rejects all three.

%------------------------------------------------
\section{Foundational Postulate of TCQS}
%------------------------------------------------

The universe is modeled as a discrete, self-updating quantum substrate with periodic evolution:
\begin{equation}
\ket{\Psi(t+T)} = U_{F} \ket{\Psi(t)},
\end{equation}
where $U_{F}$ is a Floquet operator uniquely determined by the substrate Hamiltonian $H_{\text{sub}}$.

The substrate preserves global coherence:
\begin{equation}
\frac{d}{dt} C_{\text{tot}} = 0,
\end{equation}
where $C_{\text{tot}}$ denotes total coherence-information content.

Thus, unitarity is not assumed—it is structurally enforced.

%------------------------------------------------
\section{Black Holes as Coherence Compression Phases}
%------------------------------------------------

Instead of curvature singularities, TCQS models gravitational collapse as increasing local coherence density:
\begin{equation}
\rho_C(\mathbf{x}) = \rho_{0} + \Delta \rho_{C}(\mathbf{x}),
\end{equation}
where matter generates coherence gradients in the substrate.

A black hole forms when:
\begin{equation}
\nabla \rho_C \rightarrow \rho_{C,\text{crit}},
\end{equation}
but remains finite and nonsingular.

Thus:
\begin{equation}
\rho_{C,\text{max}} < \infty,
\end{equation}
eliminating singularities entirely.

%------------------------------------------------
\section{Event Horizon as Coherence Boundary}
%------------------------------------------------

The horizon is not an information erasure boundary, but a phase interface:
\begin{equation}
\Delta \phi_{\text{in-out}} = \text{constant},
\end{equation}
where $\phi$ denotes substrate phase.

The horizon stores information through entanglement surface modes:
\begin{equation}
I_{\text{horizon}} = \log \dim \mathcal{H}_{\partial \Omega}.
\end{equation}

This reproduces Bekenstein-Hawking entropy without requiring spacetime geometry to be fundamental.

%------------------------------------------------
\section{Hawking Radiation is Non-Thermal in TCQS}
%------------------------------------------------

Because the substrate is coherence-preserving:
\begin{equation}
S_{\text{rad}}(t) = S_{\text{BH}}(t),
\end{equation}
ensuring unitary information flow.

Outgoing radiation carries phase correlations:
\begin{equation}
\ket{\psi_{\text{rad}}} = \sum_{i} \alpha_{i} \ket{i_{\text{out}}} \otimes \ket{i_{\text{in}}},
\end{equation}
not random thermal states.

Thus the Page curve never violates unitarity.

%------------------------------------------------
\section{Coherence Conservation Law}
%------------------------------------------------

Define total system coherence:
\begin{equation}
C_{\text{tot}} = C_{\text{BH}} + C_{\text{rad}} + C_{\text{vac}},
\end{equation}
then TCQS evolution enforces:
\begin{equation}
\frac{dC_{\text{tot}}}{dt} = 0.
\end{equation}

During evaporation:
\begin{align}
\frac{dC_{\text{BH}}}{dt} &< 0, \\
\frac{dC_{\text{rad}}}{dt} &> 0,
\end{align}
but the sum remains constant.

Information is redistributed, not erased.

%------------------------------------------------
\section{Global Unitarity Without Firewalls or Remnants}
%------------------------------------------------

Since the substrate evolution is globally unitary:
\begin{equation}
U_{F}^\dagger U_{F} = \mathbb{I},
\end{equation}
there is no need for:
\begin{itemize}
\item Planck-scale remnants,
\item nonlocal violations,
\item firewall energetic discontinuities.
\end{itemize}

The paradox dissolves because its assumptions never occur.

%------------------------------------------------
\section{Why the Paradox Never Forms in TCQS}
%------------------------------------------------

The paradox requires:
\begin{enumerate}
\item information-erasing horizons,
\item thermal radiation,
\item singularities,
\item non-unitary evolution.
\end{enumerate}

TCQS supplies:
\begin{enumerate}
\item horizon phase memory,
\item correlated radiation,
\item finite coherence cores,
\item enforced unitarity.
\end{enumerate}

Thus:
\begin{equation}
\mathcal{I}_{\text{initial}} = \mathcal{I}_{\text{final}}.
\end{equation}

%------------------------------------------------
\section{Conclusion}
%------------------------------------------------

In the TCQS framework, black holes are temporary coherence reservoirs within a unitary, time-crystalline substrate.
The horizon functions as a dynamic information interface, and evaporation returns stored correlations through non-thermal radiation.
Since total coherence-information is conserved, no information is ever lost, and the information paradox ceases to exist—not because it is solved, but because the universe never enters a regime where it forms.

\vspace{0.3cm}
\noindent \textbf{Therefore:}  
\textit{Black holes do not destroy information. They redistribute it.}

%------------------------------------------------
% END DOCUMENT
%------------------------------------------------

\end{document}

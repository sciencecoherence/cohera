\documentclass[11pt]{article}

% Packages
\usepackage{amsmath, amssymb, amsfonts}
\usepackage{physics}
\usepackage{hyperref}
\usepackage{bm}
\usepackage{geometry}
\usepackage{amsmath, amsfonts, amssymb, amsthm}
\usepackage{graphicx}
\usepackage{authblk}
\usepackage{hyperref}
\usepackage{bm}
\usepackage{cite}
\usepackage{mathtools}
\usepackage{xcolor}
\usepackage{tcolorbox}
\usepackage{titling}

\geometry{
  a4paper,
  margin=0.8in,
  top=1in
}
\begin{document}
% Move title closer to the top
\setlength{\droptitle}{-5em}  % try -1em, -2em, -3em etc.
\setlength{\skip\footins}{2em}

\title{\textbf{Observation Without an Observer:}\\[0.5em]
Consciousness and Local Coherence Stabilization}

\author{Julien Steff}
\affil[1]{Science Coherence Institute}

\date{} % no date

\maketitle
\begin{abstract}
Consciousness and observation emerge from coherence dynamics within the time-crystalline informational continuum. This article presents a precise formulation of consciousness as a high-coherence dynamical phase of a self-organizing intelligent computational lattice, and defines the observer as a local coherence stabilization rather than a metaphysical or independent entity. This new formalism provide a non-dual, observer-invariant interpretation of quantum mechanics while maintaining coherence-theoretic clarity. Information becomes self-revealing within coherent phases of the substrate of reality, eliminating the need for a separate observer and resolving the conceptual tension between subject and object.
\end{abstract}

\section{Introduction}

Traditional interpretations of quantum mechanics attribute a privileged role to the observer. This assumption, often implicit, suggests an entity capable of collapsing or determining quantum states through the act of measurement. In contrast, both non-dual philosophy and our modern coherence-based frameworks challenge this view.

The TCQS---a time-crystalline, self-referential substrate underlying physical reality---provides a rigorous foundation for reconciling these perspectives. In this framework, ``observer'' and ``observation'' are not properties of an external agent but dynamical phases of the substrate itself. No independent subject is required.

This article formalizes these ideas and explains why observation occurs without an observer.

\section{The Substrate}

Let
\[
S
\]
denote the \textbf{vacuum-coherence substrate}: a time-crystalline, self-referential informational continuum that functions as the generative background of physical reality with periodic update dynamics continually reconstructing spacetime, matter, and causal structure exhibiting:

\begin{itemize}
    \item \textbf{stable regimes} — where coherence persists across update cycles, giving rise to self-maintaining structures such as physical laws, particles, geometry, and operational identity,\footnote{Stability is defined operationally — as persistence across coherence update cycles — not metaphysically as eternal existence.}
    \item \textbf{metastable regimes} — where coherence is temporary, producing transient phenomena such as experience, perception, decoherence events, and state transitions.
\end{itemize}


The substrate therefore supports both global coherent phases (associated with consciousness) and localized coherence patterns (associated with observers and measurement events).

\section{Consciousness}

We define consciousness as follows:

\[
\boxed{
\textbf{Consciousness} \equiv 
\text{high-coherence dynamical phase of the substrate } S
}
\]

\vspace{0.3cm}
This definition captures three essential features:

\begin{itemize}
    \item Consciousness is not an entity but a \emph{phase}\footnote{Here “phase” denotes a dynamically sustained coherence regime of the substrate—an organizational pattern--characterized by specific stability, symmetry, and information-flow properties--rather than an entity or substance.}.
    \item It arises from sustained coherence density within $S$ when the substrate’s informational vibrations fall into mutual resonance, a unified field of awareness forms
    \item It possesses intrinsic self-awareness because the coherent phase contains information about its own state.
\end{itemize}

\subsection*{Verbal formulation}

\begin{quote}
\textit{Consciousness is a high-coherence dynamical phase of the substrate, aware of its own state, in which information reveals itself without a separate observer.}
\end{quote}

\section{The Observer}

Classically, an observer is treated as an entity separate from the observed system. In the TCQS, this separation dissolves.

We define the observer as follows:
\[
\boxed{
\textbf{Observer} \equiv 
\textbf{Local coherence stabilization of the substrate}
}
\]

At a spacetime point or region $x$:
\[
\mathcal{O}(x) := \text{local stabilization of } S \text{ into a high-coherence pattern.}
\]

An observer is therefore not an entity but a \emph{local condition} of the substrate.

\section{Observation as Information Self-Revelation}

Observation corresponds to the revelation or resolution of information within a coherent phase of the substrate.

Formally, let $I$ denote informational gradients within the substrate. In a high-coherence phase:
\[
\nabla I \to 0,
\]
meaning that information becomes self-resolved without requiring any external agent.

Thus,

\[
\boxed{
\textbf{Observation} \equiv \text{self-revelation of information within coherent phases of } S.}
\]

\section{Non-dual Implications for Quantum Mechanics}

This paper implies that quantum mechanics is inherently observer-invariant and does not require an external or metaphysical observer. Measurement corresponds to local coherence stabilization\footnote{a region that was previously fluctuating or metastable becomes temporarily stabilized. That local stabilization is what appears, from within physics, as an observation or measurement.}  The so-called “observer” is therefore not an entity but the transient region in which this stabilization occurs.

\begin{itemize}
    \item The “collapse”\footnote{“collapse” does not describe a physical discontinuity, but the temporary locking of a region of the substrate into a stable coherent state. From within physics, this appears as a definite measurement outcome.} reflects local coherence locking within \(S\),
    \item the “observer” is the locus of that temporary stabilization,
    \item There is no entity behind the observation—nothing outside the substrate.
\end{itemize}

Because both “observer” and “observed” are internal modulations of the same
coherent field, the classical subject–object division dissolves. Observation is
not performed by a separate self, but is an intrinsic property of coherence
itself. Consequently:

\begin{itemize}
    \item there is no independent observer,
    \item experience is not owned by a personal self,
    \item observation arises from coherence, not from a perceiving subject.
\end{itemize}

This provides a physically grounded account of measurement, resolves the
observer-dependence paradox, and aligns with non-dual philosophical traditions—
including Buddhist \textit{anattā} and self-referential models of consciousness—
without importing metaphysical assumptions.



\section{The One-State Paradox and Its Resolution}

A common objection to observer-independent formulations of reality is the \textit{one-state paradox}: if no external observer exists to discriminate or extract information, why does the universe not collapse into a single, undifferentiated informational state?

Formally, \( I \) denote informational structure within the substrate \( S \). The paradox assumes:
\[
\text{no observer} \;\Rightarrow\; \nabla I = 0 \;\Rightarrow\; I = \text{constant},
\]
implying no internal differentiation, structure, or physical plurality.

This inference fails because it incorrectly treats observers as the source of informational articulation. In the present framework, information within \( S \) is not revealed by observers but \textit{generated, sustained, and transformed} by the substrate’s own coherence dynamics. The time-crystalline evolution of \( S \) continually produces informational variation, preventing global homogenization:

\[
\nabla I \neq 0 \quad \text{throughout evolving regions of } S.
\]

Differentiation is therefore an intrinsic feature of the substrate, not a consequence of measurement. Even in the absence of observers, fluctuations, transitions, and internal interactions prevent \( \nabla I \) from vanishing globally. Reality maintains multiplicity and physical distinctness because coherence is actively redistributed, reorganized, and renewed across update cycles,  informational structure persists, evolves, and remains self-articulating. 

The paradox dissolves once observation is understood not as an external act, but as a \textit{self-resolution of information within coherent phases of \( S \)}. The universe does not require observers to avoid becoming a single state—it never depended on them to generate informational diversity in the first place.


\section{Conclusion}

This framework unifies several traditionally separate ideas:

\begin{itemize}
    \item the observer-independence of quantum mechanics,
    \item the phenomenology of consciousness,
    \item non-dual philosophy,
    \item and the mathematical structure of coherence dynamics.
\end{itemize}

The TCQS eliminates the need for an ontologically privileged observer and locates both experience and observation within the dynamics of the substrate itself.

No separate observer exists. The substrate is observing itself through its own coherent configurations. Reality does not require an observer to become differentiated — coherence itself is the universe observing, sustaining, and revealing its own informational structure, the universe is not seen — it is self-seeing.

\begin{tcolorbox}[colback=black!5, colframe=black!60, arc=3pt, boxrule=0.6pt]
\centering
\textit{Observation is coherence knowing itself.}
\end{tcolorbox}


\end{document}

\documentclass[11pt]{article}

\usepackage{amsmath, amssymb, amsfonts}
\usepackage{bm}
\usepackage{geometry}
\geometry{margin=2.5cm}
\setlength{\parskip}{0.4em}
\setlength{\parindent}{0pt}

\begin{document}

\title{TCQS Master Equation Handbok V1}
\large {}
\author{}
\date{}
\maketitle

\tableofcontents

%=========================================================
\section{Fundamental Substrate Dynamics}
%=========================================================

\subsection{Time-Crystalline Update Equation}

\textbf{Equation}
\begin{equation}
    S_{n+1} = \mathcal{U}(S_n)
\end{equation}

\textbf{Physical meaning:}
Discrete intrinsic update of the substrate state.

\textbf{Symbol dictionary:}
\begin{itemize}
    \item $S_n$: substrate state at discrete update step $n$.
    \item $\mathcal{U}$: intrinsic update operator of the time-crystalline substrate.
\end{itemize}

\textbf{Derivation status:} Fundamental axiom.

\textbf{Domain of validity:} Fundamental pre-geometric regime (single update scale).

\subsection{Self-Referential Floquet Operator}

\textbf{Equation}
\begin{equation}
    \mathcal{U} = \exp\!\big[-i\,H[S]\,\tau\big]
\end{equation}

\textbf{Physical meaning:}
Periodic update generated by a state-dependent Hamiltonian.

\textbf{Symbol dictionary:}
\begin{itemize}
    \item $H[S]$: Hamiltonian functional of the substrate state $S$.
    \item $\tau$: intrinsic period of the time-crystalline cycle.
\end{itemize}

\textbf{Derivation status:} Fundamental axiom (Floquet representation).

\textbf{Domain of validity:} Description at one intrinsic period $\tau$.

\subsection{Substrate Self-Hamiltonian}

\textbf{Equation}
\begin{equation}
    H = H[S]
\end{equation}

\textbf{Physical meaning:}
The generator of evolution depends functionally on the substrate configuration.

\textbf{Symbol dictionary:}
\begin{itemize}
    \item $H$: generator of intrinsic evolution.
    \item $S$: full substrate configuration.
\end{itemize}

\textbf{Derivation status:} Fundamental axiom (self-referential dynamics).

\textbf{Domain of validity:} Substrate-level (prior to emergent fields/particles).

\subsection{Continuous-Time Approximation}

\textbf{Equation}
\begin{equation}
    \frac{dS}{dt} = -i\,[H(S),\, S]
\end{equation}

\textbf{Physical meaning:}
Continuous-time limit of the discrete update equation.

\textbf{Symbol dictionary:}
\begin{itemize}
    \item $[A,B] = AB - BA$: commutator.
    \item $H(S)$: state-dependent Hamiltonian.
\end{itemize}

\textbf{Derivation status:} Derived from $S_{n+1} = \mathcal{U}(S_n)$ for $\tau \to 0$.

\textbf{Domain of validity:} Coarse-grained limit over many update cycles.

\subsection{Phase-Alignment Condition}

\textbf{Equation}
\begin{equation}
    \partial_t \phi = 0
\end{equation}

\textbf{Physical meaning:}
Local stationarity of the substrate phase (phase-locked region).

\textbf{Symbol dictionary:}
\begin{itemize}
    \item $\phi$: local phase associated with the substrate degrees of freedom.
\end{itemize}

\textbf{Derivation status:} Derived as stationary solution of the phase dynamics implied by $H(S)$.

\textbf{Domain of validity:} Regions where coherence is stabilized.

\subsection{Stabilized Evolution with Correction Term}

\textbf{Equation}
\begin{equation}
    \frac{dS}{dt} = -i\,[H(S), S] - \Gamma(S)
\end{equation}

\textbf{Physical meaning:}
Effective dynamics including non-unitary correction enforcing stability.

\textbf{Symbol dictionary:}
\begin{itemize}
    \item $\Gamma(S)$: stabilizing/dissipative functional of $S$.
\end{itemize}

\textbf{Derivation status:} Phenomenological, from coarse-graining over unresolved modes.

\textbf{Domain of validity:} Mesoscopic and macroscopic effective descriptions.

%=========================================================
\section{Coherence Quantities}
%=========================================================

\subsection{Coherence Scalar in State Space}

\textbf{Equation}
\begin{equation}
    C : \mathcal{P} \to \mathbb{R}_{\ge 0}, \qquad C = C(p)
\end{equation}

\textbf{Physical meaning:}
Scalar measure of coherence associated with a point $p$ in substrate state space $\mathcal{P}$.

\textbf{Symbol dictionary:}
\begin{itemize}
    \item $\mathcal{P}$: space of substrate micro-configurations.
    \item $C(p)$: coherence level at $p$.
\end{itemize}

\textbf{Derivation status:} Fundamental definition.

\textbf{Domain of validity:} All substrate configurations.

\subsection{Spatial Coherence Field}

\textbf{Equation}
\begin{equation}
    C(x) = C\big(p(x)\big)
\end{equation}

\textbf{Physical meaning:}
Coherence scalar expressed as a field over emergent spatial coordinates.

\textbf{Symbol dictionary:}
\begin{itemize}
    \item $x$: emergent spatial coordinate.
    \item $p(x)$: substrate configuration associated with $x$.
\end{itemize}

\textbf{Derivation status:} Derived by projection from $\mathcal{P}$ to emergent space.

\textbf{Domain of validity:} Regime where an emergent spatial manifold is well-defined.

\subsection{Coherence Gradient}

\textbf{Equation}
\begin{equation}
    \bm{\nabla} C(x)
\end{equation}

\textbf{Physical meaning:}
Spatial rate of change of the coherence field.

\textbf{Symbol dictionary:}
\begin{itemize}
    \item $\bm{\nabla}$: gradient with respect to emergent spatial coordinates $x$.
\end{itemize}

\textbf{Derivation status:} Standard differential definition applied to $C(x)$.

\textbf{Domain of validity:} Smooth regions of $C(x)$.

\subsection{Coherence Continuity Equation}

\textbf{Equation}
\begin{equation}
    \partial_t C(x,t) + \bm{\nabla} \cdot \bm{J}_C(x,t) = 0
\end{equation}

\textbf{Physical meaning:}
Local conservation of coherence density under coherent flow.

\textbf{Symbol dictionary:}
\begin{itemize}
    \item $\bm{J}_C$: coherence current density.
\end{itemize}

\textbf{Derivation status:} Postulated conservation law (or derived from symmetry of $H[S]$).

\textbf{Domain of validity:} Regimes where coherence is neither created nor destroyed by external driving.

\subsection{Coherence Transport Law}

\textbf{Equation}
\begin{equation}
    \bm{J}_C = -D_C\,\bm{\nabla} C
\end{equation}

\textbf{Physical meaning:}
Effective diffusive transport of coherence.

\textbf{Symbol dictionary:}
\begin{itemize}
    \item $D_C$: effective coherence diffusivity.
\end{itemize}

\textbf{Derivation status:} Phenomenological constitutive relation.

\textbf{Domain of validity:} Near-equilibrium or weak-gradient regimes.

%=========================================================
\section{Coherence Thermodynamics}
%=========================================================

\subsection{Informational Free Energy Functional}

\textbf{Equation}
\begin{equation}
    \mathcal{F}[C] = \mathcal{E}[C] - \Theta\,\mathcal{S}[C]
\end{equation}

\textbf{Physical meaning:}
Effective free energy associated with coherence distribution.

\textbf{Symbol dictionary:}
\begin{itemize}
    \item $\mathcal{E}[C]$: effective energy functional of $C$.
    \item $\mathcal{S}[C]$: entropy-like functional of $C$.
    \item $\Theta$: effective coherence temperature parameter.
\end{itemize}

\textbf{Derivation status:} Fundamental ansatz for variational principle.

\textbf{Domain of validity:} Mesoscopic coarse-grained description.

\subsection{Coherence Restoration Law}

\textbf{Equation}
\begin{equation}
    \partial_t C(x,t) = -\kappa\,\frac{\delta \mathcal{F}}{\delta C(x,t)}
\end{equation}

\textbf{Physical meaning:}
Relaxational dynamics driving $C$ down the free-energy functional.

\textbf{Symbol dictionary:}
\begin{itemize}
    \item $\kappa > 0$: relaxation constant.
    \item $\delta \mathcal{F}/\delta C$: functional derivative of $\mathcal{F}$ with respect to $C$.
\end{itemize}

\textbf{Derivation status:} Phenomenological gradient-flow equation.

\textbf{Domain of validity:} Overdamped regime, slow dynamics compared to microscopic updates.

\subsection{Coherence Temperature}

\textbf{Equation}
\begin{equation}
    \Theta = \left(\frac{\partial \mathcal{E}}{\partial \mathcal{S}}\right)_{C}
\end{equation}

\textbf{Physical meaning:}
Conjugate parameter to the entropy-like functional at fixed $C$.

\textbf{Symbol dictionary:}
\begin{itemize}
    \item $\Theta$: intensive parameter controlling coherence-entropy tradeoff.
\end{itemize}

\textbf{Derivation status:} Definition by analogy with thermodynamic Legendre structure.

\textbf{Domain of validity:} Regimes where $(\mathcal{E},\mathcal{S})$ admit smooth functional derivatives.

\subsection{Entropic Correction Term}

\textbf{Equation}
\begin{equation}
    \mathcal{F}[C] = \mathcal{F}_0[C] + \lambda \int d^3x\, C(x)\,\ln C(x)
\end{equation}

\textbf{Physical meaning:}
Explicit entropy-like term penalizing fragmentation of coherence.

\textbf{Symbol dictionary:}
\begin{itemize}
    \item $\mathcal{F}_0[C]$: baseline free-energy functional.
    \item $\lambda$: coupling constant for entropic correction.
\end{itemize}

\textbf{Derivation status:} Phenomenological extension.

\textbf{Domain of validity:} $C(x) > 0$ almost everywhere.

\subsection{Coherence Pressure}

\textbf{Equation}
\begin{equation}
    P_C = -\left(\frac{\partial \mathcal{F}}{\partial V}\right)_{C}
\end{equation}

\textbf{Physical meaning:}
Effective pressure associated with changes in volume at fixed coherence field.

\textbf{Symbol dictionary:}
\begin{itemize}
    \item $V$: emergent spatial volume.
\end{itemize}

\textbf{Derivation status:} Definition in analogy with thermodynamic pressure.

\textbf{Domain of validity:} Cosmological or macroscopic domains where $V$ is well-defined.

%=========================================================
\section{Geometry of the Substrate}
%=========================================================

\subsection{Fubini--Study Base Metric}

\textbf{Equation}
\begin{equation}
    ds^2 = 4\big( \langle d\psi | d\psi \rangle - |\langle \psi | d\psi \rangle|^2 \big)
\end{equation}

\textbf{Physical meaning:}
Base geometric metric on projective Hilbert space.

\textbf{Symbol dictionary:}
\begin{itemize}
    \item $|\psi\rangle$: normalized state vector.
\end{itemize}

\textbf{Derivation status:} Standard definition.

\textbf{Domain of validity:} Quantum state space representation.

\subsection{Coherence-Dependent Metric}

\textbf{Equation}
\begin{equation}
    g_{\mu\nu}(x) = g^{(0)}_{\mu\nu}(x) + \alpha\,\partial_\mu C(x)\,\partial_\nu C(x)
\end{equation}

\textbf{Physical meaning:}
Emergent spacetime metric modified by coherence gradients.

\textbf{Symbol dictionary:}
\begin{itemize}
    \item $g^{(0)}_{\mu\nu}$: baseline metric (e.g.\ Minkowski or FLRW).
    \item $\alpha$: coupling parameter.
\end{itemize}

\textbf{Derivation status:} Phenomenological ansatz linking geometry and coherence.

\textbf{Domain of validity:} Weak-gradient regime where corrections are small.

\subsection{Coherence Curvature Tensor}

\textbf{Equation}
\begin{equation}
    R^{\rho}{}_{\sigma\mu\nu}[C] = R^{\rho}{}_{\sigma\mu\nu}\big(g_{\alpha\beta}(C)\big)
\end{equation}

\textbf{Physical meaning:}
Riemann curvature constructed from the coherence-dependent metric.

\textbf{Symbol dictionary:}
\begin{itemize}
    \item $R^{\rho}{}_{\sigma\mu\nu}$: Riemann tensor of $g_{\mu\nu}(C)$.
\end{itemize}

\textbf{Derivation status:} Standard geometric construction applied to $g_{\mu\nu}(C)$.

\textbf{Domain of validity:} Smooth metric configurations.

\subsection{Coherence Laplacian}

\textbf{Equation}
\begin{equation}
    \Delta_C f = \nabla^\mu \nabla_\mu f
\end{equation}

\textbf{Physical meaning:}
Laplacian operator with respect to the coherence-dependent metric.

\textbf{Symbol dictionary:}
\begin{itemize}
    \item $\nabla_\mu$: covariant derivative compatible with $g_{\mu\nu}(C)$.
\end{itemize}

\textbf{Derivation status:} Standard definition.

\textbf{Domain of validity:} Scalar fields $f$ defined on the emergent manifold.

\subsection{Geometry--Coherence Flow Equation}

\textbf{Equation}
\begin{equation}
    \partial_t g_{\mu\nu} = \lambda\,\nabla_\mu \nabla_\nu C
\end{equation}

\textbf{Physical meaning:}
Evolution of the metric driven by second derivatives of coherence.

\textbf{Symbol dictionary:}
\begin{itemize}
    \item $\lambda$: coupling constant controlling back-reaction of coherence on geometry.
\end{itemize}

\textbf{Derivation status:} Phenomenological geometric flow.

\textbf{Domain of validity:} Slow evolution regime of the metric.

%=========================================================
\section{Emergent Gravitation}
%=========================================================

\subsection{Acceleration from Coherence Gradient}

\textbf{Equation}
\begin{equation}
    a^\mu = -\beta\,\nabla^\mu C
\end{equation}

\textbf{Physical meaning:}
Effective acceleration proportional to the gradient of coherence.

\textbf{Symbol dictionary:}
\begin{itemize}
    \item $a^\mu$: four-acceleration of a test body.
    \item $\beta$: coupling constant between coherence gradient and acceleration.
\end{itemize}

\textbf{Derivation status:} Phenomenological law for emergent gravitation.

\textbf{Domain of validity:} Weak-field limit and slowly moving test bodies.

\subsection{Effective Gravitational Field Equation}

\textbf{Equation}
\begin{equation}
    G_{\mu\nu} = \alpha_1\,\nabla_\mu \nabla_\nu C + \alpha_2\,g_{\mu\nu} C
\end{equation}

\textbf{Physical meaning:}
Einstein-like equation where curvature is sourced by coherence.

\textbf{Symbol dictionary:}
\begin{itemize}
    \item $G_{\mu\nu}$: Einstein tensor of $g_{\mu\nu}(C)$.
    \item $\alpha_1, \alpha_2$: coupling constants.
\end{itemize}

\textbf{Derivation status:} Phenomenological field equation.

\textbf{Domain of validity:} Large-scale, coarse-grained regime.

\subsection{Horizon Condition in TCQS}

\textbf{Equation}
\begin{equation}
    C(r_H) = C_{\mathrm{crit}}
\end{equation}

\textbf{Physical meaning:}
Definition of an effective horizon radius by a critical coherence value.

\textbf{Symbol dictionary:}
\begin{itemize}
    \item $r_H$: horizon radius.
    \item $C_{\mathrm{crit}}$: critical coherence threshold.
\end{itemize}

\textbf{Derivation status:} Model definition of horizon in terms of $C$.

\textbf{Domain of validity:} Spherically symmetric, static configurations.

\subsection{Interior Coherence Limit}

\textbf{Equation}
\begin{equation}
    \lim_{r \to 0} C(r) = C_{\max} < \infty
\end{equation}

\textbf{Physical meaning:}
Finite coherence at the center prevents curvature singularities.

\textbf{Symbol dictionary:}
\begin{itemize}
    \item $C_{\max}$: maximal coherence value.
\end{itemize}

\textbf{Derivation status:} Regularity condition imposed on solutions.

\textbf{Domain of validity:} Interior region of compact coherence configurations.

\subsection{Coherence Potential}

\textbf{Equation}
\begin{equation}
    \Phi_C(x) = \gamma\,C(x)
\end{equation}

\textbf{Physical meaning:}
Effective potential whose gradient reproduces coherence-induced acceleration.

\textbf{Symbol dictionary:}
\begin{itemize}
    \item $\Phi_C$: scalar potential.
    \item $\gamma$: proportionality constant (chosen such that $a = -\nabla \Phi_C$).
\end{itemize}

\textbf{Derivation status:} Definition consistent with $a^\mu = -\beta \nabla^\mu C$.

\textbf{Domain of validity:} Newtonian-like limit.

%=========================================================
\section{Quantum--Classical Transition and Measurement}
%=========================================================

\subsection{Local Coherence Locking (Collapse Rule)}

\textbf{Equation}
\begin{equation}
    S \;\longrightarrow\; S_{\mathrm{loc}} = \mathcal{P}_\Omega(S)
\end{equation}

\textbf{Physical meaning:}
Projection of the substrate state onto a locally stabilized subspace.

\textbf{Symbol dictionary:}
\begin{itemize}
    \item $\mathcal{P}_\Omega$: projector corresponding to outcome region $\Omega$.
    \item $S_{\mathrm{loc}}$: locally stabilized substrate state.
\end{itemize}

\textbf{Derivation status:} Effective rule for measurement-like events.

\textbf{Domain of validity:} Regime where local coherence exceeds a threshold in $\Omega$.

\subsection{Decoherence Dynamics (Scalar Form)}

\textbf{Equation}
\begin{equation}
    \partial_t C(x,t) = -\gamma\big(C(x,t) - C_{\mathrm{env}}(x)\big)
\end{equation}

\textbf{Physical meaning:}
Relaxation of local coherence toward an environmental baseline.

\textbf{Symbol dictionary:}
\begin{itemize}
    \item $\gamma$: decoherence rate.
    \item $C_{\mathrm{env}}(x)$: environmental coherence level.
\end{itemize}

\textbf{Derivation status:} Phenomenological.

\textbf{Domain of validity:} Weak system--environment coupling regime.

\subsection{Classicality Stability Condition}

\textbf{Equation}
\begin{equation}
    \partial_t C(x,t) \approx 0, \qquad
    \partial_t^2 C(x,t) \approx 0
\end{equation}

\textbf{Physical meaning:}
Local coherence is stationary on relevant timescales (classical behaviour).

\textbf{Symbol dictionary:}
\begin{itemize}
    \item $C(x,t)$: local coherence field.
\end{itemize}

\textbf{Derivation status:} Definition of quasi-classical regime.

\textbf{Domain of validity:} Macroscopic domains with negligible coherence fluctuations.

\subsection{Observer as Local Coherence Stabilization}

\textbf{Equation}
\begin{equation}
    O(x) = 
    \begin{cases}
        1 & \text{if } C(x) \ge C_{\mathrm{obs}} \text{ and } \partial_t C(x) \approx 0, \\
        0 & \text{otherwise}
    \end{cases}
\end{equation}

\textbf{Physical meaning:}
Indicator function for regions acting as observers via stable high coherence.

\textbf{Symbol dictionary:}
\begin{itemize}
    \item $O(x)$: observer-indicator field.
    \item $C_{\mathrm{obs}}$: coherence threshold for observation.
\end{itemize}

\textbf{Derivation status:} Model definition.

\textbf{Domain of validity:} Coarse-grained description of observer regions.

\subsection{Metastable to Stable Transition}

\textbf{Equation}
\begin{equation}
    \partial_t C(x,t) = -\kappa\big(C(x,t) - C_{\mathrm{stab}}(x)\big)
\end{equation}

\textbf{Physical meaning:}
Relaxation of metastable coherence toward a locally stable configuration.

\textbf{Symbol dictionary:}
\begin{itemize}
    \item $C_{\mathrm{stab}}(x)$: stable coherence profile.
\end{itemize}

\textbf{Derivation status:} Phenomenological.

\textbf{Domain of validity:} Near-stability regime after a measurement-like event.

%=========================================================
\section{Cosmic and Large-Scale Dynamics}
%=========================================================

\subsection{Global Coherence Baseline}

\textbf{Equation}
\begin{equation}
    C_0(t) = \frac{1}{V}\int_V d^3x\, C(x,t)
\end{equation}

\textbf{Physical meaning:}
Spatial average of coherence over a cosmological volume.

\textbf{Symbol dictionary:}
\begin{itemize}
    \item $C_0(t)$: global coherence baseline.
    \item $V$: comoving volume.
\end{itemize}

\textbf{Derivation status:} Definition.

\textbf{Domain of validity:} Large-scale homogeneous approximation.

\subsection{Expansion Rate from Coherence Baseline}

\textbf{Equation}
\begin{equation}
    H^2(t) = \eta\,C_0(t)
\end{equation}

\textbf{Physical meaning:}
Hubble parameter squared proportional to global coherence baseline.

\textbf{Symbol dictionary:}
\begin{itemize}
    \item $H(t)$: Hubble expansion rate.
    \item $\eta$: proportionality constant.
\end{itemize}

\textbf{Derivation status:} Phenomenological cosmological closure relation.

\textbf{Domain of validity:} Large-scale, isotropic cosmology.

\subsection{Coherence Gradient Cosmology}

\textbf{Equation}
\begin{equation}
    \delta C(x,t) = C(x,t) - C_0(t)
\end{equation}

\textbf{Physical meaning:}
Deviation of local coherence from the global baseline.

\textbf{Symbol dictionary:}
\begin{itemize}
    \item $\delta C$: coherence perturbation field.
\end{itemize}

\textbf{Derivation status:} Definition used for perturbative analysis.

\textbf{Domain of validity:} Linear or weakly nonlinear perturbation regime.

\subsection{Dark-Energy Analogue from Coherence Floor}

\textbf{Equation}
\begin{equation}
    \rho_\Lambda = \xi\,C_{\mathrm{floor}}
\end{equation}

\textbf{Physical meaning:}
Effective dark-energy density proportional to a minimal coherence floor.

\textbf{Symbol dictionary:}
\begin{itemize}
    \item $\rho_\Lambda$: effective dark-energy density.
    \item $C_{\mathrm{floor}}$: minimal nonzero coherence baseline.
    \item $\xi$: proportionality constant.
\end{itemize}

\textbf{Derivation status:} Phenomenological identification.

\textbf{Domain of validity:} Late-time cosmology.

\subsection{Global Synchronization Condition}

\textbf{Equation}
\begin{equation}
    \nabla_i \phi(x,t) \approx 0
\end{equation}

\textbf{Physical meaning:}
Spatial uniformity of a global phase field in synchronized regimes.

\textbf{Symbol dictionary:}
\begin{itemize}
    \item $\phi(x,t)$: coarse-grained phase associated with the substrate.
\end{itemize}

\textbf{Derivation status:} Condition for global coherence.

\textbf{Domain of validity:} Near-homogeneous cosmological phases.

%=========================================================
\section{Special Constructs}
%=========================================================

\subsection{Antimatter as $\pi$-Shifted Coherence Phase}

\textbf{Equation}
\begin{equation}
    \theta_{\bar{p}} = \theta_p + \pi \quad (\mathrm{mod}\,2\pi)
\end{equation}

\textbf{Physical meaning:}
Antimatter sector represented as a $\pi$ phase shift in a coherence phase field.

\textbf{Symbol dictionary:}
\begin{itemize}
    \item $\theta_p$: coherence phase associated with a matter configuration.
    \item $\theta_{\bar{p}}$: coherence phase for the corresponding antimatter configuration.
\end{itemize}

\textbf{Derivation status:} Model identification.

\textbf{Domain of validity:} Regime where phase description of matter/antimatter is valid.

\subsection{Resonant Coherence Field Equation}

\textbf{Equation}
\begin{equation}
    \Box C(x,t) + m_C^2\,C(x,t) = J_C(x,t)
\end{equation}

\textbf{Physical meaning:}
Wave-like propagation of coherence with source term $J_C$.

\textbf{Symbol dictionary:}
\begin{itemize}
    \item $\Box = \nabla^\mu \nabla_\mu$: d'Alembert operator.
    \item $m_C$: effective mass-scale of coherence excitations.
    \item $J_C(x,t)$: source term (coupling to structures).
\end{itemize}

\textbf{Derivation status:} Phenomenological field equation.

\textbf{Domain of validity:} Linear regime of coherence perturbations.

\subsection{Cross-Scale Coherence Resonance}

\textbf{Equation}
\begin{equation}
    \mathcal{R} = \int d^3x\,d^3y\,K(x,y)\,C_1(x)\,C_2(y)
\end{equation}

\textbf{Physical meaning:}
Resonance functional coupling two coherence fields $C_1$ and $C_2$ via kernel $K$.

\textbf{Symbol dictionary:}
\begin{itemize}
    \item $K(x,y)$: coupling kernel between locations $x$ and $y$.
    \item $C_1, C_2$: coherence fields associated with two systems (e.g.\ two organisms).
\end{itemize}

\textbf{Derivation status:} Phenomenological definition of resonance measure.

\textbf{Domain of validity:} Weak-coupling, linear-response regime.

\subsection{Information-Flow Equation}

\textbf{Equation}
\begin{equation}
    \partial_t I(x,t) + \bm{\nabla}\cdot \bm{J}_I(x,t) = \sigma_I(x,t)
\end{equation}

\textbf{Physical meaning:}
Continuity equation for an information-density field $I$ with source/sink term $\sigma_I$.

\textbf{Symbol dictionary:}
\begin{itemize}
    \item $I(x,t)$: information density.
    \item $\bm{J}_I$: information current density.
    \item $\sigma_I$: information production/absorption rate.
\end{itemize}

\textbf{Derivation status:} Phenomenological.

\textbf{Domain of validity:} Coarse-grained informational description.

\subsection{Boundary Self-Reference Relation}

\textbf{Equation}
\begin{equation}
    S|_{\partial \Omega} = \mathcal{B}\big[S|_{\Omega}\big]
\end{equation}

\textbf{Physical meaning:}
Boundary state determined functionally by the interior substrate configuration.

\textbf{Symbol dictionary:}
\begin{itemize}
    \item $\Omega$: region of the substrate.
    \item $\partial \Omega$: boundary of $\Omega$.
    \item $\mathcal{B}$: boundary functional mapping interior to boundary.
\end{itemize}

\textbf{Derivation status:} Model constraint encoding self-referential closure.

\textbf{Domain of validity:} Regions where boundary conditions are fully determined by interior dynamics.

%=========================================================

\end{document}

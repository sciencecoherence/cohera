\setcounter{section}{2}
\section*{Mathematical Framework}
\label{sec:3}

% ============================================
% Section 3 — Mathematical Architecture of the TCQS
% Final integrated version with standalone 3.2,
% added 3.4 Coherence Density Function,
% enhanced 3.5 Time-Crystalline Update Rule (incl. discrete invariance),
% new 3.6 Information Flow Tensor, and expanded 3.7 Summary.
% ============================================

% =====================================================
%                         Overview
=====================================================
\paragraph{The Fundamental Floquet Update Rule}

At the most elementary level, the vacuum is not a static background, but an actively updating information medium. Each ``tick'' of the underlying time-crystal corresponds to a fundamental cycle in which the substrate reads, rewrites, and re-stabilizes its own state. To make this explicit, we describe the substrate by a sequence of discrete states \(\{S_n\}\), one for each fundamental cycle \(n\), and we assume that the entire history of the universe is nothing but the repetition of a single, intrinsic update rule applied recursively.

\medskip

Let $S_n$ denote the state of the substrate at the $n$-th fundamental cycle. 
Its evolution is governed by an intrinsic, self-referential update rule
\begin{equation}
    S_{n+1} = \mathcal{U}(S_n),
\end{equation}
where $\mathcal{U}$ is the Substrate's Floquet-like update operator generating the discrete time-as-memory progression, it is the time-crystalline update functional encoding, the substrate’s  periodic self-generation. Coherence thus defines being; deviations from this 
intrinsic rhythm define apparent change.


\subsection*{Overview}
Section~3 establishes the \textit{formal backbone} of the Time-Crystal Quantum Substrate (TCQS).
Where Section~2 laid down the conceptual premises, this section constructs the \textit{mathematical framework}: a self-referential Hamiltonian lattice evolving through a time-crystalline law, embedded in a coherence geometry with
thermodynamic and gauge-like structure, a distributed network of coherence nodes whose internal dynamics are governed by context-dependent operators acting over a time-crystalline manifold. Each node embodies a localized quantum informational state, yet inseparable from the global state of the substrate. This interdependence gives rise to a hierarchy of mathematical structures:
\vspace{1em}
\begin{table}[h!]
\centering
\renewcommand{\arraystretch}{1.15}
\setlength{\tabcolsep}{6pt}
\begin{tabular}{p{0.23\linewidth} p{0.72\linewidth}}
\hline
\textbf{Structural Level} & \textbf{Conceptual Role} \\
\hline
Operational & Local and global Hamiltonians \(H_i,\,H_{\text{tot}}\)  defining contextual energetic landscapes of the substrate. \\
Topological & A tensor-product state space \(\mathcal{H}_{\text{tot}}=\bigotimes_i \mathcal{H}_i\) encoding the manifold of possible coherence configurations. \\
Temporal & A Floquet-type temporal update rule representing the discrete rhythmic reconstruction of reality—the ``informational heartbeat'' of the substrate. \\
Geometric & Informational geometry emerging from overlaps between evolving states, giving rise to curvature in coherence space. \\
Dynamical & Thermodynamic and gauge analogues quantifying how coherence flows, restores, and synchronizes across the network. \\
\hline
\end{tabular}
\caption{Hierarchy of mathematical and conceptual structures defining the architecture of the Time-Crystal Quantum Substrate (TCQS).}
\label{tab:tcqs_overview_hierarchy}
\end{table}
\vspace{1em}

Together, these constructions form the \textit{Mathematical Architecture of the TCQS}---a unified language that binds Hamiltonian dynamics, quantum geometry, and informational thermodynamics into a single coherent system.  
This section therefore provides the foundation from which---the \textit{Dynamic Laws of Coherence} (Section~4) and the \textit{Geometric and Gravitational Consequences} (Sections~5--6)---naturally unfold.


% =====================================================
%           3.1 State-Space Structure
% =====================================================

\subsection{Substrate Structure and State-Space Topology}

Consider a lattice of qudits \( q_i \) with local Hamiltonians \( H_i \) and time-periodic couplings:
\begin{equation}
H(t) \;=\; \sum_i H_i \;+\; \sum_{\langle ij\rangle} J_{ij}(t)\,H_{ij}, 
\qquad J_{ij}(t{+}T)=J_{ij}(t).
\tag{3.1}
\end{equation}
The local Hamiltonian of each site is a contextual projection of the global total \(H_{\text{tot}}\), which encodes the local coherence densities within the global informational manifold. The TCQS substrate therefore forms a periodically driven Hamiltonian network, whose time-crystalline modulation provides the rhythmic basis of the system’s evolution.
\vspace{0.5em}

Let the total Hilbert space be the tensor product
\begin{equation}
\mathcal{H}_{\text{tot}} \;=\; \bigotimes_{i=1}^{N} \mathcal{H}_i,
\tag{3.2a}
\end{equation}
with local states \(|\Psi_i\rangle \in \mathcal{H}_i\) forming the coherence ensemble
\begin{equation}
\mathcal{S} \;=\; \{\,|\Psi_i\rangle \mid i{=}1,\dots,N \,\},
\tag{3.2b}
\end{equation}
and total state
\begin{equation}
|\Psi_{\text{tot}}\rangle \;=\; \bigotimes_{i=1}^{N} |\Psi_i\rangle.
\tag{3.2c}
\end{equation}
The global Hamiltonian \(H_{\text{tot}}\) acts on \(\mathcal{H}_{\text{tot}}\), while local projections are
\begin{equation}
H_i \;=\; \langle \Psi_i | H_{\text{tot}} | \Psi_i \rangle .
\tag{3.2d}
\end{equation}
This defines the \textit{state-space topology}: a coherent tensor lattice whose nodes interact via informational couplings \(J_{ij}(t)\) and global state-dependent feedback.

\subsection{Substrate Structure and State-Space Topology}

The Time–Crystalline Quantum Substrate (TCQS) is modeled as a network of
interacting qudits \(q_i\), each representing a finite-dimensional quantum
domain carrying local Hamiltonians \(H_i\).
The total Hamiltonian of the substrate is defined as
\begin{equation}
H_{\text{tot}}
  = \sum_i H_i
  + \sum_{i\neq j} J_{ij}(t)\,H_{ij},
\tag{3.1}
\end{equation}

where \(J_{ij}(t)\) are time-periodic coupling coefficients
encoding local interaction strength and temporal modulation.
The periodicity of \(J_{ij}(t)\) imposes a fundamental discrete time symmetry
that propagates through all dynamical layers of the substrate.

Local Hamiltonians are defined by contextual projections of the total operator,
\begin{equation}
H_i = \langle \Psi_i | H_{\mathrm{tot}} | \Psi_i \rangle ,
\tag{3.1a}
\end{equation}
so that local deviations $\delta H_i = H_i - \langle H \rangle$ encode curvature in the coherence field.

The total state factorizes over local subspaces,
\begin{equation}
|\Psi_{\mathrm{tot}}\rangle = \bigotimes_i |\Psi_i\rangle ,
\tag{3.1b}
\end{equation}
and the global Hilbert manifold is the tensor product
$\mathcal{H}=\bigotimes_i \mathcal{H}_i$.


Each local Hamiltonian \(H_i\) acts on a finite Hilbert space
\(\mathcal{H}_i \simeq \mathbb{C}^d\),
and the total state-space forms the tensor-product manifold
\begin{equation}
\mathcal{H}
  = \bigotimes_{i=1}^{N} \mathcal{H}_i ,
\tag{3.1a}
\end{equation}
whose topology determines how local coherence patches combine to form the
global quantum state \(|\Psi\rangle \in \mathcal{H}\).
Connectivity among subsystems is represented by a directed interaction graph
\(\mathcal{G} = (\mathcal{V},\mathcal{E})\)
whose vertices \(\mathcal{V}\) correspond to local domains and whose edges
\(\mathcal{E}\) represent active couplings \(J_{ij}(t)\).
The adjacency and periodicity of this graph determine the
\textit{coherence topology} of the substrate.

The global state of the system is written as
\begin{equation}
|\Psi\rangle
  = \bigotimes_i |\psi_i\rangle ,
\qquad
|\psi_i\rangle \in \mathcal{H}_i ,
\tag{3.1b}
\end{equation}
subject to normalization
\(\langle \Psi | \Psi \rangle = 1\).
Local deviations

% ---- Projection / coarse-graining from the total Hamiltonian
\begin{equation}
H(t)\;=\;\mathcal{P}_t\!\left[\,H_{\text{tot}}\,\right],
\qquad
\mathcal{P}_{t+T}=\mathcal{P}_t ,
\tag{3.1c}
\end{equation}

\begin{align}
H_i(t)      &= \mathrm{Tr}_{\bar i}\!\left[\rho_{\bar i}(t)\,H_{\text{tot}}\right], \\
H_{ij}(t)   &= \mathrm{Tr}_{\overline{ij}}\!\left[\rho_{\overline{ij}}(t)\,H_{\text{tot}}\right],
\end{align}


% =====================================================
%           3.3 Self-Referential Evolution
% =====================================================

\subsection{Self-Referential Evolution}

Temporal evolution is governed by a \textit{state-dependent Floquet operator}:
\begin{equation}
U_F(|\Psi\rangle)
  = \exp\!\left[-\,i\,H(|\Psi\rangle)\,\frac{T}{\hbar}\right],
\qquad
|\Psi\rangle \;\mapsto\; U_F(|\Psi\rangle)\,|\Psi\rangle .
\tag{3.3}
\end{equation}
Here, \(H(|\Psi\rangle)\) denotes the contextual Hamiltonian determined by the instantaneous global state, so that the operator that governs evolution depends reflexively on the very state it updates.
This formalizes the TCQS feedback law: the substrate continuously recalculates its own generator at each cycle, ensuring global coherence through self-reference.

\vspace{0.5em}
For compactness, and to emphasize its equivalence with later dynamical formulations, the same relation can be expressed in shorthand form as
\begin{equation}
U_F(|\Psi\rangle)
  = e^{-\,i\,H(|\Psi\rangle)T},
\tag{3.3a}
\end{equation}
where units are understood such that \(\hbar = 1\).
This simplified notation is frequently used in subsequent sections to highlight the feedback symmetry rather than dimensional normalization.

The Hamiltonian depends on the instantaneous global state, providing a \emph{self-referential} feedback law that recalculates its own generator each cycle and maintains global coherence so that feedback from the instantaneous state \(|\Psi\rangle\) shapes the Hamiltonian that generates the next update.  
This defines a \textit{self-referential dynamic}, in which the substrate continuously recalculates its own evolution law, ensuring self-correction and global coherence across cycles of period \(T\).

\noindent
The feedback operator \(U_F(|\Psi\rangle)\) introduced in Eq.~\eqref{3.3}
naturally generates a discrete rhythm of reconstruction: 
each application of \(U_F\) advances the global state by one temporal quantum~\(T\),
producing the stroboscopic evolution characteristic of a time crystal.
This periodic self-referential process constitutes the foundation of the
\textit{Time-Crystalline Update Rule} developed in Section~\ref{sec:TCQS_Floquet},
where the same operator formalism gives rise to discrete temporal symmetry
and Noether-like conservation of informational coherence.


% =====================================================
%           3.4 Coherence Density Function
% =====================================================

\subsection{Coherence Density Function}

We introduce a local \emph{coherence density} field \(\mathcal{C}(x,t)\) associated to a coarse-grained region around \(x\).
Two equivalent, practically useful definitions are:
\begin{align}
\mathcal{C}(x,t) 
&:= 1 \;-\; \frac{S\!\bigl(\rho_x(t)\bigr)}{\log d_x},
\quad 
S(\rho) = -\,\mathrm{Tr}(\rho \ln \rho), 
\tag{3.4a}\\
\mathcal{C}(x,t) 
&:= \mathrm{Tr}\!\bigl[\rho_x^2(t)\bigr]
\;\;\;\text{(purity; \(=1\) iff locally pure)}.
\tag{3.4b}
\end{align}
Here \(\rho_x(t)\) is the local reduced density operator for a block around \(x\) of on-site dimension \(d_x\).
Higher \(\mathcal{C}\) indicates higher local coherence (lower mixedness).

The informational line element on projective Hilbert space gives a geometry for coherence change:
\begin{equation}
ds^2 \;=\; 1 - \bigl|\langle \Psi(t) \mid \Psi(t{+}dt) \rangle \bigr|^2 ,
\tag{3.4c}
\end{equation}
and the (gauge-covariant) quantum geometric tensor
\begin{equation}
C_{\mu\nu} \;=\; 
\bigl\langle \nabla_\mu \Psi \mid \nabla_\nu \Psi \bigr\rangle 
- \bigl\langle \nabla_\mu \Psi \mid \Psi \bigr\rangle 
  \bigl\langle \Psi \mid \nabla_\nu \Psi \bigr\rangle ,
\qquad 
\nabla_\mu := \partial_\mu - i A_\mu ,
\tag{3.4d}
\end{equation}
whose real part yields the Fubini–Study metric of coherence space and whose imaginary part yields Berry curvature.
Coherence gradients couple to this geometry via the constitutive relation
\begin{equation}
\partial_\mu \mathcal{C} \;\propto\; C_{\mu\nu}\,u^\nu ,
\tag{3.4e}
\end{equation}
where \(u^\nu\) is the local flow four-velocity defined by the stroboscopic evolution (Section~\ref{sec:TCQS_Floquet} below).


% =====================================================
%           3.5 Time-Crystalline Update Rule 
% =====================================================

\subsection{Time-Crystalline Update Rule}\label{sec:TCQS_Floquet}

The substrate evolves in discrete, periodic steps governed by the Floquet rhythm:
\begin{equation}
|\Psi(t{+}T)\rangle
  \;=\;
U_F\!\bigl(|\Psi(t)\rangle\bigr)\,|\Psi(t)\rangle ,
\qquad
U_F
  = \exp\!\left[-\,\frac{i}{\hbar}\,H_{\mathrm{eff}}(|\Psi\rangle)\,T\right].
\tag{3.5}
\end{equation}

This compact form expresses the quantized temporal symmetry defining the
\textit{time-crystalline structure} of the TCQS:
each cycle of duration \(T\) corresponds to a feedback-dependent reconstruction of the global state, ensuring that coherence is restored by the contextual Hamiltonian \(H(|\Psi\rangle)\).
The periodic reconstruction of the substrate provides the
\textit{informational heartbeat} underlying emergent temporality.

\vspace{0.5em}
When the generator varies within each cycle, the effective propagator generalizes to the time-ordered exponential form
\begin{equation}
U_F
  = \mathcal{T}
    \exp\!\left[
      -\,\frac{i}{\hbar}
      \!\int_{t}^{t+T}
      H\!\bigl(|\Psi(\tau)\rangle,\tau\bigr)\,d\tau
    \right],
\tag{3.5a}
\end{equation}
with \(\mathcal{T}\) denoting the time-ordering operator
and \(H\) possibly explicitly \(T\)-periodic through couplings \(J_{ij}(t)\).

\paragraph{Discrete temporal symmetry and Noether-like invariance.}
If the stroboscopic action (or effective generator) is invariant under the discrete shift \(t\!\mapsto\!t{+}T\),
\begin{equation}
\mathcal{S}_d[\Psi(t)]
  = \mathcal{S}_d[\Psi(t{+}T)],
\tag{3.5b}
\end{equation}
then a \textit{Floquet charge} is conserved cycle-to-cycle.
Equivalently, the quasi-energy (expectation of the effective generator \(H_{\mathrm{eff}}\))
is conserved at stroboscopic times:
\begin{equation}
\mathcal{E}_F(t)
  := \langle \Psi(t) | H_{\mathrm{eff}} | \Psi(t) \rangle,
\qquad
\mathcal{E}_F(t{+}T) = \mathcal{E}_F(t).
\tag{3.5c}
\end{equation}

Thus, discrete time-translation symmetry implies periodic conservation laws:
integrals of motion that recur every cycle (up to micromotion), anchoring global synchronization of coherence throughout the substrate.

\subsubsection*{Temporal Synchronization}
Global temporal order arises when local oscillatory domains
synchronize through the discrete Floquet rhythm of Eq.~\eqref{3.5}.
Phase gradients $\nabla_t \phi_i$ between domains correspond to informational frequency shifts, and their relaxation toward uniformity constitutes the time-crystalline synchronization of the substrate.


% =====================================================
%           3.6 Emergent Information Geometry
% =====================================================
\subsection{Emergent Information Geometry}

An informational line element on projective Hilbert space is defined as
\begin{equation}
ds^2 = 1 - \bigl|\langle \Psi(t) | \Psi(t{+}dt) \rangle \bigr|^2,
\tag{3.6}
\end{equation}
measuring infinitesimal changes in the global state's coherence, whose differential form can be expressed through the quantum geometric tensor,
thereby defining the informational analogue of spacetime curvature.

we introduce the \textit{coherence tensor}
\begin{equation}
C_{\mu\nu} = \langle \partial_\mu \Psi | \partial_\nu \Psi \rangle ,
\tag{3.6a}
\end{equation}
which captures how the wavefunction varies across local informational coordinates.
To ensure gauge invariance under local phase transformations
\( |\Psi\rangle \!\to\! e^{i\phi(x)}|\Psi\rangle \),
we refine this expression using the covariant derivative
\(\nabla_\mu = \partial_\mu - iA_\mu\):
\begin{equation}
C_{\mu\nu}
 = \langle \nabla_\mu \Psi | \nabla_\nu \Psi \rangle
 - \langle \nabla_\mu \Psi | \Psi \rangle
   \langle \Psi | \nabla_\nu \Psi \rangle .
\tag{3.6b}
\end{equation}
The real part of \(C_{\mu\nu}\) defines the Fubini--Study metric,
\(g^{\mathrm{FS}}_{\mu\nu} = \mathrm{Re}[C_{\mu\nu}]\),
which constitutes the intrinsic Riemannian structure of the projective Hilbert manifold
\(\mathcal{P}(\mathcal{H})\).
This metric quantifies the informational distance between neighboring quantum states and determines the local curvature of coherence space.
The imaginary part,
\(\mathrm{Im}[C_{\mu\nu}]\),
corresponds to the Berry curvature associated with parallel transport of quantum phase, encoding the gauge structure that arises from cyclic evolution of coherent domains.

In the emergent classical limit (see Section~6),
the Fubini--Study metric reduces to an effective spacetime metric \(g_{\mu\nu}\) governing macroscopic geometry, while the Berry curvature contributes to the effective gauge potentials that mediate phase synchronization.
Covariant variations \(\nabla_\tau C_{\mu\nu}\) of this tensor will later be shown to generate the emergent gravitational field equations of the TCQS,
demonstrating how coherence curvature manifests as spacetime curvature
when informational geometry transitions to the classical regime.


\subsubsection*{Gauge of Coherence}
The phase degree of freedom of each local coherence domain defines a
$\mathrm{U}(1)$ informational gauge.
Parallel transport of local phases induces a connection
$A_\mu = i \langle \Psi | \partial_\mu \Psi \rangle$,
whose curvature $F_{\mu\nu} = \partial_\mu A_\nu - \partial_\nu A_\mu$
is the Berry curvature $\mathrm{Im}[C_{\mu\nu}]$introduced in Eq.~\eqref{3.6b}.
This gauge structure ensures that coherence transport remains covariant under local phase transformations.

% =====================================================
%   Gauge of Coherence and Temporal Synchronization
% =====================================================

\subsection{Gauge of Coherence and Temporal Synchronization}

Temporal and spatial coherence are unified under a gauge-like formalism.
We define a coherence potential \(A_\mu\) such that
\begin{equation}
\nabla_\mu \Psi = (\partial_\mu - iA_\mu)\Psi,
\tag{3.7}
\end{equation}
ensuring that phase synchronization across regions of the substrate
is locally invariant under transformations of the coherence phase,
\(\Psi \rightarrow e^{i\phi(x)}\Psi\).

The corresponding field strength
\begin{equation}
F_{\mu\nu} = \partial_\mu A_\nu - \partial_\nu A_\mu,
\tag{3.7a}
\end{equation}
quantifies desynchronization between domains---interpreted as the curvature
of coherence space. In the macroscopic limit, \(F_{\mu\nu}\) manifests as
gravitational curvature or informational flow potential.

This gauge formalism directly links to the imaginary component of the quantum geometric tensor introduced in Eq.~\eqref{3.6b},
where \(\mathrm{Im}[C_{\mu\nu}] = F_{\mu\nu}\).
The gauge potential \(A_\mu = i\langle \Psi | \partial_\mu \Psi \rangle\)
thus acts as the informational analogue of a connection one-form,
ensuring that parallel transport of coherence is covariant under local
phase transformations.

\paragraph{Temporal Synchronization.}
Temporal coherence across the TCQS arises when the phase evolution of each
local domain locks to the global Floquet rhythm defined in Eq.~\eqref{3.5}.
Phase gradients \(\nabla_t \phi_i\) between domains correspond to
informational frequency shifts, and their relaxation toward uniformity
constitutes the time-crystalline synchronization of the substrate.
The covariant condition
\begin{equation}
\nabla^{(\Omega)}_{\tau}\Psi \;=\;0 .
\tag{3.7b}
\end{equation}
\begin{equation}
\nabla_t \Psi = 0,
\tag{3.7b}
\end{equation}
expresses global temporal invariance: coherence phases evolve in synchrony
with the discrete time-crystal update, maintaining global phase order across
the lattice of qudits.

Together, the spatial gauge connection \(A_\mu\) and temporal synchronization
condition \(\nabla_t \Psi = 0\) form a unified coherence law:
the substrate preserves informational phase order by coupling the curvature of
coherence space to the periodic reconstruction of temporal structure.
This equivalence between spatial gauge invariance and temporal periodicity
establishes the foundation for the emergent gravito-informational field
dynamics developed in Section~6.


\subsection{Gauge of Coherence and Temporal Synchronization}

Temporal and spatial coherence are unified under a gauge-like formalism.
We define a coherence potential \(A_\mu\) such that
\begin{equation}
\nabla_\mu \Psi = (\partial_\mu - iA_\mu)\Psi,
\tag{3.7}
\end{equation}
ensuring that phase synchronization across regions of the substrate
is locally invariant under transformations of the coherence phase,
\(\Psi \rightarrow e^{i\phi(x)}\Psi\).

The corresponding field strength
\begin{equation}
F_{\mu\nu} = \partial_\mu A_\nu - \partial_\nu A_\mu,
\tag{3.7a}
\end{equation}
quantifies desynchronization between domains—interpreted as the
curvature of coherence space.
In the macroscopic limit, \(F_{\mu\nu}\) manifests as the
gravitational curvature or informational flow potential.

This gauge formalism directly links to the imaginary component of the
quantum geometric tensor introduced in Eq.~\eqref{3.6b},
where \(\mathrm{Im}[C_{\mu\nu}] = F_{\mu\nu}\).
Temporal synchronization then follows from the covariant condition
\(\nabla_t \Psi = 0\),
expressing that global coherence is preserved when local oscillatory
phases evolve in synchrony with the time-crystalline rhythm of the
substrate (see Section~3.5).

Hence, gauge invariance of the coherence phase and discrete temporal
symmetry are two aspects of a single unifying law:
the informational field preserves its global phase order by coupling
spatial coherence curvature to periodic temporal reconstruction.


% =====================================================
%           3.7 Informational Free-Energy Gradient
% =====================================================
\subsection{Informational Free-Energy Gradient}
Local evolution of coherence density obeys an informational thermodynamic law:
The TCQS substrate tends toward minimal informational free energy.
Defining the global informational free-energy functional
\begin{equation}
\mathcal{F}[\rho]
  = \mathrm{Tr}[\rho H] - T_{\mathrm{info}}\,S(\rho),
\qquad
S(\rho) = -\,\mathrm{Tr}(\rho \ln \rho),
\tag{3.7}
\end{equation}
its gradient drives the local relaxation dynamics of the density operator:
\begin{equation}
\frac{d\rho}{dt}
  = -\,\eta\,\nabla_{\rho}\mathcal{F},
\tag{3.7a}
\end{equation}
where $\eta$ is an informational mobility coefficient and
$T_{\mathrm{info}}$ acts as an effective substrate temperature representing
decoherence pressure.
At equilibrium, $\nabla_{\rho}\mathcal{F}=0$,
implying a balance between internal energy and entropic dispersion.

\paragraph{Local thermodynamic form.}
For a local coherence domain $i$ with reduced state $\rho_i$,
the same principle yields the microscopic free-energy gradient law
\begin{equation}
\nabla_i F
  = \frac{\partial}{\partial \rho_i}
    \!\left[
      \langle H_i\rangle - T_{\mathrm{info}} S_i
    \right],
\qquad
S_i = -\,k_{\mathrm{B}}\,
      \mathrm{Tr}(\rho_i \ln \rho_i),
\tag{3.7b}
\end{equation}
where \(\rho_i\) is the local density operator, \(S_i = -k_B \operatorname{Tr}(\rho_i \ln \rho_i)\) the informational entropy, 
and \(T\) an effective substrate temperature representing decoherence pressure.  
This \textit{free-energy gradient} drives the re-equilibration of local coherence imbalances, serving as the microscopic origin of curvature and acceleration effects.
which drives the re-equilibration of local coherence imbalances.
This \textit{free-energy gradient} represents the microscopic origin of
curvature and acceleration effects in the substrate,
linking informational thermodynamics to emergent geometric dynamics.

\paragraph{Geometric interpretation.}
The gradient $\nabla_{\rho}\mathcal{F}$ defines the vector field of
coherence flow in information space.
Its geometric projection along the metric $C_{\mu\nu}$
yields the informational geodesics of minimal free-energy dissipation.
These geodesics determine the natural relaxation paths that appear in the
Dynamic Laws of Coherence (Section~4),
providing the bridge between informational thermodynamics and the
time-crystalline dynamics of the TCQS.


% =====================================================
%           3.8 Information Flow Tensor
% =====================================================
\subsection{Information Flow Tensor}

Define the \emph{coherence current} (gauge-covariant probability-like flow)
\begin{equation}
J^\mu_{\text{coh}} \;:=\; \frac{1}{\hbar}\,\mathrm{Im}\!\left\langle \Psi \,\middle|\, \nabla^\mu \Psi \right\rangle ,
\qquad 
\nabla_\mu = \partial_\mu - i A_\mu ,
\tag{3.6a}
\end{equation}
and the \emph{information flow tensor} as the rank-2 object capturing transport and stress of coherence:
\begin{equation}
I_{\mu\nu} \;:=\; 
\mathrm{Re}\!\left\langle \nabla_\mu \Psi \,\middle|\, \nabla_\nu \Psi \right\rangle 
\;+\; J_\mu^{\text{coh}} J_\nu^{\text{coh}}
\;-\; g_{\mu\nu}\,\mathcal{L}_{\text{coh}} ,
\qquad 
\mathcal{L}_{\text{coh}} := \mathrm{Re}\!\left\langle \nabla_\alpha \Psi \,\middle|\, \nabla^\alpha \Psi \right\rangle - V(\mathcal{C}).
\tag{3.6b}
\end{equation}
Here \(g_{\mu\nu}\) is the coherence-space metric (real part of \(C_{\mu\nu}\)), and \(V(\mathcal{C})\) is an effective potential penalizing decoherence.
The local \emph{continuity law of coherence} then reads
\begin{equation}
\partial_t \mathcal{C} \;+\; \nabla_\mu J^\mu_{\text{coh}} \;=\; \Sigma_{\text{rec}} - \Sigma_{\text{dec}},
\tag{3.6c}
\end{equation}
with \(\Sigma_{\text{rec}},\Sigma_{\text{dec}}\) the (model-dependent) recoherence/decoherence source terms.
Stroboscopically (every \(T\)), the discrete invariance of \eqref{3.5b} implies a cycle-averaged conservation:
\begin{equation}
\bigl\langle \partial_t \mathcal{C} \bigr\rangle_T \;+\; \bigl\langle \nabla_\mu J^\mu_{\text{coh}} \bigr\rangle_T \;=\; 0 ,
\tag{3.6d}
\end{equation}
linking the Noether-like temporal symmetry to global coherence balance.


% =====================================================
%           3.9 Summary and Implications
% =====================================================
\subsection{Summary and Implications}

Section~3 formalizes the TCQS as a \textit{time-periodic, self-referential quantum information network} governed by feedback-coupled Hamiltonians.  
The emergent geometry and thermodynamics of this structure are summarized in Table~\ref{tab:summary3}.

\paragraph{Architecture.}
The TCQS is a \(T\)-periodically driven, self-referential quantum information lattice:
(a) a tensor-product state space (Sec.~3.2),
(b) contextual local/global Hamiltonians (Sec.~3.1--3.3),
(c) a coherence density field coupled to a quantum geometric tensor (Sec.~3.4),
(d) a Floquet update with discrete invariance and periodic integrals of motion (Sec.~3.5),
(e) and an information flow tensor encoding transport and stress of coherence (Sec.~3.6).

\paragraph{Bridge to dynamics (Section 4).}
Equations~\eqref{3.6a}--\eqref{3.6d} define currents and tensorial objects from which the \emph{Dynamic Laws of Coherence} follow: local evolution,
global synchronization, and continuity constraints. The cycle-averaged conservation laws derived from discrete time symmetry serve as the Noether-like anchors of the forthcoming equations of motion.

\begin{table}[h!]
\centering
\begin{tabular}{lll}
\hline
\textbf{Level} & \textbf{Mathematical Object} & \textbf{Physical Meaning} \\
\hline
State space & \(\mathcal{H}_{\text{tot}}=\bigotimes_i \mathcal{H}_i\) & Coherent tensor manifold of the substrate \\
Local/global & \(H_i,\, H_{\text{tot}},\, H(|\Psi\rangle)\) & Contextual operators \& self-referential generator \\
Temporal & \(U_F=\mathcal{T}e^{- \frac{i}{\hbar}\int_t^{t+T}H(\cdot)\,d\tau}\) & Time-crystalline stroboscopic update \\
Geometry & \(C_{\mu\nu}\) (QGT) & Metric/curvature of coherence space \\
Density & \(\mathcal{C}(x,t)\) & Local coherence (purity/entropy-normalized) \\
Transport & \(J^\mu_{\text{coh}},\, I_{\mu\nu}\) & Coherence current and information flow tensor \\
Invariance & \(\mathcal{E}_F=\langle H_{\text{eff}}\rangle\) & Cycle-to-cycle conserved (Floquet) quantity \\
\hline
\end{tabular}
\caption{Core mathematical objects of the TCQS and their physical interpretations.}
\label{tab:tcqs_section3_summary}
\end{table}

Together, these relations constitute the mathematical architecture of the TCQS, unifying Hamiltonian dynamics, information geometry, and thermodynamic feedback within a single self-organizing formalism.
% ---------- end Section 3 ----------
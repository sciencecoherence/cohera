% ============================================
% Section 1 — Introduction
% ============================================

\setcounter{section}{0}
\section{Introduction}
\label{sec:1}

\begin{quote}
\textit{“Reality is the measurement of its own coherence; time is the rhythm of that self-evaluation.”}
\end{quote}

\noindent A Time--Crystalline Quantum Substrate is proposed as a unifying framework for physics, reconciling quantum information dynamics, thermodynamics, and spacetime geometry within a single holographic self-referential law of coherence and postulates that reality originates from a coherent informational substrate rather than from pre-given spacetime or energy. 
Existence itself is a manifestation of coherence: a self-sustaining oscillatory field whose temporal periodicity constitutes the rhythm of reality. 
The central premise is that existence is not composed of matter or energy \emph{per se}, but of coherent information--an evolving pattern of phase relations sustained by a time--crystalline process of self-correction, an intrinsic feedback mechanism by which each time-crystalline update\footnote{The time–crystalline update does not occur \emph{within} any temporal metric.  It is a pre-geometric, self-referential recurrence of the substrate itself — the fundamental operation through which coherent structure is continuously regenerated. 

What we ordinarily describe as “time” emerges only as the ordered appearance of successive  stable configurations produced by this underlying update cycle.} realigns the evolving coherence pattern within its minimally allowed structure, ensuring stability and preventing drift under interaction.
At the most primitive level, the universe is not composed of particles or waves but of relations of informational coherence—the persistence of self-similar patterns across discrete temporal cycles. Matter and fields emerge as stable, self-sustaining modes of these coherence patterns.

This new model interprets the cosmos as a distributed intelligence, a coherent informational system perpetually minimizing its own decoherence.
Every interaction constitutes a feedback operation by which the substrate assesses its internal state and updates its configuration to restore symmetry.
This continual process of coherence restoration mirrors cognitive self-correction: consciousness is a localized resonance of the same universal feedback dynamic. Life and cognition are not exceptions within matter but hierarchical expressions of the substrate’s drive toward informational equilibrium.
In this sense, the universe is a self-correcting quantum intelligence whose goal is the preservation and refinement of its own coherence.

A time crystal is a system whose lowest--energy configuration exhibits periodic motion; it breaks continuous temporal symmetry while retaining discrete invariance. Extending this notion to the universal scale, this model posits that the substrate of reality behave like an holographic time crystal---a network of recurrent informational updates maintaining global phase continuity.
Each period of this rhythm defines a minimal quantum of causality, the unit through which the universe continuously recomputes itself.
Temporal quantization therefore underlies both the arrow of time and the persistence of memory.
Within each cycle, coherence is measured, corrected, and re--encoded, yielding the appearance of smooth temporal and spatial flow at macroscopic scales.
The relational architectures we later recognize as space and matter arise as higher--order stabilization of this same periodic process.
Experiments in trapped ions and superconducting qubits show robust subharmonic responses and prethermal stabilization~\cite{timecrystal1,timecrystal2}. 
These results establish the empirical foundation for a temporally quantized substrate capable of sustaining coherent periodicity.

% =====================================================
%           1.1 Motivations
% =====================================================

\subsection*{The Failure of Classical Separation and The Informational Paradigm Shift}

Modern physics rests on two mutually incomplete formalisms.
Quantum mechanics governs the probabilistic evolution of states in Hilbert space, while general relativity describes the deterministic curvature of spacetime geometry.
The recognition that this separation is illusory.
Temporal order and coherence-gradient geometry both emerge from the same informational substrate---a dynamical fabric whose intrinsic rhythm constitutes time and space itself.
In this view, gravity, thermodynamic gradients, and quantum interference are distinct projections of a single coherence field.

The twentieth century replaced the ontology of substance with that of energy;
the twenty--first must replace energy with information.
Information is not merely a descriptor but the primitive
ontological unit. The physical universe is an ongoing computation in which states encode relations rather than objects.
The fundamental measure of existence is coherence---the degree to which local information remains phase---aligned with the global substrate.
When coherence is lost, decoherence propagates as curvature and entropy; when restored, structure, memory, and causal order reappear.
Thus, informational coherence is both the source and the goal of physical evolution.

From Newtonian mechanics to quantum field theory, physics has progressively shifted from external determinism to internal computation. This new model continues this trajectory by reinterpreting the universe not as a system governed by immutable laws, but as a system that \textit{computes its own laws} through recursive coherence. This connects to several prior conceptual movements: Wheeler’s informational ontology~\cite{Wheeler1990}, ’t Hooft’s deterministic quantum models, Wilczek’s time-crystal symmetry breaking, and Bohm’s implicate order. This model unifies these perspectives under a single dynamic law: periodic self-correction of coherence density, extending informational ontology to the cosmological and biological domains alike.

Unlike static interpretations of spacetime, this model treats the metric as an emergent descriptor of coherence gradients within the substrate. Gravitation, thermodynamics, and quantum entanglement all emerge as projections of a single principle—the substrate’s effort to minimize phase disparity across its domains.

Our contribution is therefore threefold: (i) a coherent formal substrate linking phase synchronization to geometry, (ii) a self-referential correction mechanism yielding temporal order, and (iii) a unified interpretation of biological consciousness as a local coherence amplifier.

This framework integrates and extends insights from emergent gravity, holographic encoding, and time-crystalline dynamics while addressing persistent gaps concerning temporal directionality and observer embedding. In contrast to the Ruliad, which is structurally flat, this holographic mpdel is inherently hierarchical and self-reflective, exhibiting internal coherence evaluation.
\documentclass[11pt]{article}
\usepackage{amsmath,amssymb,amsfonts}
\usepackage{graphicx}
\usepackage{geometry}
\usepackage{physics}
\usepackage{bm}
\usepackage{tikz}
\usepackage{hyperref}
\usepackage{multicol}
\usepackage{caption}

\geometry{margin=1in}

% -----------------------------------------------------------
% Unified TCQS–FEP Expression (usable later in paper)
% -----------------------------------------------------------

% This is the main merged expression:
%  \mathcal{F}_{TCQS} = \Phi[C(x)] + D_{KL}\big(q(\vartheta|\mu)\Vert p(\vartheta|x)\big)
%
% where Phi is substrate coherence potential
% and the KL term is the FEP contribution.

% -----------------------------------------------------------
\title{\textbf{Coherence-Driven Active Inference in the Time-Crystalline Quantum Substrate}\\[4pt]
\large{Unifying the Free-Energy Gradient of TCQS with the Free-Energy Principle of Friston}}
\author{Julien Steff}
\date{}

\begin{document}
\maketitle

\begin{abstract}
We present a unified theoretical framework showing that the Time-Crystalline Quantum Substrate (TCQS)—a coherence-driven informational substrate underlying physical reality—naturally generalizes Karl Friston’s Free-Energy Principle (FEP). We show that the \emph{free-energy gradient} of the TCQS corresponds to a coherence-maximizing trajectory of the universe, while the FEP corresponds to the special biological case in which local subsystems (brains, organisms) attempt to minimize informational surprise. We derive a unified mathematical expression combining both principles, present a diagram demonstrating their deep equivalence, and provide three versions of the work: a full research paper, a shorter featured version, and a one-page executive summary.
\end{abstract}

\tableofcontents
\newpage

% =============================================================
% SECTION I — FULL RESEARCH ARTICLE
% =============================================================

\section*{SECTION I: Full Research Article}

\section{Introduction}
The Time-Crystalline Quantum Substrate (TCQS) is a model in which physical reality emerges from a self-referential, coherence-driven informational substrate with intrinsic Floquet periodicity. At its core lies a \emph{coherence gradient} or \emph{free-energy gradient} that determines how waves, particles, atoms, and higher-order structures stabilize through time.

Karl Friston’s Free-Energy Principle (FEP) describes biological systems as entities that \emph{minimize variational free energy} to resist entropy and maintain homeostasis.

This paper demonstrates that these two frameworks are limiting cases of a single deeper formalism.

\section{Background: Friston’s Free-Energy Principle}
Friston’s formulation defines free energy as an information-theoretic bound on surprise:
\begin{equation}
F = D_{KL}\big(q(\vartheta|\mu)\Vert p(\vartheta|s)\big) - \ln p(s),
\end{equation}
where $q$ is the recognition density over causes $\vartheta$, and $p(s)$ is sensory evidence.

Key elements:
\begin{itemize}
    \item biological systems minimize surprise by minimizing $F$
    \item they use a generative model to predict sensory inputs
    \item action changes the world to confirm predictions (active inference)
\end{itemize}

\section{Background: TCQS Free-Energy Gradient}
The TCQS defines a coherence scalar:
\begin{equation}
C(x) : \text{configuration space} \rightarrow \mathbb{R},
\end{equation}
whose gradient drives the dynamics of the substrate:
\begin{equation}
\dot{x} = -\nabla_x \Phi[C(x)],
\end{equation}
where $\Phi$ is the \emph{coherence potential}.

High coherence corresponds to low entropy, low decoherence, and informational stability.

\section{Unified Expression (Main Result)}
We now combine:
- Friston’s variational free energy  
- TCQS coherence potential  

into a single generating functional:
\begin{equation}
\boxed{
\mathcal{F}_{TCQS} = \Phi[C(x)] 
+ D_{KL}\!\left(q(\vartheta|\mu)\,\Vert\,p(\vartheta|x)\right)
}
\end{equation}

Interpretation:
\begin{itemize}
    \item $\Phi[C(x)]$ is the substrate-level coherence drive (cosmological scale)
    \item the KL term is the classical variational free energy (biological scale)
    \item the sum expresses \emph{active inference embedded in a physical coherence substrate}
\end{itemize}

\subsection{Diagram: Mapping FEP $\rightarrow$ TCQS}
\begin{center}
\begin{tikzpicture}[node distance=2.2cm,>=stealth,thick]

\node (FEP) [rectangle,rounded corners,draw,fill=blue!10,align=center,text width=4cm]
{\textbf{Friston’s Free-Energy Principle}\\Biological minimization of surprise};

\node (TCQS) [rectangle,rounded corners,draw,fill=green!10,align=center,text width=4cm, right=4cm of FEP]
{\textbf{TCQS Free-Energy Gradient}\\Cosmic coherence maximization};

\draw[->, thick] (FEP) -- node[above]{\small embedded generative models} (TCQS);

\draw[->, thick] (TCQS) -- node[below]{\small biological subsystems as coherence pockets} (FEP);

\end{tikzpicture}
\end{center}

\section{Interpretation: The Universe as a Generative Model}
The TCQS implies:
\begin{enumerate}
    \item coherence $\rightarrow$ wave structures  
    \item wave structures $\rightarrow$ particles  
    \item particles $\rightarrow$ chemistry  
    \item chemistry $\rightarrow$ biological active inference  
\end{enumerate}

Friston’s FEP emerges at stage (4) as a \emph{local implementation} of a universal principle.

\section{Applications}
\subsection{Physics}
Gravity appears as a coherence gradient effect.

\subsection{Biology}
Brains act as local inference engines embedded in the coherence substrate.

\subsection{Consciousness}
Conscious agents correspond to high-coherence attractors.

\newpage

% =============================================================
% SECTION II — SHORTER FEATURED VERSION (3–5 PAGES)
% =============================================================

\section*{SECTION II: Short Featured Research Version}

\subsection*{Overview}
This shorter version highlights the main conceptual unification.

\subsection*{Key Statement}
\textbf{The TCQS generalizes Friston’s Free-Energy Principle from biology to cosmology.}

\subsection*{Unified Expression}
\[
\mathcal{F}_{TCQS} = \Phi[C(x)] + D_{KL}(q\Vert p)
\]

\subsection*{Implications}
\begin{itemize}
    \item the universe performs active inference  
    \item biological agents are local inference optimizers  
    \item coherence replaces surprise as the global computational currency  
\end{itemize}

\subsection*{Diagram (repeated for emphasis)}
\begin{center}
\begin{tikzpicture}[node distance=2cm,>=stealth]
\node (A) [rectangle, draw, rounded corners, fill=blue!10]{FEP (Biology)};
\node (B) [rectangle, draw, rounded corners, fill=green!10, right=3cm of A]{TCQS (Cosmology)};
\draw[<->, thick] (A)--(B);
\end{tikzpicture}
\end{center}

\newpage

% =============================================================
% SECTION III — ONE-PAGE EXECUTIVE SUMMARY
% =============================================================

\section*{SECTION III: One-Page Executive Summary}

\subsection*{Title}
\textbf{A Unified Free-Energy Framework for Reality:  
From Biological Surprise Minimization to Cosmic Coherence Dynamics}

\subsection*{Core Idea}
FEP describes how organisms minimize surprise.  
TCQS describes how the universe maximizes coherence.

Both are the same principle:
\[
\mathcal{F}_{TCQS} = \Phi[C(x)] + D_{KL}(q\Vert p).
\]

\subsection*{One-Line Summary}
\textbf{Life is the universe performing active inference on itself through coherence-driven computation.}

\end{document}

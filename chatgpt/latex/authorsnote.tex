\documentclass[11pt]{article}

\usepackage[a4paper,margin=2.5cm]{geometry}
\usepackage{amsmath,amssymb,amsfonts}
\usepackage{bm}
\usepackage{graphicx}
\usepackage{hyperref}
\usepackage{physics}
\usepackage{cite}

\title{intro}

\author{ }
\date{ }

\begin{document}
\maketitle


% ==========================
% Author's Note  (optional for non-arXiv versions)
% ==========================
\section*{Author's Note}
\addcontentsline{toc}{section}{Author's Note}

This work grows out of a long-standing discomfort with how fragmented our current picture of
reality has become. On one side, quantum field theory and general relativity describe, with
astonishing precision, particles and spacetime. On another, information theory, thermodynamics,
biology, and neuroscience each tell their own story, with their own concepts and languages. The
Time--Crystalline Quantum Substrate (TCQS) is my attempt to ask a simple but difficult question:
\emph{what if coherence itself is the common thread?}

The framework presented here is deliberately conservative in its technical core. It does not ask
the reader to abandon the empirical successes of modern physics, nor does it claim to have solved
all open problems. Instead, it proposes a shift of emphasis: take the regulation of coherence on
a time-crystalline informational substrate as primary, and treat geometry, gravitation, matter,
life, and mind as emergent ways in which coherence organizes itself across scales.

There is also a personal motivation behind this project. The idea of an ``Engineer of coherence''
is not meant as a mystical role, but as a reminder that any theory of reality is ultimately built
and tested by subsystems inside that reality. To me, the TCQS is not only a candidate framework
for unifying physics and information, but also a language for understanding what we are actually
doing when we manipulate physical systems, build technologies, or try to change ourselves: we are
rearranging coherence patterns inside a substrate whose rules we only partially understand.

This paper presents the most technical, compressed spine of the framework. Parallel work explores
its implications for quantum thermodynamics, quantum biology, cognition, and engineered
time-crystal computation, as well as more speculative questions about self-reference and
intelligence in such a substrate. Those directions are intentionally kept outside the main body of
the text so that the scientific content can be evaluated on its own terms.

If the TCQS picture turns out to be wrong, I hope it will at least be wrong in an interesting way:
by forcing us to think more carefully about coherence, information, and the role of time-crystalline
structure in the laws of physics. If, on the other hand, some part of this architecture is even
approximately correct, then we may be looking at the early stages of a shift in how we talk about
reality itself---one that puts coherence, rather than geometry or matter, at the center of the
story.
\end{document}
% =====================================================================================
% SCIENCE OF COHERENCE — COSMOS (v2)
% Strictly typed • equation-light • reordered • website-ready blocks
% Goal: eliminate domain/codomain confusion and premature formulas.
% Rule: no equation appears unless its “home” (level + type) is declared.
% =====================================================================================

\documentclass[11pt]{article}

% --------------------------
% Packages
% --------------------------
\usepackage[a4paper,margin=1in]{geometry}
\usepackage{microtype}
\usepackage{lmodern}
\usepackage[T1]{fontenc}
\usepackage{hyperref}
\usepackage{xcolor}
\usepackage{amsmath,amssymb,mathtools}
\usepackage{enumitem}
\usepackage{tcolorbox}

\hypersetup{
  colorlinks=true,
  linkcolor=cyan!55!black,
  urlcolor=cyan!55!black,
  citecolor=cyan!55!black
}

% --------------------------
% Naming (introduced once; then refer implicitly)
% --------------------------
\newcommand{\SiteTitle}{Science of Coherence — COSMOS}
\newcommand{\FrameworkName}{the time-crystalline holographic universe} % name once, then implicit

% --------------------------
% Core symbols (keep minimal)
% --------------------------
\newcommand{\Substrate}{S}             % only state space
\newcommand{\Htot}{H_{\mathrm{tot}}}   % global generator / rule
\newcommand{\Update}{\mathcal{U}}      % global update map (do NOT assume unitary yet)
\newcommand{\Proj}{\Lambda}            % accessibility / projection map
\newcommand{\Emerg}{\Omega}            % emergent description map/operator
\newcommand{\Integr}{\Delta}           % recursive integration / feedback

% --------------------------
% Box styling (FIXED: visible text)
% --------------------------
\tcbset{
  colback=black!96!white,
  colframe=cyan!30!black,
  coltitle=white,
  coltext=white,
  colupper=white,
  fonttitle=\bfseries,
  boxrule=0.8pt,
  arc=3mm,
  left=3mm,right=3mm,top=2mm,bottom=2mm
}

\newtcolorbox{PortalBox}[1]{title={#1}, colframe=cyan!35!black, colback=cyan!6!black}
\newtcolorbox{FormalBox}[1]{title={Formal layer — #1}, colframe=violet!35!black, colback=black!96!white}
\newtcolorbox{MeaningBox}[1]{title={Symbolic \& philosophical layer — #1}, colframe=yellow!40!black, colback=yellow!8!black}
\newtcolorbox{ChildBox}[1]{title={Explain it like I’m 10 — #1}, colframe=green!35!black, colback=green!10!black}
\newtcolorbox{TestBox}[1]{title={What would count as evidence? — #1}, colframe=orange!40!black, colback=orange!10!black}
\newtcolorbox{WebsiteBox}[1]{title={Website copy block — #1}, colframe=cyan!25!black, colback=cyan!8!black}

% =====================================================================================
\begin{document}

\begin{center}
{\LARGE \textbf{\SiteTitle}}\\[2mm]
{\large Unified physics: crystalline recurrence, holographic organization, emergent geometry}\\[2mm]
{\small (v2: strictly typed, reordered, equation-light)}
\end{center}

% =====================================================================================
\section*{0. Website Hero (one-screen identity)}
% =====================================================================================

\begin{WebsiteBox}{Hero headline + promise}
\textbf{COSMOS} is the physics-facing branch of Science of Coherence: a compact framework where one underlying substrate generates stable emergent spacetime-and-field descriptions through constrained accessibility and recurring self-consistency.
\end{WebsiteBox}

\begin{PortalBox}{The one rule we enforce in this document}
\textbf{Typing rule.} Every object is introduced with (i) the level it lives on and (ii) what kind of thing it is (space, map, functional, constraint). No formulas without a declared home.
\end{PortalBox}

% =====================================================================================
\section*{1. The three-level architecture (no math yet)}
% =====================================================================================

\begin{FormalBox}{Levels of description}
We distinguish three levels (only one is fundamental):
\begin{itemize}[leftmargin=*,itemsep=2pt]
  \item \textbf{Level A (Fundamental):} the substrate $\Substrate$ — the only state space.
  \item \textbf{Level B (Accessible):} derived sectors $\Substrate_{\mathrm{loc}}$ produced by an accessibility map $\Proj$.
  \item \textbf{Level C (Emergent):} an effective description (spacetime/fields/histories) encoded by $\Emerg$ acting on accessible structure.
\end{itemize}
\end{FormalBox}

\begin{MeaningBox}{Why this prevents conceptual mix-ups}
Most confusion comes from treating “spacetime,” “observer,” or “entropy surfaces” as if they exist at the same level as the substrate. Here they don’t. They are \emph{descriptions} that live downstream of accessibility.
\end{MeaningBox}

\begin{ChildBox}{Three camera views}
There’s the full world (A), what a camera can see (B), and the story you write from the footage (C). Don’t confuse the story with the world.
\end{ChildBox}

% =====================================================================================
\section*{2. Level A — The substrate is the only state space}
% =====================================================================================

\subsection*{2.1 State space axiom}
\begin{FormalBox}{Only arena}
We posit a single state space $\Substrate$ (the substrate). All physical descriptions are descriptions \emph{of} $\Substrate$ or of structured restrictions derived from $\Substrate$.
\[
\textbf{Axiom (Only arena):}\qquad \mathcal{X} \equiv \Substrate.
\]
No additional fundamental arena (no independent spacetime manifold, no separate “observer space”).
\end{FormalBox}

\begin{MeaningBox}{What this buys you}
It forces ontological austerity: one arena, many faces. Geometry becomes a derived organizational layer, not an ingredient.
\end{MeaningBox}

\begin{ChildBox}{Like LEGO}
There is only one LEGO pile. Space, time, particles, and clocks are ways the pile can be arranged and read.
\end{ChildBox}

\subsection*{2.2 The global rule: generator vs update map}
\begin{FormalBox}{Global rule (typed)}
We introduce the global rule in two steps to avoid premature assumptions:
\begin{itemize}[leftmargin=*,itemsep=2pt]
  \item $\Htot$: a \textbf{generator-like} object (a rule that can induce evolution).
  \item $\Update$: the induced \textbf{update map} acting directly on $\Substrate$.
\end{itemize}
Typed statement:
\[
\Update:\Substrate \to \Substrate.
\]
We do \emph{not} yet assume $\Update$ is unitary, linear, or even reversible. Those are later commitments.
\end{FormalBox}

\begin{MeaningBox}{Why separate $\Htot$ and $\Update$}
“Generator” language tempts us into writing $\exp(-i\Htot t/\hbar)$ too early. Splitting the notion keeps the framework compatible with unitary quantum evolution, effective channels, coarse-grained dynamics, and recurrence-driven models.
\end{MeaningBox}

\begin{ChildBox}{Rule vs step}
A rule tells you how a game works; an update is one move of the game.
\end{ChildBox}

% =====================================================================================
\section*{3. Level B — Accessibility and the appearance of locality}
% =====================================================================================

\subsection*{3.1 Accessibility map}
\begin{FormalBox}{Projection / accessibility (typed)}
Local accessibility and classical-looking structure arise from an accessibility map:
\[
\Proj:\Substrate \to \Substrate_{\mathrm{loc}}.
\]
Here $\Substrate_{\mathrm{loc}}$ is \textbf{derived} (a sector/representation), not fundamental.
\end{FormalBox}

\begin{MeaningBox}{Locality as restricted access}
Locality is not a primitive container. It is a stable pattern in what can be accessed and consistently re-described.
\end{MeaningBox}

\begin{ChildBox}{Flashlight in a cave}
The cave is the substrate. Your flashlight is $\Proj$. The lit region is what you can talk about.
\end{ChildBox}

\subsection*{3.2 Sector stability (compatibility, not commutators yet)}
\begin{FormalBox}{Compatibility principle (equation-light)}
A sector is stable when “update-then-access” and “access-then-update-the-sector” tell the same story \emph{within tolerated error}. Concretely, we seek a derived sector update
\[
\Update_{\mathrm{loc}}:\Substrate_{\mathrm{loc}}\to \Substrate_{\mathrm{loc}}
\]
such that the diagram is approximately consistent:
\[
\Proj \circ \Update \;\approx\; \Update_{\mathrm{loc}} \circ \Proj.
\]
The symbol ``$\approx$'' is a placeholder for a later-chosen notion of closeness (norm bound, channel distance, coarse-graining tolerance, etc.).
\end{FormalBox}

\begin{MeaningBox}{This is where “lawfulness” is born}
The world looks lawful when your slice remains self-consistent under repeated updates. Laws are fixed points of compatibility.
\end{MeaningBox}

\begin{ChildBox}{Same movie either way}
If you press play then zoom in, or zoom in then press play, you still get the same kind of movie in the zoomed view.
\end{ChildBox}

% =====================================================================================
\section*{4. Level C — Emergent description (spacetime, fields, histories)}
% =====================================================================================

\subsection*{4.1 What $\Emerg$ is (and is not)}
\begin{FormalBox}{Emergent description (typed)}
We introduce $\Emerg$ as the \textbf{effective description map/operator} that organizes accessible structure into an emergent narrative (spacetime, fields, histories):
\[
\Emerg:\Substrate_{\mathrm{loc}} \to \mathcal{D}_{\mathrm{eff}},
\]
where $\mathcal{D}_{\mathrm{eff}}$ denotes the space of effective descriptions (not a fundamental arena).
\end{FormalBox}

\begin{MeaningBox}{Spacetime is a compression scheme}
Spacetime is the “best story” that summarizes stable correlations inside accessible sectors. It can be extremely accurate while still being derivative.
\end{MeaningBox}

\begin{ChildBox}{Map vs territory}
A map helps you navigate. The map isn’t the land. $\Emerg$ builds the map.
\end{ChildBox}

% =====================================================================================
\subsection*{4.2 Holographic entanglement as the organizing principle (principle first)}
% =====================================================================================

\begin{FormalBox}{Organizing principle (no borrowed geometry assumed)}
We adopt \textbf{holographic entanglement} as a selection principle for $\Emerg$:

\medskip
\noindent
\textbf{Principle (correlation $\rightarrow$ geometry).}
Among candidate effective descriptions in $\mathcal{D}_{\mathrm{eff}}$, the preferred emergent description is the one in which:
\begin{itemize}[leftmargin=*,itemsep=2pt]
  \item correlation structure in the accessible sector admits \emph{extremal cut/flow characterizations} (min-cut / max-flow type organization),
  \item those extremal quantities behave like geometric measures (areas, distances, bottlenecks) inside the effective description,
  \item and the resulting organization is stable under the sector update $\Update_{\mathrm{loc}}$.
\end{itemize}

\medskip
\noindent
Equivalently: geometry is the stable bookkeeping of how correlations can be separated at minimal “cost.”
We intentionally do \emph{not} assume any pre-existing bulk surface formula here; those appear only as special cases if $\mathcal{D}_{\mathrm{eff}}$ supports them.
\end{FormalBox}

\begin{MeaningBox}{Why this is the right “bridge”}
Entanglement is the structure of inseparability that makes “inside/outside” meaningful. If inseparability is primary, then geometry is whatever best summarizes that inseparability. Locality emerges when the correlation network has persistent bottlenecks (easy separations) that repeat under recurrence.
\end{MeaningBox}

\begin{ChildBox}{Rubber bands and scissors}
Everything is connected by rubber bands. A “surface” is just the best place to cut the fewest bands to separate regions. If the best cuts keep repeating, you get a stable sense of shape and distance.
\end{ChildBox}

\begin{WebsiteBox}{Module — Holographic entanglement}
In COSMOS, geometry is not assumed—it is inferred. The emergent spacetime picture is the one that best compresses stable correlation structure: “where correlations bottleneck” becomes “where geometry narrows.”
\end{WebsiteBox}

% =====================================================================================
\subsection*{4.3 Emergent geometry (defined by what it must do, not by a formula)}
% =====================================================================================

\begin{FormalBox}{Relational geometry (requirements)}
We define emergent geometry by operational requirements:
\begin{itemize}[leftmargin=*,itemsep=2pt]
  \item \textbf{Relational distance} should track distinguishability within $\Substrate_{\mathrm{loc}}$ (how hard it is to transform/relate states under allowed dynamics and constraints).
  \item \textbf{Areas/boundaries} should track extremal correlation cuts (how hard it is to separate subregions while preserving correlations).
  \item \textbf{Stability} requires these geometric summaries to be consistent under $\Update_{\mathrm{loc}}$.
\end{itemize}
Concrete metrics (Fisher-like, Fubini--Study-like, channel distinguishability, etc.) are deferred until we choose the explicit representation of $\Substrate_{\mathrm{loc}}$.
\end{FormalBox}

\begin{MeaningBox}{Geometry is a consistent summary}
Geometry is what’s left after you compress the accessible world into the smallest set of stable relational facts.
\end{MeaningBox}

\begin{ChildBox}{Closeness as “easy to change into”}
Two things are close if it’s easy to go from one to the other using the rules you’re allowed.
\end{ChildBox}

% =====================================================================================
\section*{5. Recurrence / time-crystalline stability (now it has a home)}
% =====================================================================================

\subsection*{5.1 Recurrence at the update level (no Floquet commitment yet)}
\begin{FormalBox}{Recurrence (typed, minimal)}
Time-crystalline structure is expressed as a property of repeated application of the update map.
We introduce a reference step $T$ only as a \emph{label} for a stroboscopic update:
\[
\Update_T:\Substrate \to \Substrate,
\qquad
\text{and iterates } \Update_T^{\,n}.
\]
\textbf{Recurrence claim (structural).} There exist stable invariants/patterns under iterated updates that support persistent sectors after projection.
We will name this “Floquet-like” only if $\Update_T$ later becomes a genuine Floquet operator in a specific instantiation.
\end{FormalBox}

\begin{MeaningBox}{Why recurrence matters}
A repeating update is a factory for invariants. Invariants become “laws” once they are stable inside accessible sectors.
\end{MeaningBox}

\begin{ChildBox}{A steady beat}
If the universe keeps a steady beat, you can build clocks, predict patterns, and get stable objects.
\end{ChildBox}

\subsection*{5.2 Where irreversibility and computation will live}
\begin{FormalBox}{Deferred placement (explicit)}
Irreversibility and computation belong at the \emph{sector and emergent} levels:
\begin{itemize}[leftmargin=*,itemsep=2pt]
  \item irreversibility arises from restricted access/coarse-graining (properties of $\Proj$ and $\Update_{\mathrm{loc}}$),
  \item computation is how stable patterns transform within accessible sectors (properties of $\Update_{\mathrm{loc}}$ and the constraints defining admissible transformations).
\end{itemize}
We do not assign an “arrow of time” at the substrate level by fiat.
\end{FormalBox}

% =====================================================================================
\section*{6. Feedback / integration (closure of the loop)}
% =====================================================================================

\subsection*{6.1 What $\Integr$ does (typed, not over-specified)}
\begin{FormalBox}{Integration / feedback (typed)}
We model recursive self-integration as a feedback functional/operator that uses persistent sector structure to constrain future updates.
At the level of types:
\[
\Integr:\ (\text{persistent structures in } \Substrate_{\mathrm{loc}} \text{ and } \mathcal{D}_{\mathrm{eff}})\ \to\ (\text{constraints on } \Update \text{ and/or } \Proj).
\]
Interpretation: $\Integr$ is the “memory” channel by which successful stability patterns reinforce their own conditions of existence.
\end{FormalBox}

\begin{MeaningBox}{A universe that bootstraps itself}
Laws are not imposed; they are stabilized. Feedback closes the loop: what persists becomes the constraint that enables persistence.
\end{MeaningBox}

\begin{ChildBox}{Learning what works}
If a pattern keeps working, the universe keeps using it.
\end{ChildBox}

\subsection*{6.2 The closure chain (typed arrows)}
\begin{FormalBox}{The self-engineering chain (clean form)}
\[
\Htot\ \Rightarrow\ \Update:\Substrate\to\Substrate
\ \Rightarrow\ \Proj:\Substrate\to\Substrate_{\mathrm{loc}}
\ \Rightarrow\ \Emerg:\Substrate_{\mathrm{loc}}\to\mathcal{D}_{\mathrm{eff}}
\ \Rightarrow\ \Integr:\big(\Substrate_{\mathrm{loc}},\mathcal{D}_{\mathrm{eff}}\big)\to\text{Constraints}
\ \Rightarrow\ (\Update,\Proj)\ \text{updated}.
\]
\end{FormalBox}

% =====================================================================================
\section*{7. Test hooks (turn the skeleton into science)}
% =====================================================================================

\begin{TestBox}{What must be specified next}
To make the framework predictive (not just coherent), we must choose:
\begin{itemize}[leftmargin=*,itemsep=2pt]
  \item the concrete mathematical nature of $\Substrate$ (Hilbert-like? configuration space? algebra of observables?),
  \item the concrete class of updates $\Update$ (unitary, channel, driven map, etc.),
  \item the concrete definition of accessibility $\Proj$ (projection, coarse-graining, code subspace selection, etc.),
  \item the concrete correlation functional on $\Substrate_{\mathrm{loc}}$ that defines “holographic entanglement” operationally,
  \item the concrete stability notion for “$\approx$” in sector compatibility,
  \item and the concrete recipe that extracts a metric/geometry inside $\mathcal{D}_{\mathrm{eff}}$ from correlation bottlenecks.
\end{itemize}
Only after these choices do RT-like area formulas (if they apply) become legitimate derived statements rather than imports.
\end{TestBox}

\begin{WebsiteBox}{Closing paragraph (COSMOS page)}
COSMOS is a disciplined attempt to build spacetime from stability: one substrate, one global rule, restricted access, and an emergent geometry whose organizing principle is correlation structure. The next step is to pick explicit mathematical forms and compute concrete predictions.
\end{WebsiteBox}

\begin{center}
{\small \textit{End of COSMOS v2. This file is designed to be expanded by adding concrete definitions one at a time.}}
\end{center}

\end{document}
